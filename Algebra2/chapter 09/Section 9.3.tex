\documentclass[addpoints,12pt]{exam}
\usepackage{amsmath}
\usepackage{amsthm}
\usepackage{amsfonts}
\usepackage{systeme}
\usepackage{graphicx}
\usepackage{caption}
\usepackage{xfrac}
\usepackage{physics}
\usepackage{microtype}
\usepackage{eulervm}
%\usepackage[framemethod=tikz]{mdframed}
\usepackage{thmtools}
\usepackage{etoolbox}
%\usepackage{fouriernc}
\usepackage{mdframed}
\usepackage[explicit]{titlesec}

\pagestyle{headandfoot}
\runningfootrule
\firstpageheadrule
\runningheadrule

\newcommand{\class}{Math 0098}
\newcommand{\sem}{2201}
\newcommand{\due}{}
\newcommand{\sect}{9.3}
\newcommand{\topic}{Equations and Inequalities with Absolute Values}

\firstpageheader{\class}{\sect - \topic}{}
\runningheader{\class}{\sect - \topic}{}
\firstpagefooter{\class}{}{Page \thepage\ of \numpages}
\runningfooter{\class}{}{Page \thepage\ of \numpages}

\newif\ifprintselected
\printselectedtrue
%\printselectedfalse

\newenvironment{select}
{\ifprintselected
	\printanswers
	\fi
}
{}

\theoremstyle{definition}
\newtheorem{theorem}{Theorem}
\newtheorem{example}{Example}[subsection]
%\newtheorem{definition}{Definition}
\newtheorem{definition}{Definition}[subsection]
%\newmdtheoremenv{example}{Example}[subsection]
\AtBeginEnvironment{defn}{\begin{minipage}{\textwidth}}
\AtEndEnvironment{defn}{\end{minipage}}
%\AtBeginEnvironment{example}{\begin{minipage}{\textwidth}}
%\AtEndEnvironment{example}{\end{minipage}}
\newcommand{\iu}{{i\mkern1mu}}

\setlength{\gridsize}{5mm}
\setlength{\gridlinewidth}{0.1pt}

\printanswers
\DeclareMathSizes{12}{12}{12}{12}

\titleformat{\subsubsection}
   {\large\bfseries}% format
   {}% label
   {0pt}% sep
   {\titlerule \vspace{.1in} #1}% before code
      [{\titlerule[0.4pt]\vspace{.1in}}]% after code
\titlespacing{\subsubsection}
   {0pt}% left
   {0pt}% before sep
   {\baselineskip}% after sep


\begin{document}
\setcounter{section}{9}
\setcounter{subsection}{2}

\subsection{Equations and Inequalities with Absolute Values}

\subsubsection*{Equations with Absolute Values}

Recall that the absolute value of a number, $\abs{x}$, represents the distance from 0 for that number. Essentially, it takes any number and turns it positive. If we have an equation involving absolute values, we typically end up with 2 values as answers - why?
\vspace{.75in}

\begin{mdframed}
\textbf{Rewriting Absolute Value Equations}

If $c$ is some positive number and $u$ is an algebraic expression, then we can rewrite $\abs{u} = c$ as $u = c \text{ or } u = -c$.
\end{mdframed}

\vspace{.2in}
\begin{example}
Solve:
\[\abs{2x-1} = 5\]
\vspace{1.5in}
\end{example}

\begin{example}
Solve: \[2\abs{1-3x}-28 = 0\]
\vspace{1.5in}
\end{example}

\newpage

\begin{mdframed}
\textbf{Rewriting Equations with 2 Absolute Values}

If $u$ and $v$ are both some algebraic expression, then we can rewrite $\abs{u} = \abs{v}$ as $u = v \text{ or } u = -v$.
\end{mdframed}

\vspace{.2in}

\begin{example}
Solve: \[ \abs{2x-7} = \abs{x+3}\]
\vspace{2in}
\end{example}

\subsubsection*{Inequalities with Absolute Values}

\begin{mdframed}
\textbf{Inequalities of the form $\abs{u} < c$}

If $c$ is a positive number and $u$ is an algebraic expression, then $\abs{u} < c$ can be rewritten as the compound inequality $-c < u < c$.
\end{mdframed}

\begin{example}
Solve: \[ \abs{x-2} < 5\]
\vspace{1.5in}
\end{example}

\newpage

\begin{example}
Solve: \[-3\abs{5x-2} + 20 \ge -19\]
\vspace{2in}
\end{example}

\begin{mdframed}
\textbf{Inequalities of the form $\abs{u} > c$}

If $c$ is a positive number and $u$ is an algebraic expression, then $\abs{u} > c$ can be rewritten as $u < -c \text{ or } u > c$.
\end{mdframed}

\vspace{.2in}

\begin{example}
Solve: \[\abs{2x-5}\ge 3\]
\vspace{1.5in}
\end{example}



\end{document}