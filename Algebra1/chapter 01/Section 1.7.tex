\documentclass[addpoints,12pt]{exam}
\usepackage{amsmath}
\usepackage{amsthm}
\usepackage{amsfonts}
\usepackage{systeme}
\usepackage{graphicx}
\usepackage{caption}
\usepackage{xfrac}
\usepackage{physics}
\usepackage{microtype}
\usepackage{eulervm}
%\usepackage[framemethod=tikz]{mdframed}
\usepackage{thmtools}
\usepackage{etoolbox}
%\usepackage{fouriernc}
\usepackage{mdframed}
\usepackage[overload]{empheq}
\usepackage{adjustbox}
\usepackage{enumitem}
\usepackage[explicit]{titlesec}

\pagestyle{headandfoot}
\runningfootrule
\firstpageheadrule
\runningheadrule

\newcommand{\class}{Math 0097}
\newcommand{\sem}{2211}
\newcommand{\due}{}
\newcommand{\sect}{1.7}
\newcommand{\topic}{Multiplication \& Division of Real Numbers}

\firstpageheader{\class}{\sect - \topic}{}
\runningheader{\class}{\sect - \topic}{}
\firstpagefooter{\class}{}{Page \thepage\ of \numpages}
\runningfooter{\class}{}{Page \thepage\ of \numpages}

\newif\ifprintselected
\printselectedtrue
%\printselectedfalse

\newenvironment{select}
{\ifprintselected
	\printanswers
	\fi
}
{}

\theoremstyle{definition}
\newtheorem{theorem}{Theorem}
\newtheorem{example}{Example}[subsection]
%\newtheorem{definition}{Definition}
\newtheorem{definition}{Definition}[subsection]
%\newmdtheoremenv{example}{Example}[subsection]
\AtBeginEnvironment{defn}{\begin{minipage}{\textwidth}}
\AtEndEnvironment{defn}{\end{minipage}}
%\AtBeginEnvironment{example}{\begin{minipage}{\textwidth}}
%\AtEndEnvironment{example}{\end{minipage}}
\newcommand{\iu}{{i\mkern1mu}}

\setlength{\gridsize}{5mm}
\setlength{\gridlinewidth}{0.1pt}

\printanswers
\DeclareMathSizes{12}{12}{12}{12}

\titleformat{\subsubsection}
   {\large\bfseries}% format
   {}% label
   {0pt}% sep
   {\titlerule \vspace{.1in} #1}% before code
      [{\titlerule[0.4pt]\vspace{.1in}}]% after code
\titlespacing{\subsubsection}
   {0pt}% left
   {0pt}% before sep
   {\baselineskip}% after sep


\begin{document}
\setcounter{section}{1}
\setcounter{subsection}{6}

\subsection{Multiplication \& Division of Real Numbers}

\vspace{.25in}

\begin{mdframed}
\textbf{Sign Rules for Multiplication}
\begin{itemize}
\item positive $\times$ positive $=$ positive
\item negative $\times$ negative $=$ positive
\item positive $\times$ negative $=$ negative
\item negative $\times$ positive $=$ negative
\item $a \cdot 0 = 0$
\item $0 \cdot a = 0$
\end{itemize}
\noindent \emph{Same signs - positive; different signs - negative}
\end{mdframed}
\vspace{.15in}

\begin{mdframed}
\textbf{Sign Rules for Division}
\begin{itemize}
\item positive $\divisionsymbol$ positive $=$ positive
\item negative $\divisionsymbol$ negative $=$ positive
\item positive $\divisionsymbol$ negative $=$ negative
\item negative $\divisionsymbol$ positive $=$ negative
\item $0 \divisionsymbol a = 0$
\item $a \divisionsymbol 0 = $ undefined
\end{itemize}
\noindent \emph{Same signs - positive; different signs - negative}
\end{mdframed}

\vspace{.15in}
\noindent What if there are more than two numbers involved? \emph{If there are an even number of negatives, then the result is positive. An odd number of negatives gives a negative result.}

\newpage

\begin{example}
Find $(-2)(3)(-1)(4) = $
\vspace{.5in}
\end{example}

\begin{example}
Find $(-37)(423)(0)(-55)(-3.7) = $
\vspace{.5in}
\end{example}

\begin{mdframed}
\textbf{Other Properties of Multiplication}
\begin{itemize}
\item Identity: $a\cdot 1 = 1\cdot a = a$
\item Inverse: $a \cdot \dfrac{1}{a} = \dfrac{1}{a}\cdot a = 1$
\end{itemize}
\end{mdframed}
\vspace{.15in}
\noindent Division is defined using multiplication. How so?\\

\noindent Let $a$ and $b$ be real numbers with $b \neq 0$, then: \[a\divisionsymbol b = \dfrac{a}{b} = a \cdot \dfrac{1}{b}\]
\\
\subsubsection*{Negatives with Parentheses}

\noindent When there is a negative sign in front of a quantity (set of parentheses), distribute a negative one to each term.

\begin{example}
Simplify fully: \[4(3y-7) - (13y-2)\]
\end{example}

\newpage

\begin{example}
Simplify fully: \[-4(-\dfrac{3}{4}y)\]
\vspace{1in}
\end{example}

\begin{example}
Simplify fully: \[4(2y-3)-(7y+2)\]
\vspace{1in}
\end{example}

\begin{example}
Is $x = -8$ a solution to the following equation? \[4(6-x) + 7x = 0\]
\vspace{1in}
\end{example}

\begin{example}
Is $m = -4$ a solution to the following equation? \[\dfrac{5m-1}{6} = \dfrac{3m-2}{4}\]
\end{example}

\end{document}