\documentclass[addpoints,12pt]{exam}
\usepackage{amsmath}
\usepackage{amsthm}
\usepackage{amsfonts}
\usepackage{systeme}
\usepackage{graphicx}
\usepackage{caption}
\usepackage{xfrac}
\usepackage{physics}
\usepackage{microtype}
\usepackage{eulervm}
%\usepackage[framemethod=tikz]{mdframed}
\usepackage{thmtools}
\usepackage{etoolbox}
%\usepackage{fouriernc}
\usepackage{mdframed}
\usepackage[overload]{empheq}
\usepackage{adjustbox}
\usepackage{enumitem}
\usepackage[explicit]{titlesec}
% adds in \varnothing for empty set
\usepackage{amssymb}
% adds in formated SI units
%\usepackage{siunitx}
\usepackage{pgfplots}
\usepackage{multirow}
\usepackage{array}

\pagestyle{headandfoot}
\runningfootrule
\firstpageheadrule
\runningheadrule

\newcommand{\class}{Math 0097}
\newcommand{\sem}{2211}
\newcommand{\due}{}
\newcommand{\sect}{6.4}
\newcommand{\topic}{Factoring Special Forms}

\firstpageheader{\class}{\sect - \topic}{}
\runningheader{\class}{\sect - \topic}{}
\firstpagefooter{\class}{}{Page \thepage\ of \numpages}
\runningfooter{\class}{}{Page \thepage\ of \numpages}

\newif\ifprintselected
\printselectedtrue
%\printselectedfalse

\newenvironment{select}
{\ifprintselected
	\printanswers
	\fi
}
{}

\theoremstyle{definition}
\newtheorem{theorem}{Theorem}
%\newtheorem{example}{Example}[subsection]
%\newtheorem{definition}{Definition}
%\newmdtheoremenv{definition}{Definition}[subsection]
%\newmdtheoremenv{example}{Example}[subsection]
\AtBeginEnvironment{defn}{\begin{minipage}{\textwidth}}
\AtEndEnvironment{defn}{\end{minipage}}
%\AtBeginEnvironment{example}{\begin{minipage}{\textwidth}}
%\AtEndEnvironment{example}{\end{minipage}}
\newcommand{\iu}{{i\mkern1mu}}

\setlength{\gridsize}{5mm}
\setlength{\gridlinewidth}{0.1pt}

\printanswers
\DeclareMathSizes{12}{12}{12}{12}

%%%%%%%%%%%%%%%%%%%%%%%%
% Create bars around subsubsection
%%%%%%%%%%%%%%%%%%%%%%%%

\titleformat{\subsubsection}
   {\large\bfseries}% format
   {}% label
   {0pt}% sep
   {\titlerule \vspace{.1in} #1}% before code
      [{\titlerule[0.4pt]\vspace{.1in}}]% after code
\titlespacing{\subsubsection}
   {0pt}% left
   {0pt}% before sep
   {\baselineskip}% after sep
   
%%%%%%%%%%%%%%%%%%%%%%%
% Create line break after definition label
%%%%%%%%%%%%%%%%%%%%%%%   
\newtheoremstyle{break}
  {\topsep}{\topsep}%
  {}{}%\itshape
  {\bfseries}{}%
  {\newline}{}%
\theoremstyle{break}
\newmdtheoremenv{definition}{Definition}[subsection]
\theoremstyle{break}
\newtheorem{example}{Example}[subsection]

%%%%%%%%%%%%%%%%%%%%%%
% start document
% set section, subsection (use n-1 for sub)
%%%%%%%%%%%%%%%%%%%%%%


\begin{document}
\setcounter{section}{6}
\setcounter{subsection}{3}

\subsection{Factoring Special Forms}

\vspace{.15in}
Special forms were discussed in chapter 5 and were given as shortcuts for FOILing specific products. These forms can be used in reverse to factor as well.

\begin{mdframed}
\textbf{Special Forms}
\begin{itemize}
\item Difference of Squares: $a^2 - b^2 = (a-b)(a+b)$
\item Square of a Binomial Sum: $(a+b)^2 = a^2 + 2ab + b^2$
\item Square of a Binomial Difference: $(a-b)^2 = a^2 - 2ab + b^2$
\end{itemize}
\end{mdframed}
\vspace{.15in}

\subsubsection*{Factoring with Difference of Squares}

\begin{example}
Factor $x^2 - 81$
\vspace{1.15in}
\end{example}

\begin{example}
Factor $36x^2 - 25$
\vspace{1.15in}
\end{example}

\begin{example}
Factor $49 - 4x^{10}$
\end{example}

\newpage


\begin{example}
Factor $100x^4 - 9y^6$
\vspace{1in}
\end{example}

\begin{example}
Factor $18x^3 - 2x$
\vspace{1in}
\end{example}

\begin{example}
Factor $72 - 18x^2$
\vspace{1in}
\end{example}

\noindent On occasion, we may end up needing to factor repeatedly to get to the final answer.
\vspace{.15in}

\begin{example}
Factor $81x^4 - 16$
\end{example}

\newpage

\subsubsection*{Factoring Perfect Square Trinomials}
\begin{example}
Factor $x^2 + 14x + 49$
\vspace{1.5in}
\end{example}

\begin{example}
Factor $x^2 -6x + 9$
\vspace{1.5in}
\end{example}

\begin{example}
Factor $16x^2 - 56x + 49$
\vspace{1.5in}
\end{example}

\begin{example}
Factor $16x^2 + 40xy + 25y^2$
\vspace{1.5in}
\end{example}

\subsubsection*{Factoring the Sum/Difference of Cubes}
\begin{itemize}
\item $a^3 + b^3 = (a+b)(a^2 - ab + b^2)$
\item $a^3 - b^3 = (a-b)(a^2 + ab + b^2)$
\end{itemize}
\vspace{.15in}

\begin{example}
Factor $x^3 + 27$
\vspace{1.5in}
\end{example}

\begin{example}
Factor $1 - y^3$
\vspace{1.5in}
\end{example}

\begin{example}
Factor $125x^3 + 8y^6$
\end{example}

\end{document}