\documentclass[addpoints,12pt]{exam}
\usepackage{amsmath}
\usepackage{amsthm}
\usepackage{amsfonts}
\usepackage{systeme}
\usepackage{graphicx}
\usepackage{caption}
\usepackage{xfrac}
\usepackage{physics}
\usepackage{microtype}
\usepackage{eulervm}
%\usepackage[framemethod=tikz]{mdframed}
\usepackage{thmtools}
\usepackage{etoolbox}
%\usepackage{fouriernc}
\usepackage{mdframed}
\usepackage[overload]{empheq}
\usepackage{adjustbox}
\usepackage{enumitem}
\usepackage[explicit]{titlesec}
% adds in \varnothing for empty set
\usepackage{amssymb}
% adds in formated SI units
%\usepackage{siunitx}
\usepackage{pgfplots}
\usepackage{multirow}
\usepackage{array}

\pagestyle{headandfoot}
\runningfootrule
\firstpageheadrule
\runningheadrule

\newcommand{\class}{Math 0097}
\newcommand{\sem}{2211}
\newcommand{\due}{}
\newcommand{\sect}{5.5}
\newcommand{\topic}{Dividing Polynomials}

\firstpageheader{\class}{\sect - \topic}{}
\runningheader{\class}{\sect - \topic}{}
\firstpagefooter{\class}{}{Page \thepage\ of \numpages}
\runningfooter{\class}{}{Page \thepage\ of \numpages}

\newif\ifprintselected
\printselectedtrue
%\printselectedfalse

\newenvironment{select}
{\ifprintselected
	\printanswers
	\fi
}
{}

\theoremstyle{definition}
\newtheorem{theorem}{Theorem}
%\newtheorem{example}{Example}[subsection]
%\newtheorem{definition}{Definition}
%\newmdtheoremenv{definition}{Definition}[subsection]
%\newmdtheoremenv{example}{Example}[subsection]
\AtBeginEnvironment{defn}{\begin{minipage}{\textwidth}}
\AtEndEnvironment{defn}{\end{minipage}}
%\AtBeginEnvironment{example}{\begin{minipage}{\textwidth}}
%\AtEndEnvironment{example}{\end{minipage}}
\newcommand{\iu}{{i\mkern1mu}}

\setlength{\gridsize}{5mm}
\setlength{\gridlinewidth}{0.1pt}

\printanswers
\DeclareMathSizes{12}{12}{12}{12}

%%%%%%%%%%%%%%%%%%%%%%%%
% Create bars around subsubsection
%%%%%%%%%%%%%%%%%%%%%%%%

\titleformat{\subsubsection}
   {\large\bfseries}% format
   {}% label
   {0pt}% sep
   {\titlerule \vspace{.1in} #1}% before code
      [{\titlerule[0.4pt]\vspace{.1in}}]% after code
\titlespacing{\subsubsection}
   {0pt}% left
   {0pt}% before sep
   {\baselineskip}% after sep
   
%%%%%%%%%%%%%%%%%%%%%%%
% Create line break after definition label
%%%%%%%%%%%%%%%%%%%%%%%   
\newtheoremstyle{break}
  {\topsep}{\topsep}%
  {}{}%\itshape
  {\bfseries}{}%
  {\newline}{}%
\theoremstyle{break}
\newmdtheoremenv{definition}{Definition}[subsection]
\theoremstyle{break}
\newtheorem{example}{Example}[subsection]

%%%%%%%%%%%%%%%%%%%%%%
% start document
% set section, subsection (use n-1 for sub)
%%%%%%%%%%%%%%%%%%%%%%


\begin{document}
\setcounter{section}{5}
\setcounter{subsection}{4}

\subsection{Dividing Polynomials}

\vspace{.15in}

\begin{definition}[Quotient Rule for Exponents]
$\dfrac{b^x}{b^y} = b^{x-y}$
\end{definition}

\vspace{.15in}

\begin{example}

Simplify each of the following:
\begin{enumerate}
\item $\dfrac{5^{12}}{5^4} = $
\vspace{.25in}
\item $\dfrac{x^9}{x^2} = $
\vspace{.25in}
\item $\dfrac{y^3}{y^5} = $
\vspace{.25in}
\end{enumerate}
\end{example}

\noindent What if, however, both exponents match?

\[\dfrac{b^x}{b^x} = \]

\noindent Why does this work?
\vspace{.75in}

\begin{definition}[Zero-Exponent Rule]
$b^0 = 1$ for any $b \neq 0$
\end{definition}

\newpage

\begin{example}
Simplify each:
\begin{enumerate}
\item $14^0 = $
\vspace{.15in}
\item $(-10)^0 = $
\vspace{.15in}
\item $-10^0 = $
\vspace{.15in}
\item $20x^0 = $
\vspace{.15in}
\item $(20x)^0 = $
\vspace{.15in}
\end{enumerate}
\end{example}

\begin{definition}[Powers of Quotients]
$\left(\dfrac{a}{b}\right)^x = \dfrac{a^x}{b^x}$
\end{definition}
\vspace{.15in}

\begin{example}
Simplify each:
\begin{enumerate}
\item $\left(\dfrac{x}{5}\right)^2 = $
\vspace{.25in}
\item $\left(\dfrac{x^4}{2}\right)^3 = $
\vspace{.25in}
\item $\left(\dfrac{2a^{10}}{b^3}\right)^4 = $
\end{enumerate}
\end{example}

\newpage

\begin{example}
Find \[(-15x^9 + 6x^5 - 9x^3) \divisionsymbol 3x^2\]
\vspace{1.5in}
\end{example}

\begin{example}
Find \[\dfrac{25x^9 - 7x^4 + 10x^3}{5x^3}\]
\vspace{1.5in}
\end{example}

\begin{example}
Find \[\dfrac{18x^7y^6 - 6x^2y^3 + 60xy^2}{6xy^2}\]
\end{example}

\end{document}