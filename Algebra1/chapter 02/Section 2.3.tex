\documentclass[addpoints,12pt]{exam}
\usepackage{amsmath}
\usepackage{amsthm}
\usepackage{amsfonts}
\usepackage{systeme}
\usepackage{graphicx}
\usepackage{caption}
\usepackage{xfrac}
\usepackage{physics}
\usepackage{microtype}
\usepackage{eulervm}
%\usepackage[framemethod=tikz]{mdframed}
\usepackage{thmtools}
\usepackage{etoolbox}
%\usepackage{fouriernc}
\usepackage{mdframed}
\usepackage[overload]{empheq}
\usepackage{adjustbox}
\usepackage{enumitem}
\usepackage[explicit]{titlesec}
% adds in \varnothing for empty set
\usepackage{amssymb}

\pagestyle{headandfoot}
\runningfootrule
\firstpageheadrule
\runningheadrule

\newcommand{\class}{Math 0097}
\newcommand{\sem}{2211}
\newcommand{\due}{}
\newcommand{\sect}{2.3}
\newcommand{\topic}{Solving Linear Equations}

\firstpageheader{\class}{\sect - \topic}{}
\runningheader{\class}{\sect - \topic}{}
\firstpagefooter{\class}{}{Page \thepage\ of \numpages}
\runningfooter{\class}{}{Page \thepage\ of \numpages}

\newif\ifprintselected
\printselectedtrue
%\printselectedfalse

\newenvironment{select}
{\ifprintselected
	\printanswers
	\fi
}
{}

\theoremstyle{definition}
\newtheorem{theorem}{Theorem}
%\newtheorem{example}{Example}[subsection]
%\newtheorem{definition}{Definition}
%\newmdtheoremenv{definition}{Definition}[subsection]
%\newmdtheoremenv{example}{Example}[subsection]
\AtBeginEnvironment{defn}{\begin{minipage}{\textwidth}}
\AtEndEnvironment{defn}{\end{minipage}}
%\AtBeginEnvironment{example}{\begin{minipage}{\textwidth}}
%\AtEndEnvironment{example}{\end{minipage}}
\newcommand{\iu}{{i\mkern1mu}}

\setlength{\gridsize}{5mm}
\setlength{\gridlinewidth}{0.1pt}

\printanswers
\DeclareMathSizes{12}{12}{12}{12}

%%%%%%%%%%%%%%%%%%%%%%%%
% Create bars around subsubsection
%%%%%%%%%%%%%%%%%%%%%%%%

\titleformat{\subsubsection}
   {\large\bfseries}% format
   {}% label
   {0pt}% sep
   {\titlerule \vspace{.1in} #1}% before code
      [{\titlerule[0.4pt]\vspace{.1in}}]% after code
\titlespacing{\subsubsection}
   {0pt}% left
   {0pt}% before sep
   {\baselineskip}% after sep
   
%%%%%%%%%%%%%%%%%%%%%%%
% Create line break after definition label
%%%%%%%%%%%%%%%%%%%%%%%   
\newtheoremstyle{break}
  {\topsep}{\topsep}%
  {}{}%\itshape
  {\bfseries}{}%
  {\newline}{}%
\theoremstyle{break}
\newmdtheoremenv{definition}{Definition}[subsection]
\theoremstyle{break}
\newtheorem{example}{Example}[subsection]

%%%%%%%%%%%%%%%%%%%%%%
% start document
% set section, subsection (use n-1 for sub)
%%%%%%%%%%%%%%%%%%%%%%


\begin{document}
\setcounter{section}{2}
\setcounter{subsection}{2}

\subsection{Solving Linear Equations}

\vspace{.25in}

\begin{mdframed}
\textbf{Method - Solving Linear Equations}
\begin{enumerate}
\item simplify the algebraic expressions on both sides
\item collect all variable terms on one side, constants to the other
\item isolate the variable
\item check your solution
\end{enumerate}
\end{mdframed}

\vspace{.15in}

\begin{example}
Solve for $x$ and check: \[-7x + 25 + 3x = 16 - 2x - 3\]
\vspace{2in}
\end{example}

\begin{example}
Solve for $x$ and check: \[8x = 2(x+6)\]
\end{example}

\newpage

\begin{example}
Solve for $x$ and check: \[4(2x + 1) - 29 = 3(2x - 5)\]
\vspace{2.5in}
\end{example}

\begin{example}
Solve for $x$ and check: \[\dfrac{x}{4} = \dfrac{2x}{3} + \dfrac{5}{6}\]
\end{example}

\newpage

\subsubsection*{Equations with No or Infinite Solutions}
\noindent Consider the equation $x = x + 4$. If we solve it by subtracting $x$ from both sides, we determine that $0 = 4$ which we know is not true - i.e., a \emph{false statement}. This tells us that there are \textbf{no} values of $x$ that satisfy the original equation. We write this as either $\varnothing$ or $\{\}$.

\vspace{.15in}

\noindent Now consider a similar statement, $x + 3 = 5 + x - 2$. Solving this yields the statement that $3=3$, which we should recognize as being true - i.e., a \emph{true statement}. This is called an \emph{identity} and tells us that \emph{any} value of $x$ is a solution to the original equation. This is typically notated as one of the following ways: "all real numbers", $\mathbb{R}$, $\{x \mid x \text{ is a real number}\}$, or $\{x \mid x \in\mathbb{R}\}$.

\vspace{.15in}

\begin{example}
Solve for $x$ and verify: \[3x + 7 = 3(x+1)\]
\vspace{2in}
\end{example}

\begin{example}
Solve for $x$ and verify: \[3(x-1) + 9 = 8x + 6 - 5x\]
\end{example}
\end{document}