\documentclass[addpoints,12pt]{exam}
\usepackage{amsmath}
\usepackage{amsthm}
\usepackage{amsfonts}
\usepackage{systeme}
\usepackage{graphicx}
\usepackage{caption}
\usepackage{xfrac}
\usepackage{physics}
\usepackage{microtype}
\usepackage{eulervm}
%\usepackage[framemethod=tikz]{mdframed}
\usepackage{thmtools}
\usepackage{etoolbox}
%\usepackage{fouriernc}
\usepackage{mdframed}
\usepackage[overload]{empheq}
\usepackage{adjustbox}
\usepackage{enumitem}
\usepackage[explicit]{titlesec}
% adds in \varnothing for empty set
\usepackage{amssymb}
% adds in formated SI units
%\usepackage{siunitx}
\usepackage{pgfplots}
\usepackage{multirow}
\usepackage{array}

\pagestyle{headandfoot}
\runningfootrule
\firstpageheadrule
\runningheadrule

\newcommand{\class}{Math 0097}
\newcommand{\sem}{2211}
\newcommand{\due}{}
\newcommand{\sect}{5.3}
\newcommand{\topic}{Special Products}

\firstpageheader{\class}{\sect - \topic}{}
\runningheader{\class}{\sect - \topic}{}
\firstpagefooter{\class}{}{Page \thepage\ of \numpages}
\runningfooter{\class}{}{Page \thepage\ of \numpages}

\newif\ifprintselected
\printselectedtrue
%\printselectedfalse

\newenvironment{select}
{\ifprintselected
	\printanswers
	\fi
}
{}

\theoremstyle{definition}
\newtheorem{theorem}{Theorem}
%\newtheorem{example}{Example}[subsection]
%\newtheorem{definition}{Definition}
%\newmdtheoremenv{definition}{Definition}[subsection]
%\newmdtheoremenv{example}{Example}[subsection]
\AtBeginEnvironment{defn}{\begin{minipage}{\textwidth}}
\AtEndEnvironment{defn}{\end{minipage}}
%\AtBeginEnvironment{example}{\begin{minipage}{\textwidth}}
%\AtEndEnvironment{example}{\end{minipage}}
\newcommand{\iu}{{i\mkern1mu}}

\setlength{\gridsize}{5mm}
\setlength{\gridlinewidth}{0.1pt}

\printanswers
\DeclareMathSizes{12}{12}{12}{12}

%%%%%%%%%%%%%%%%%%%%%%%%
% Create bars around subsubsection
%%%%%%%%%%%%%%%%%%%%%%%%

\titleformat{\subsubsection}
   {\large\bfseries}% format
   {}% label
   {0pt}% sep
   {\titlerule \vspace{.1in} #1}% before code
      [{\titlerule[0.4pt]\vspace{.1in}}]% after code
\titlespacing{\subsubsection}
   {0pt}% left
   {0pt}% before sep
   {\baselineskip}% after sep
   
%%%%%%%%%%%%%%%%%%%%%%%
% Create line break after definition label
%%%%%%%%%%%%%%%%%%%%%%%   
\newtheoremstyle{break}
  {\topsep}{\topsep}%
  {}{}%\itshape
  {\bfseries}{}%
  {\newline}{}%
\theoremstyle{break}
\newmdtheoremenv{definition}{Definition}[subsection]
\theoremstyle{break}
\newtheorem{example}{Example}[subsection]

%%%%%%%%%%%%%%%%%%%%%%
% start document
% set section, subsection (use n-1 for sub)
%%%%%%%%%%%%%%%%%%%%%%


\begin{document}
\setcounter{section}{5}
\setcounter{subsection}{2}

\subsection{Special Products}

\vspace{.15in}

In the previous section we discussed how to multiply polynomials and it ended with examples of the FOIL method for multiplying. This section provides shortcut methods for multiplying polynomials that are given in specific forms. Anything given in this section can be found using the FOIL method, but knowing these shortcuts can make some of what we do in chapter 6 - factoring.

\vspace{.15in}

\begin{definition}[Difference of Squares]
$a^2 - b^2 = (a-b)(a+b)$
\end{definition}

\begin{definition}[Square of a Binomial Sum]
$(a+b)^2 = a^2 + 2ab + b^2$
\end{definition}

\begin{definition}[Difference of a Binomial Sum]
$(a-b)^2 = a^2 - 2ab + b^2$
\end{definition}

\vspace{.15in}

\begin{example}
Find the following using a special product formula.
\[ (7y+8)(7y-8)\]
\vspace{.5in}
\end{example}

\begin{example}
Find the following using a special product formula.
\[ (2a^3 + 3)(2a^3 - 3)\]
\vspace{.5in}
\end{example}

\newpage

\begin{example}
Find the following using a special product formula.
\[ (x+10)^2\]
\vspace{.6in}
\end{example}

\begin{example}
Find the following using a special product formula.
\[ (5x+4)^2\]
\vspace{.6in}
\end{example}

\begin{example}
Find the following using a special product formula.
\[ (x-9)^2\]
\vspace{.6in}
\end{example}

\begin{example}
Find the following using a special product formula.
\[ (7x-3)^2\]
\end{example}

\end{document}