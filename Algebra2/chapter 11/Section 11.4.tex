\documentclass[addpoints,12pt]{exam}
\usepackage{amsmath}
\usepackage{amsthm}
\usepackage{amsfonts}
\usepackage{systeme}
\usepackage{graphicx}
\usepackage{caption}
\usepackage{xfrac}
\usepackage{physics}
\usepackage{microtype}
\usepackage{eulervm}
%\usepackage[framemethod=tikz]{mdframed}
\usepackage{thmtools}
\usepackage{etoolbox}
%\usepackage{fouriernc}
\usepackage{mdframed}
\usepackage[overload]{empheq}
\usepackage{adjustbox}
\usepackage{enumitem}
\usepackage[explicit]{titlesec}
% adds in \varnothing for empty set
\usepackage{amssymb}
% adds in formated SI units
%\usepackage{siunitx}
\usepackage{pgfplots}
\usepackage{multirow}
\usepackage{array}

\pagestyle{headandfoot}
\runningfootrule
\firstpageheadrule
\runningheadrule

\newcommand{\class}{Math 0098}
\newcommand{\sem}{2211}
\newcommand{\due}{}
\newcommand{\sect}{11.4}
\newcommand{\topic}{Equations in Quadratic Form}

\firstpageheader{\class}{\sect - \topic}{}
\runningheader{\class}{\sect - \topic}{}
\firstpagefooter{\class}{}{Page \thepage\ of \numpages}
\runningfooter{\class}{}{Page \thepage\ of \numpages}

\newif\ifprintselected
\printselectedtrue
%\printselectedfalse

\newenvironment{select}
{\ifprintselected
	\printanswers
	\fi
}
{}

\theoremstyle{definition}
\newtheorem{theorem}{Theorem}
%\newtheorem{example}{Example}[subsection]
%\newtheorem{definition}{Definition}
%\newmdtheoremenv{definition}{Definition}[subsection]
%\newmdtheoremenv{example}{Example}[subsection]
\AtBeginEnvironment{defn}{\begin{minipage}{\textwidth}}
\AtEndEnvironment{defn}{\end{minipage}}
%\AtBeginEnvironment{example}{\begin{minipage}{\textwidth}}
%\AtEndEnvironment{example}{\end{minipage}}
\newcommand{\iu}{{i\mkern1mu}}

\setlength{\gridsize}{5mm}
\setlength{\gridlinewidth}{0.1pt}

\printanswers
\DeclareMathSizes{12}{12}{12}{12}

%%%%%%%%%%%%%%%%%%%%%%%%
% Create bars around subsubsection
%%%%%%%%%%%%%%%%%%%%%%%%

\titleformat{\subsubsection}
   {\large\bfseries}% format
   {}% label
   {0pt}% sep
   {\titlerule \vspace{.1in} #1}% before code
      [{\titlerule[0.4pt]\vspace{.1in}}]% after code
\titlespacing{\subsubsection}
   {0pt}% left
   {0pt}% before sep
   {\baselineskip}% after sep
   
%%%%%%%%%%%%%%%%%%%%%%%
% Create line break after definition label
%%%%%%%%%%%%%%%%%%%%%%%   
\newtheoremstyle{break}
  {\topsep}{\topsep}%
  {}{}%\itshape
  {\bfseries}{}%
  {\newline}{}%
\theoremstyle{break}
\newmdtheoremenv{definition}{Definition}[subsection]
\theoremstyle{break}
\newtheorem{example}{Example}[subsection]

%%%%%%%%%%%%%%%%%%%%%%
% start document
% set section, subsection (use n-1 for sub)
%%%%%%%%%%%%%%%%%%%%%%


\begin{document}
\setcounter{section}{11}
\setcounter{subsection}{3}

\subsection{Equations in Quadratic Form}
We can sometimes rewrite non-quadratic equations as quadratics by making a substitution or "change of variable" as it is more formally referred to. Essentially, we choose a value such as $u = x^2$ and change the variables from $x$ to $u$ where we can. If done correctly, your new equation should look like a normal quadratic equation, but only have $u$ as the variable, not $x$.
\vspace{.15in}

\begin{example}
Rewrite $x^4 - 10x^2 + 9 = 0$ as a quadratic.
\vspace{1in}
\end{example}

\begin{example}
Rewrite $5x^{\sfrac{2}{3}} + 11x^{\sfrac{1}{3}} + 2 = 0$ as a quadratic.
\vspace{1in}
\end{example}

\begin{example}
Solve $x^4 - 5x^2 + 6 = 0$.
\end{example}

\newpage

\begin{example}
Solve $x - 2\sqrt{x} - 8 = 0$. Be sure to check your answers.
\end{example}

\newpage

\begin{example}
Solve $(x^2 - 4)^2 + (x^2 - 4) - 6 = 0$.
\vspace{3in}
\end{example}

\begin{example}
Solve $2x^{-2} + x^{-1} - 1 = 0$.
\end{example}

\newpage

\begin{example}
Solve $3x^{\sfrac{2}{3}} - 11x^{\sfrac{1}{3}} - 4 = 0$.
\end{example}

\end{document}