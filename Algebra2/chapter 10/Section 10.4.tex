\documentclass[addpoints,12pt]{exam}
\usepackage{amsmath}
\usepackage{amsthm}
\usepackage{amsfonts}
\usepackage{systeme}
\usepackage{graphicx}
\usepackage{caption}
\usepackage{xfrac}
\usepackage{physics}
\usepackage{microtype}
\usepackage{eulervm}
%\usepackage[framemethod=tikz]{mdframed}
\usepackage{thmtools}
\usepackage{etoolbox}
%\usepackage{fouriernc}
\usepackage{mdframed}
\usepackage[overload]{empheq}
\usepackage{adjustbox}
\usepackage{enumitem}
\usepackage[explicit]{titlesec}

\pagestyle{headandfoot}
\runningfootrule
\firstpageheadrule
\runningheadrule

\newcommand{\class}{Math 0098}
\newcommand{\sem}{2201}
\newcommand{\due}{}
\newcommand{\sect}{10.4}
\newcommand{\topic}{Adding, Subtracting, and Dividing Radical Expressions}

\firstpageheader{\class}{\sect - \topic}{}
\runningheader{\class}{\sect - \topic}{}
\firstpagefooter{\class}{}{Page \thepage\ of \numpages}
\runningfooter{\class}{}{Page \thepage\ of \numpages}

\newif\ifprintselected
\printselectedtrue
%\printselectedfalse

\newenvironment{select}
{\ifprintselected
	\printanswers
	\fi
}
{}

\theoremstyle{definition}
\newtheorem{theorem}{Theorem}
\newtheorem{example}{Example}[subsection]
%\newtheorem{definition}{Definition}
\newtheorem{definition}{Definition}[subsection]
%\newmdtheoremenv{example}{Example}[subsection]
\AtBeginEnvironment{defn}{\begin{minipage}{\textwidth}}
\AtEndEnvironment{defn}{\end{minipage}}
%\AtBeginEnvironment{example}{\begin{minipage}{\textwidth}}
%\AtEndEnvironment{example}{\end{minipage}}
\newcommand{\iu}{{i\mkern1mu}}

\setlength{\gridsize}{5mm}
\setlength{\gridlinewidth}{0.1pt}

\printanswers
\DeclareMathSizes{12}{12}{12}{12}

\titleformat{\subsubsection}
   {\large\bfseries}% format
   {}% label
   {0pt}% sep
   {\titlerule \vspace{.1in} #1}% before code
      [{\titlerule[0.4pt]\vspace{.1in}}]% after code
\titlespacing{\subsubsection}
   {0pt}% left
   {0pt}% before sep
   {\baselineskip}% after sep


\begin{document}
\setcounter{section}{10}
\setcounter{subsection}{3}

\subsection{Adding, Subtracting, and Dividing Radical Expressions}

\subsubsection*{Adding and Subtracting Rational Expressions}

We know that we can combine like terms to simplify expressions. For example,
\[2x + 3x = (2+3)x = 5x\]

We can add and subtract radicals in the same manner, however, we have to have the same radicand for it to work.

\[2\sqrt{3} + 3\sqrt{3} = (2+3)\sqrt{3} = 5\sqrt{3}\]

If instead, we had unlike radicands, we would treat them as unlike terms and would be unable to combine them:

\[-4\sqrt{7}+8\sqrt{7} + 2\sqrt{3} = (-4+8)\sqrt{7} + 2\sqrt{3} = 4\sqrt{7} + 2\sqrt{3}\]

\vspace{.25in}

\begin{example}
Simplify by combining like terms:
\begin{enumerate}
\item $8\sqrt{13} + 2\sqrt{13}$
\vspace{1in}
\item $9\sqrt[3]{7}-6x\sqrt[3]{7}+12\sqrt[3]{7}$
\vspace{1in}
\item $7\sqrt[4]{3x} - 2\sqrt[4]{3x} + 2\sqrt[3]{3x}$
\vspace{1.5in}
\end{enumerate}
\end{example}

\newpage

\begin{example}
Simplify by combining like terms, but reduce the radicals if necessary.
\begin{enumerate}
\item $3\sqrt{20} + 5\sqrt{45}$
\vspace{1.25in}
\item $3\sqrt{12x} - 6\sqrt{27x}$
\vspace{1.25in}
\item $8\sqrt{5} - 6\sqrt{2}$
\vspace{1.25in}
\end{enumerate}
\end{example}

\begin{example}
Simplify by combining like terms, but reduce the radicals if necessary.

\begin{enumerate}
\item $3\sqrt[3]{24} - 5\sqrt[3]{81}$
\vspace{1.25in}
\item $5\sqrt[3]{x^2y} + \sqrt[3]{27x^5y^4}$
\vspace{1.25in}
\end{enumerate}
\end{example}

\begin{mdframed}
\textbf{Quotient Rule for Radicals}

Let $\sqrt[n]{a}$ and $\sqrt[n]{b}$ be two real numbers with $b\neq 0$, then \[\sqrt[n]{\dfrac{a}{b}} = \dfrac{\sqrt[n]{a}}{\sqrt[n]{b}}\]
\end{mdframed}

\vspace{.15in}

\begin{example}
Simplify using the quotient rule:
\begin{enumerate}
\begin{minipage}{.5\textwidth}
\item $\sqrt[3]{\dfrac{24}{125}}$
\vspace{1.05in}
\item $\sqrt{\dfrac{9x^3}{y^{10}}}$
\vspace{1.05in}
\end{minipage}%
\begin{minipage}{.5\textwidth}
\item $\sqrt[3]{\dfrac{8y^7}{x^{12}}}$
\vspace{1.05in}
\item $\sqrt{\dfrac{x^2}{25y^6}}$
\vspace{1.05in}
\end{minipage}%
\end{enumerate}
\end{example}

\begin{example}
Divide and simplify as appropriate.
\begin{enumerate}
\begin{minipage}{.5\textwidth}
\item $\dfrac{\sqrt{40x^5}}{\sqrt{2x}}$
\vspace{1.15in}
\item $\dfrac{\sqrt{50xy}}{2\sqrt{2}}$
\end{minipage}%
\begin{minipage}{.5\textwidth}
\item $\dfrac{\sqrt[3]{48xy}}{\sqrt[3]{6xy^{-2}}}$
\vspace{1.15in}
\item $\dfrac{\sqrt[3]{16x^5y^2}}{\sqrt[3]{2xy^{-1}}}$
\end{minipage}%
\end{enumerate}
\end{example}

\end{document}