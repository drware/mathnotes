\documentclass[addpoints,12pt]{exam}
\usepackage{amsmath}
\usepackage{amsthm}
\usepackage{amsfonts}
\usepackage{systeme}
\usepackage{graphicx}
\usepackage{caption}
\usepackage{xfrac}
\usepackage{physics}
\usepackage{microtype}
\usepackage{eulervm}
%\usepackage[framemethod=tikz]{mdframed}
\usepackage{thmtools}
\usepackage{etoolbox}
%\usepackage{fouriernc}
\usepackage{mdframed}
\usepackage[overload]{empheq}
\usepackage{adjustbox}
\usepackage{enumitem}
\usepackage[explicit]{titlesec}
% adds in \varnothing for empty set
\usepackage{amssymb}
% adds in formated SI units
%\usepackage{siunitx}
\usepackage{pgfplots}
\usepackage{multirow}
\usepackage{array}

\pagestyle{headandfoot}
\runningfootrule
\firstpageheadrule
\runningheadrule

\newcommand{\class}{Math 0097}
\newcommand{\sem}{2211}
\newcommand{\due}{}
\newcommand{\sect}{4.2}
\newcommand{\topic}{The Substitution Method}

\firstpageheader{\class}{\sect - \topic}{}
\runningheader{\class}{\sect - \topic}{}
\firstpagefooter{\class}{}{Page \thepage\ of \numpages}
\runningfooter{\class}{}{Page \thepage\ of \numpages}

\newif\ifprintselected
\printselectedtrue
%\printselectedfalse

\newenvironment{select}
{\ifprintselected
	\printanswers
	\fi
}
{}

\theoremstyle{definition}
\newtheorem{theorem}{Theorem}
%\newtheorem{example}{Example}[subsection]
%\newtheorem{definition}{Definition}
%\newmdtheoremenv{definition}{Definition}[subsection]
%\newmdtheoremenv{example}{Example}[subsection]
\AtBeginEnvironment{defn}{\begin{minipage}{\textwidth}}
\AtEndEnvironment{defn}{\end{minipage}}
%\AtBeginEnvironment{example}{\begin{minipage}{\textwidth}}
%\AtEndEnvironment{example}{\end{minipage}}
\newcommand{\iu}{{i\mkern1mu}}

\setlength{\gridsize}{5mm}
\setlength{\gridlinewidth}{0.1pt}

\printanswers
\DeclareMathSizes{12}{12}{12}{12}

%%%%%%%%%%%%%%%%%%%%%%%%
% Create bars around subsubsection
%%%%%%%%%%%%%%%%%%%%%%%%

\titleformat{\subsubsection}
   {\large\bfseries}% format
   {}% label
   {0pt}% sep
   {\titlerule \vspace{.1in} #1}% before code
      [{\titlerule[0.4pt]\vspace{.1in}}]% after code
\titlespacing{\subsubsection}
   {0pt}% left
   {0pt}% before sep
   {\baselineskip}% after sep
   
%%%%%%%%%%%%%%%%%%%%%%%
% Create line break after definition label
%%%%%%%%%%%%%%%%%%%%%%%   
\newtheoremstyle{break}
  {\topsep}{\topsep}%
  {}{}%\itshape
  {\bfseries}{}%
  {\newline}{}%
\theoremstyle{break}
\newmdtheoremenv{definition}{Definition}[subsection]
\theoremstyle{break}
\newtheorem{example}{Example}[subsection]

%%%%%%%%%%%%%%%%%%%%%%
% start document
% set section, subsection (use n-1 for sub)
%%%%%%%%%%%%%%%%%%%%%%


\begin{document}
\setcounter{section}{4}
\setcounter{subsection}{1}

\subsection{The Substitution Method}

\vspace{.15in}

\begin{mdframed}
\textbf{Process - Substitution Method}
\begin{enumerate}
\item Solve one of the equations for either $x$ or $y$.
\item Substitute this new equation into the other equation.
\item Solve this new one-variable equation.
\item Plug this value into either of the original equations to find the missing value.
\item Write the solution as a point.
\end{enumerate}
\end{mdframed}

\vspace{.15in}

\begin{example}
Solve the following with the substitution method:
\[\systeme{-5x + y = -13, 2x + 3y = 12}\]
\vspace{2in}
\end{example}

\newpage

\begin{example}
Solve the following with the substitution method:
\[\systeme{3x + 2y = -1, x - y = 3}\]
\vspace{3in}
\end{example}

\begin{example}
Solve the following with the substitution method:
\[\systeme{3x + y = -5, y = -3x + 3}\]
\end{example}

\newpage

\begin{example}
Solve the following with the substitution method:
\[\systeme{-3x + y = -4, 9x - 3y = 12}\]
\end{example}
\newpage 

\begin{example}
\textbf{The Supply \& Demand of Pizza}

\begin{figure}[h]
\centering
\begin{tabular}{c | c | c | >{\centering\arraybackslash}m{5cm}}
\textbf{Price/Slice} & \textbf{Demand (Qty)} & \textbf{Supply (Qty)} & \textbf{Result}\\\hline
& & & \multirow{5}{4.5cm}{shortage - more is needed than supplied}\\
\$0.50 & 300 & 100 & \\
& & & \\
\$1.00 & 250 & 150 &  \\
& & & \\\cline{4-4}
& & & \multirow{5}{4.5cm}{surplus - more is supplied than needed; there is waste}\\
\$2.00 & 150 & 250 &  \\
& & & \\
\$3.00 & 50 & 350 & \\
& & & 
\end{tabular}
\end{figure}

\vspace{.15in}

\noindent We want to find the \emph{equilibrium price and quantity} -- that is, the price and quantity that leaves no shortage and no surplus. We can also think of it as the \emph{point} at which supply and demand meet.
\vspace{.15in}

\noindent Two equations have been constructed to model this situation. The equation $p = -0.01x + 3.5$ gives us the price ($p$) of a single slice of pizza when the demand is $x$ slices. The equation $p = 0.01x - 0.5$ gives us the price of a single slice when their are $x$ slices supplied.

\vspace{.15in}

\noindent How much do we charge per slice and how many slices do we make in order to have no shortage and no surplus?
\end{example}

\end{document}