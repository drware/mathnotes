\documentclass[addpoints,12pt]{exam}
\usepackage{amsmath}
\usepackage{amsthm}
\usepackage{amsfonts}
\usepackage{systeme}
\usepackage{graphicx}
\usepackage{caption}
\usepackage{xfrac}
\usepackage{physics}
\usepackage{microtype}
\usepackage{eulervm}
%\usepackage[framemethod=tikz]{mdframed}
\usepackage{thmtools}
\usepackage{etoolbox}
%\usepackage{fouriernc}
\usepackage{mdframed}
\usepackage[overload]{empheq}
\usepackage{adjustbox}
\usepackage{enumitem}
\usepackage[explicit]{titlesec}
% adds in \varnothing for empty set
\usepackage{amssymb}
% adds in formated SI units
%\usepackage{siunitx}
\usepackage{pgfplots}
\usepackage{multirow}
\usepackage{array}

\pagestyle{headandfoot}
\runningfootrule
\firstpageheadrule
\runningheadrule

\newcommand{\class}{Math 0097}
\newcommand{\sem}{2211}
\newcommand{\due}{}
\newcommand{\sect}{5.7}
\newcommand{\topic}{Negative Exponents \& Scientific Notation}

\firstpageheader{\class}{\sect - \topic}{}
\runningheader{\class}{\sect - \topic}{}
\firstpagefooter{\class}{}{Page \thepage\ of \numpages}
\runningfooter{\class}{}{Page \thepage\ of \numpages}

\newif\ifprintselected
\printselectedtrue
%\printselectedfalse

\newenvironment{select}
{\ifprintselected
	\printanswers
	\fi
}
{}

\theoremstyle{definition}
\newtheorem{theorem}{Theorem}
%\newtheorem{example}{Example}[subsection]
%\newtheorem{definition}{Definition}
%\newmdtheoremenv{definition}{Definition}[subsection]
%\newmdtheoremenv{example}{Example}[subsection]
\AtBeginEnvironment{defn}{\begin{minipage}{\textwidth}}
\AtEndEnvironment{defn}{\end{minipage}}
%\AtBeginEnvironment{example}{\begin{minipage}{\textwidth}}
%\AtEndEnvironment{example}{\end{minipage}}
\newcommand{\iu}{{i\mkern1mu}}

\setlength{\gridsize}{5mm}
\setlength{\gridlinewidth}{0.1pt}

\printanswers
\DeclareMathSizes{12}{12}{12}{12}

%%%%%%%%%%%%%%%%%%%%%%%%
% Create bars around subsubsection
%%%%%%%%%%%%%%%%%%%%%%%%

\titleformat{\subsubsection}
   {\large\bfseries}% format
   {}% label
   {0pt}% sep
   {\titlerule \vspace{.1in} #1}% before code
      [{\titlerule[0.4pt]\vspace{.1in}}]% after code
\titlespacing{\subsubsection}
   {0pt}% left
   {0pt}% before sep
   {\baselineskip}% after sep
   
%%%%%%%%%%%%%%%%%%%%%%%
% Create line break after definition label
%%%%%%%%%%%%%%%%%%%%%%%   
\newtheoremstyle{break}
  {\topsep}{\topsep}%
  {}{}%\itshape
  {\bfseries}{}%
  {\newline}{}%
\theoremstyle{break}
\newmdtheoremenv{definition}{Definition}[subsection]
\theoremstyle{break}
\newtheorem{example}{Example}[subsection]

%%%%%%%%%%%%%%%%%%%%%%
% start document
% set section, subsection (use n-1 for sub)
%%%%%%%%%%%%%%%%%%%%%%


\begin{document}
\setcounter{section}{5}
\setcounter{subsection}{6}

\subsection{Negative Exponents \& Scientific Notation}

\vspace{.15in}
Suppose that we have $\dfrac{b^3}{b^5}$. By our definitions so far, we can determine the following:
\[\dfrac{b^3}{b^5} = \dfrac{b\cdot b\cdot b}{b\cdot b\cdot b\cdot b\cdot b} = \dfrac{1}{b\cdot b} = \dfrac{1}{b^2}\]

However, by the quotient rule, we have $\dfrac{b^3}{b^5} = b^{3-5} = b^{-2}$. Since we assume that both methods are correct, we can safely say that $\dfrac{1}{b^2} = b^{-2}$ by the \emph{transitive property}.
\vspace{.15in}

\begin{definition}[Negative Exponent Rule]\mbox{}\\
\begin{itemize}
\item $b^{-n} = \dfrac{1}{b^n}$ for $b\neq 0$
\item $\dfrac{1}{b^{-n}} = b^n$ for $b \neq 0$
\end{itemize}
\end{definition}
\vspace{.15in}

\begin{example}
Rewrite each of the following with positive exponents.
\begin{enumerate}
\begin{minipage}{.5\textwidth}
\item $6^{-2}$
\vspace{.4in}
\item $(-3)^{-4}$
\vspace{.4in}
\item $-3^{-4}$
\vspace{.4in}
\item $7^{-1}$
\end{minipage}%
\begin{minipage}{.5\textwidth}
\item $\dfrac{2^{-3}}{7^{-2}}$
\vspace{.15in}
\item $\left(\dfrac{4}{5}\right)^{-2}$
\vspace{.15in}
\item $\dfrac{1}{7y^{-2}}$
\vspace{.15in}
\item $\dfrac{x^{-1}}{y^{-8}}$
\end{minipage}%
\end{enumerate}
\end{example}

\newpage

\subsubsection*{Simplifying Exponential Expressions}
\begin{example}
Simplify each of the following:
\begin{enumerate}
\item $x^{-12}\cdot x^2 = $
\vspace{.25in}
\item $\dfrac{x^2}{x^{10}} = $
\vspace{.25in}
\item $\dfrac{75x^3}{5x^9} = $
\vspace{.25in}
\item $\dfrac{50y^8}{-25y^{-14}} = $
\vspace{.25in}
\item $\dfrac{(5x^3)^2}{x^{10}} = $
\vspace{.35in}
\item $\left(\dfrac{x^8}{x^4}\right)^{-5} = $
\end{enumerate}
\end{example}

\newpage

\subsubsection*{Scientific Notation}
Scientific notation is used as a shorthand method of writing \emph{very} large numbers. For those of you who have taken Chemistry, you may remember Avogadro's number: $6.02214076\times 10^{23}$. If, for some unknown reason, we wanted to write this without using scientific notation, we would have:
\[602,214,076,000,000,000,000,000\]

\vspace{.15in}

\begin{definition}[Scientific Notation]
A number written in scientific notation has the form $a \times 10^n$ where $1 \le \abs{a} < 10$ and $n$ is some integer.
\end{definition}
\vspace{.15in}

\begin{mdframed}
\textbf{Procedure: Convert from Scientific Notation}
\begin{itemize}
\item If $n > 0$, move the decimal to the \emph{right} by $n$ places, adding in 0s as necessary. This should give you a \emph{large} number.
\item If $n < 0$, move the decimal the the \emph{left} by $n$ places, adding in 0s as necessary. This should give you a \emph{small} number.
\end{itemize}
\end{mdframed}
\vspace{.15in}

\begin{example}
Convert $7.4 \times 10^9$ to standard decimal notation.
\vspace{.75in}
\end{example}

\begin{example}
Convert $3.017\times 10^{-6}$ to standard decimal notation.
\vspace{.5in}
\end{example}

\newpage
\begin{mdframed}
\textbf{Procedure: Convert to Scientific Notation}
\begin{itemize}
\item Determine $a$ -- move the decimal around until $1 \le \abs{a} < 10$.
\item Determine $n$ -- $n$ is the number of places that the decimal point was moved. $n > 0$ if the original number is larger than 10 (or smaller than -10) and $n < 0$ if the original is between $-1$ and $1$.
\end{itemize}
\end{mdframed}
\vspace{.15in}

\begin{example}
Convert $7,410,000,000$ to scientific notation.
\vspace{.75in}
\end{example}

\begin{example}
Convert $-4,120,000$ to scientific notation.
\vspace{.75in}
\end{example}

\begin{example}
Convert $0.000023$ to scientific notation.
\vspace{1in}
\end{example}

\subsubsection*{Operations on Scientific Notations}
\begin{itemize}
\item Multiplication: $(a\times 10^m)(b \times 10^n) = ab \times 10^{m+n}$
\vspace{.15in}
\item Division: $\dfrac{a \times 10^m}{b \times 10^n} = \left(\dfrac{a}{b}\right) \times 10^{m-n}$
\vspace{.15in}
\item Exponentiation: $(a\times 10^m)^n = a^n \times 10^{mn}$
\end{itemize}

\newpage

\begin{example}
Simplify and write in scientific notation: 
\[(3\times 10^8)(2\times 10^2)\]
\vspace{.75in}
\end{example}

\begin{example}
Simplify and write in scientific notation: 
\[\dfrac{8.4\times 10^7}{4\times 10^{-4}}\]
\vspace{.75in}
\end{example}

\begin{example}
Simplify and write in scientific notation: 
\[(4\times 10^{-2})^3\]
\vspace{.75in}
\end{example}

\begin{example}
Simplify and write in scientific notation: 
\[\dfrac{(4\times 10^5)(9\times 10^{-4})}{2\times 10^{-3}}\]
\end{example}
\end{document}