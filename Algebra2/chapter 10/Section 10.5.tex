\documentclass[addpoints,12pt]{exam}
\usepackage{amsmath}
\usepackage{amsthm}
\usepackage{amsfonts}
\usepackage{systeme}
\usepackage{graphicx}
\usepackage{caption}
\usepackage{xfrac}
\usepackage{physics}
\usepackage{microtype}
\usepackage{eulervm}
%\usepackage[framemethod=tikz]{mdframed}
\usepackage{thmtools}
\usepackage{etoolbox}
%\usepackage{fouriernc}
\usepackage{mdframed}
\usepackage[overload]{empheq}
\usepackage{adjustbox}
\usepackage{enumitem}
\usepackage[explicit]{titlesec}

\pagestyle{headandfoot}
\runningfootrule
\firstpageheadrule
\runningheadrule

\newcommand{\class}{Math 0098}
\newcommand{\sem}{2201}
\newcommand{\due}{}
\newcommand{\sect}{10.5}
\newcommand{\topic}{Multiplying with Multiple Terms \& Rationalizing}

\firstpageheader{\class}{\sect - \topic}{}
\runningheader{\class}{\sect - \topic}{}
\firstpagefooter{\class}{}{Page \thepage\ of \numpages}
\runningfooter{\class}{}{Page \thepage\ of \numpages}

\newif\ifprintselected
\printselectedtrue
%\printselectedfalse

\newenvironment{select}
{\ifprintselected
	\printanswers
	\fi
}
{}

\theoremstyle{definition}
\newtheorem{theorem}{Theorem}
\newtheorem{example}{Example}[subsection]
%\newtheorem{definition}{Definition}
\newtheorem{definition}{Definition}[subsection]
%\newmdtheoremenv{example}{Example}[subsection]
\AtBeginEnvironment{defn}{\begin{minipage}{\textwidth}}
\AtEndEnvironment{defn}{\end{minipage}}
%\AtBeginEnvironment{example}{\begin{minipage}{\textwidth}}
%\AtEndEnvironment{example}{\end{minipage}}
\newcommand{\iu}{{i\mkern1mu}}

\setlength{\gridsize}{5mm}
\setlength{\gridlinewidth}{0.1pt}

\printanswers
\DeclareMathSizes{12}{12}{12}{12}

\titleformat{\subsubsection}
   {\large\bfseries}% format
   {}% label
   {0pt}% sep
   {\titlerule \vspace{.1in} #1}% before code
      [{\titlerule[0.4pt]\vspace{.1in}}]% after code
\titlespacing{\subsubsection}
   {0pt}% left
   {0pt}% before sep
   {\baselineskip}% after sep


\begin{document}
\setcounter{section}{10}
\setcounter{subsection}{4}

\subsection{Multiplying with More than One Term \& Rationalizing Denominators}

\subsubsection*{Multiplying Radical Expressions with Multiple Terms}

The same methods of FOILing, distributing, etc., that we've been using for polynomials still apply for radicals.
\vspace{.25in}
\begin{example}
Multiply each of the following. Be sure to simplify if possible.
\begin{enumerate}
\begin{minipage}{.5\textwidth}
\item $\sqrt{6}\left(x+\sqrt{10}\right)$
\vspace{1.5in}
\item $\sqrt[3]{y}\left(\sqrt[3]{y^2} - \sqrt[3]{7}\right)$
\vspace{1.5in}
\end{minipage}%
\begin{minipage}{.5\textwidth}
\item $\left(6\sqrt{5}+3\sqrt{2}\right)\left(2\sqrt{5}-4\sqrt{2}\right)$
\vspace{1.5in}
\item $\left(5\sqrt{2}+2\sqrt{3}\right)\left(4\sqrt{2}-3\sqrt{3}\right)$
\vspace{1.5in}
\end{minipage}%
\end{enumerate}
\end{example}
In Algebra 1, we were introduced to \emph{special forms} for multiplication -- binomial sums, binomial differences and differences of squares. These rules still hold for radicals.

\begin{mdframed}
\textbf{Special Forms}
\begin{enumerate}
\item $(a+b)^2 = a^2 + 2ab + b^2$
\item $(a-b)^2 = a^2 - 2ab + b^2$
\item $(a-b)(a+b) = a^2 - b^2$
\end{enumerate}
\end{mdframed}

\newpage

\begin{example}
Multiply each of the following special forms. Be sure to simplify if possible.
\begin{enumerate}
\begin{minipage}{.5\textwidth}
\item $\left(\sqrt{5}+\sqrt{6}\right)^2$
\vspace{1.5in}
\item $\left(\sqrt{6}-\sqrt{5}\right)^2$
\vspace{1.5in}
\end{minipage}%
\begin{minipage}{.5\textwidth}
\item $\left(\sqrt{a}-\sqrt{7}\right)\left(\sqrt{a}+\sqrt{7}\right)$
\vspace{1.5in}
\item $\left(\sqrt{2x}-\sqrt{3b}\right)\left(\sqrt{2x}+\sqrt{3b}\right)$
\vspace{1.5in}
\end{minipage}%
\end{enumerate}
\end{example}




\subsubsection*{Rationalizing Denominators with One Term}

\emph{Rationalizing a denominator} is the process of rewriting an expression so that there are no radicals in the denominator. To rationalize denominators, we need to use the trick of multiplying by 1 (written in a specific way). If we have squareroots, we use whatever the denominator. If we have a higher index radical, we need to choose what to multiply by carefully.

\vspace{.25in}

\begin{example}
Rationalize each denominator.
\begin{enumerate}
\begin{minipage}{.5\textwidth}
\item $\dfrac{\sqrt{7}}{\sqrt{3}}$
\end{minipage}%
\begin{minipage}{.5\textwidth}
\item $\dfrac{\sqrt{5}}{\sqrt{12}}$
\end{minipage}%
\end{enumerate}
\end{example}

\newpage

If, however, we have an index greater than 2, we need to choose the radicand in such a way that we have exponents that are multiples of the index -- allowing us to simplify the radical in the denominator.

\vspace{.25in}

\begin{example}
Rationalize each denominator.
\begin{enumerate}
\begin{minipage}{.5\textwidth}
\item $\dfrac{\sqrt[3]{2}}{\sqrt[3]{9}}$
\end{minipage}%
\begin{minipage}{.5\textwidth}
\item $\dfrac{\sqrt[4]{2}}{\sqrt[4]{3}}$
\end{minipage}%
\end{enumerate}
\end{example}

\vspace{1.5in}

If the radicand contains multiple factors (constants and variables), we need to make sure that each factor in the radicand ends with an exponent that is a multiple of the index. This is true regardless of the index.

\vspace{.25in}

\begin{example}
Rationalize each denominator.
\begin{enumerate}
\begin{minipage}{.33\textwidth}
\item $\sqrt{\dfrac{2x}{7x}}$
\end{minipage}%
\begin{minipage}{.33\textwidth}
\item $\dfrac{\sqrt[3]{x}}{\sqrt[3]{9y}}$
\end{minipage}%
\begin{minipage}{.33\textwidth}
\item $\dfrac{6x}{\sqrt[5]{8x^2y^4}}$
\end{minipage}%
\end{enumerate}
\end{example}

\newpage

\subsubsection*{Rationalizing Denominators with Multiple Terms}

If the denominator contains two or more terms where at least one of them is a radical, we need to multiply by the \emph{conjugate} of the denominator. The conjugate is an idea that will get used a few times in various forms throughout this class.
\vspace{.2in}
To find the conjugate of a radical expression, change the operation to the opposite - addition to subtraction, subtraction to addition. 

\vspace{.2in}
Why use the conjugate? The conjugate allows us to use the difference of squares method to simplify the denominator. If the denominator is $a + b$, then the conjugate is $a - b$. Multiplying them together gives us \[(a+b)(a-b) = a^2 - b^2\]

\begin{example}
Rationalize the following: \[\dfrac{18}{2\sqrt{3}+3}\]
\vspace{2in}
\end{example}

\begin{example}
Rationalize the following: \[\dfrac{3+\sqrt{7}}{\sqrt{5}-\sqrt{2}}\]
\vspace{2in}
\end{example}



\end{document}