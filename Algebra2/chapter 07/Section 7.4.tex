\documentclass[addpoints,12pt]{exam}
\usepackage{amsmath}
\usepackage{amsthm}
\usepackage{amsfonts}
\usepackage{systeme}
\usepackage{graphicx}
\usepackage{caption}
\usepackage{xfrac}
\usepackage{physics}
\usepackage{microtype}
\usepackage{eulervm}
%\usepackage[framemethod=tikz]{mdframed}
\usepackage{thmtools}
\usepackage{etoolbox}
%\usepackage{fouriernc}
\usepackage{mdframed}
\usepackage[overload]{empheq}
\usepackage{adjustbox}
\usepackage{enumitem}
\usepackage[explicit]{titlesec}

\pagestyle{headandfoot}
\runningfootrule
\firstpageheadrule
\runningheadrule

\newcommand{\class}{Math 0098}
\newcommand{\sem}{2211}
\newcommand{\due}{}
\newcommand{\sect}{7.4}
\newcommand{\topic}{Adding \& Subtracting with Unlike Denominators}

\firstpageheader{\class}{\sect - \topic}{}
\runningheader{\class}{\sect - \topic}{}
\firstpagefooter{\class}{}{Page \thepage\ of \numpages}
\runningfooter{\class}{}{Page \thepage\ of \numpages}

\newif\ifprintselected
\printselectedtrue
%\printselectedfalse

\newenvironment{select}
{\ifprintselected
	\printanswers
	\fi
}
{}

\theoremstyle{definition}
\newtheorem{theorem}{Theorem}
\newtheorem{example}{Example}[subsection]
%\newtheorem{definition}{Definition}
\newtheorem{definition}{Definition}[subsection]
%\newmdtheoremenv{example}{Example}[subsection]
\AtBeginEnvironment{defn}{\begin{minipage}{\textwidth}}
\AtEndEnvironment{defn}{\end{minipage}}
%\AtBeginEnvironment{example}{\begin{minipage}{\textwidth}}
%\AtEndEnvironment{example}{\end{minipage}}
\newcommand{\iu}{{i\mkern1mu}}

\setlength{\gridsize}{5mm}
\setlength{\gridlinewidth}{0.1pt}

\printanswers
\DeclareMathSizes{12}{12}{12}{12}

\titleformat{\subsubsection}
   {\large\bfseries}% format
   {}% label
   {0pt}% sep
   {\titlerule \vspace{.1in} #1}% before code
      [{\titlerule[0.4pt]\vspace{.1in}}]% after code
\titlespacing{\subsubsection}
   {0pt}% left
   {0pt}% before sep
   {\baselineskip}% after sep


\begin{document}
\setcounter{section}{7}
\setcounter{subsection}{3}

\subsection{Adding \& Subtracting with Unlike Denominators}

\noindent Addition and subtraction of rational expressions works exactly the same as with fractions even when there are unlike denominators.
\vspace{.25in}
\noindent We need to continue treating rational expressions as if they are standard fractions. This means that before we can add or subtract, we need to find the \emph{least common denominator}, or LCD.

\vspace{.25in}
\begin{example}
Find the LCD and calculate: \[\dfrac{3}{4} + \dfrac{1}{5}\]
\vspace{1.25in}
\end{example}

\begin{example}
Find the LCD of $\dfrac{3}{10x^2}$ and $\dfrac{7}{15x}$.
\vspace{2in}
\end{example}

\begin{example}
Find the LCD of $\dfrac{2}{x+3}$ and $\dfrac{4}{x-3}$.
\end{example}

\newpage

\begin{example}
Find the LCD of $\dfrac{9}{7x^2+28x}$ and $\dfrac{11}{x^2+8x+16}$.
\vspace{2in}
\end{example}

\begin{mdframed}
\textbf{Method for Addition/Subtraction}
\begin{enumerate}
\item Find the LCD
\item Multiply each rational expression to change the denominator to the LCD
\item Add or subtract the new numerators, keep the LCD as the denominator
\item Simplify if needed
\end{enumerate}
\end{mdframed}

\vspace{.25in}

\begin{example}
Find the following: \[\dfrac{3}{10x^2} + \dfrac{7}{15x}\]
\end{example}

\newpage

\begin{example}
Find the following: \[\dfrac{2}{x+3} + \dfrac{4}{x-3}\]
\vspace{1.75in}
\end{example}

\begin{example}
Find the following: \[\dfrac{x}{x+5} - 1\]
\vspace{1.75in}
\end{example}

\begin{example}
Find the following: \[\dfrac{5}{y^2-5y} - \dfrac{y}{5y-25}\]
\end{example}

\newpage

\begin{example}
Find the following: \[\dfrac{4x}{x^2-25} + \dfrac{3}{5-x}\]
\end{example}


\end{document}