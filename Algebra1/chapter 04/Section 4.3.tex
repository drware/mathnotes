\documentclass[addpoints,12pt]{exam}
\usepackage{amsmath}
\usepackage{amsthm}
\usepackage{amsfonts}
\usepackage{systeme}
\usepackage{graphicx}
\usepackage{caption}
\usepackage{xfrac}
\usepackage{physics}
\usepackage{microtype}
\usepackage{eulervm}
%\usepackage[framemethod=tikz]{mdframed}
\usepackage{thmtools}
\usepackage{etoolbox}
%\usepackage{fouriernc}
\usepackage{mdframed}
\usepackage[overload]{empheq}
\usepackage{adjustbox}
\usepackage{enumitem}
\usepackage[explicit]{titlesec}
% adds in \varnothing for empty set
\usepackage{amssymb}
% adds in formated SI units
%\usepackage{siunitx}
\usepackage{pgfplots}
\usepackage{multirow}
\usepackage{array}

\pagestyle{headandfoot}
\runningfootrule
\firstpageheadrule
\runningheadrule

\newcommand{\class}{Math 0097}
\newcommand{\sem}{2211}
\newcommand{\due}{}
\newcommand{\sect}{4.3}
\newcommand{\topic}{The Elimination or Addition Method}

\firstpageheader{\class}{\sect - \topic}{}
\runningheader{\class}{\sect - \topic}{}
\firstpagefooter{\class}{}{Page \thepage\ of \numpages}
\runningfooter{\class}{}{Page \thepage\ of \numpages}

\newif\ifprintselected
\printselectedtrue
%\printselectedfalse

\newenvironment{select}
{\ifprintselected
	\printanswers
	\fi
}
{}

\theoremstyle{definition}
\newtheorem{theorem}{Theorem}
%\newtheorem{example}{Example}[subsection]
%\newtheorem{definition}{Definition}
%\newmdtheoremenv{definition}{Definition}[subsection]
%\newmdtheoremenv{example}{Example}[subsection]
\AtBeginEnvironment{defn}{\begin{minipage}{\textwidth}}
\AtEndEnvironment{defn}{\end{minipage}}
%\AtBeginEnvironment{example}{\begin{minipage}{\textwidth}}
%\AtEndEnvironment{example}{\end{minipage}}
\newcommand{\iu}{{i\mkern1mu}}

\setlength{\gridsize}{5mm}
\setlength{\gridlinewidth}{0.1pt}

\printanswers
\DeclareMathSizes{12}{12}{12}{12}

%%%%%%%%%%%%%%%%%%%%%%%%
% Create bars around subsubsection
%%%%%%%%%%%%%%%%%%%%%%%%

\titleformat{\subsubsection}
   {\large\bfseries}% format
   {}% label
   {0pt}% sep
   {\titlerule \vspace{.1in} #1}% before code
      [{\titlerule[0.4pt]\vspace{.1in}}]% after code
\titlespacing{\subsubsection}
   {0pt}% left
   {0pt}% before sep
   {\baselineskip}% after sep
   
%%%%%%%%%%%%%%%%%%%%%%%
% Create line break after definition label
%%%%%%%%%%%%%%%%%%%%%%%   
\newtheoremstyle{break}
  {\topsep}{\topsep}%
  {}{}%\itshape
  {\bfseries}{}%
  {\newline}{}%
\theoremstyle{break}
\newmdtheoremenv{definition}{Definition}[subsection]
\theoremstyle{break}
\newtheorem{example}{Example}[subsection]

%%%%%%%%%%%%%%%%%%%%%%
% start document
% set section, subsection (use n-1 for sub)
%%%%%%%%%%%%%%%%%%%%%%


\begin{document}
\setcounter{section}{4}
\setcounter{subsection}{2}

\subsection{The Elimination or Addition Method}

\vspace{.15in}

\begin{mdframed}
\textbf{Process - Elimination/Addition Method}
\begin{enumerate}
\item Rewrite both equations in standard form ($Ax + By = C$).
\item Multiply either or both equations by some constant in order to make either the coefficients of the $x$ or $y$ terms match.
\item Add or subtract the two equations as appropriate.
\item Solve the new one variable equation.
\item Substitute this value into either original equation to solve for the second missing value.
\item Rewrite your solution as a point.
\end{enumerate}
\end{mdframed}

\vspace{.15in}

\begin{example}
Solve the following with the elimination method:
\[\systeme{x+y=5,x-y=9}\]
\vspace{2in}
\end{example}

\newpage

\begin{example}
Solve the following with the elimination method:
\[\systeme{4x-y=22,3x+4y=26}\]
\vspace{3in}
\end{example}

\begin{example}
Solve the following with the elimination method:
\[\systeme{4x+5y=3,2x-3y=7}\]
\vspace{3in}
\end{example}

\newpage

\begin{example}
Solve the following with the elimination method:
\[\systeme{2x = 9 + 3y,3x = 8-4y}\]
\vspace{4in}
\end{example}

\begin{example}
Solve the following with the elimination method:
\[\systeme{x+2y=4,3x+6y=13}\]
\end{example}

\newpage

\begin{example}
Solve the following with the elimination method:
\[\systeme{x-5y=7,3x-15y=21}\]
\vspace{4in}
\end{example}

\end{document}