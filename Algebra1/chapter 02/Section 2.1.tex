\documentclass[addpoints,12pt]{exam}
\usepackage{amsmath}
\usepackage{amsthm}
\usepackage{amsfonts}
\usepackage{systeme}
\usepackage{graphicx}
\usepackage{caption}
\usepackage{xfrac}
\usepackage{physics}
\usepackage{microtype}
\usepackage{eulervm}
%\usepackage[framemethod=tikz]{mdframed}
\usepackage{thmtools}
\usepackage{etoolbox}
%\usepackage{fouriernc}
\usepackage{mdframed}
\usepackage[overload]{empheq}
\usepackage{adjustbox}
\usepackage{enumitem}
\usepackage[explicit]{titlesec}

\pagestyle{headandfoot}
\runningfootrule
\firstpageheadrule
\runningheadrule

\newcommand{\class}{Math 0097}
\newcommand{\sem}{2211}
\newcommand{\due}{}
\newcommand{\sect}{2.1}
\newcommand{\topic}{The Addition Property of Equality}

\firstpageheader{\class}{\sect - \topic}{}
\runningheader{\class}{\sect - \topic}{}
\firstpagefooter{\class}{}{Page \thepage\ of \numpages}
\runningfooter{\class}{}{Page \thepage\ of \numpages}

\newif\ifprintselected
\printselectedtrue
%\printselectedfalse

\newenvironment{select}
{\ifprintselected
	\printanswers
	\fi
}
{}

\theoremstyle{definition}
\newtheorem{theorem}{Theorem}
%\newtheorem{example}{Example}[subsection]
%\newtheorem{definition}{Definition}
%\newmdtheoremenv{definition}{Definition}[subsection]
%\newmdtheoremenv{example}{Example}[subsection]
\AtBeginEnvironment{defn}{\begin{minipage}{\textwidth}}
\AtEndEnvironment{defn}{\end{minipage}}
%\AtBeginEnvironment{example}{\begin{minipage}{\textwidth}}
%\AtEndEnvironment{example}{\end{minipage}}
\newcommand{\iu}{{i\mkern1mu}}

\setlength{\gridsize}{5mm}
\setlength{\gridlinewidth}{0.1pt}

\printanswers
\DeclareMathSizes{12}{12}{12}{12}

%%%%%%%%%%%%%%%%%%%%%%%%
% Create bars around subsubsection
%%%%%%%%%%%%%%%%%%%%%%%%

\titleformat{\subsubsection}
   {\large\bfseries}% format
   {}% label
   {0pt}% sep
   {\titlerule \vspace{.1in} #1}% before code
      [{\titlerule[0.4pt]\vspace{.1in}}]% after code
\titlespacing{\subsubsection}
   {0pt}% left
   {0pt}% before sep
   {\baselineskip}% after sep
   
%%%%%%%%%%%%%%%%%%%%%%%
% Create line break after definition label
%%%%%%%%%%%%%%%%%%%%%%%   
\newtheoremstyle{break}
  {\topsep}{\topsep}%
  {}{}%\itshape
  {\bfseries}{}%
  {\newline}{}%
\theoremstyle{break}
\newmdtheoremenv{definition}{Definition}[subsection]
\theoremstyle{break}
\newtheorem{example}{Example}[subsection]

%%%%%%%%%%%%%%%%%%%%%%
% start document
% set section, subsection (use n-1 for sub)
%%%%%%%%%%%%%%%%%%%%%%


\begin{document}
\setcounter{section}{2}
\setcounter{subsection}{0}

\subsection{The Addition Property of Equality}

\vspace{.25in}

\begin{definition}[Linear Equation]
any equation that can be written as $ax+b=c$ where $a\neq 0$
\end{definition}

\vspace{.15in}

\begin{minipage}{.5\textwidth}
\textbf{Linear:}
\begin{itemize}
\item $2x - 5 = 3$
\item $-3x = 18$
\item $x = 4.5$
\end{itemize}
\end{minipage}%
\begin{minipage}{.5\textwidth}
\textbf{Non-Linear:}
\begin{itemize}
\item $2x^2 + 3 = 7$
\item $-\dfrac{1}{2x}=4$
\item $\abs{x} = 6$
\end{itemize}
\end{minipage}%


\vspace{.15in}

\begin{definition}[Addition Property of Equality]
If $a = b$, then $a + c = b + c$.
\end{definition}
\vspace{.15in}

\begin{example}
Solve for $x$: \[ x - 5 = 12\]
\vspace{1in}
\end{example}

\begin{example}
Solve for $x$: \[ y + 2.8 = 5.09\]
\vspace{1in}
\end{example}

\newpage

\begin{example}
Solve for x: \[-\dfrac{1}{2} = x - \dfrac{3}{4}\]
\vspace{2in}
\end{example}

\noindent \emph{Note:} These values we have solved for are called \textbf{solutions} or \textbf{roots} of the equation. They are values of the independent variable that make the statement true. \emph{Linear equations} only have \textbf{at most one solution}. \emph{Non-linear equations} may have \textbf{more than one solution}.

\vspace{.15in}
\noindent Before solving any equation, \textbf{always} simplify and combine like terms.
\vspace{.15in}

\begin{example}
Solve for $y$: \[8y + 7 - 7y - 10 = 6 + 4\]
\vspace{2in}
\end{example}

\newpage

\begin{example}
Solve for $x$: \[7x = 12 + 6x\]
\vspace{2in}
\end{example}

\begin{example}
Solve for $y$: \[3y - 9 = 2y + 6\]
\end{example}
\end{document}