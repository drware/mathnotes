\documentclass[addpoints,12pt]{exam}
\usepackage{amsmath}
\usepackage{amsthm}
\usepackage{amsfonts}
\usepackage{systeme}
\usepackage{graphicx}
\usepackage{caption}
\usepackage{xfrac}
\usepackage{physics}
\usepackage{microtype}
\usepackage{eulervm}
%\usepackage[framemethod=tikz]{mdframed}
\usepackage{thmtools}
\usepackage{etoolbox}
%\usepackage{fouriernc}
\usepackage{mdframed}
\usepackage[overload]{empheq}
\usepackage{adjustbox}
\usepackage{enumitem}
\usepackage[explicit]{titlesec}

\pagestyle{headandfoot}
\runningfootrule
\firstpageheadrule
\runningheadrule

\newcommand{\class}{Math 0098}
\newcommand{\sem}{2211}
\newcommand{\due}{}
\newcommand{\sect}{7.1}
\newcommand{\topic}{Rational Expressions \& Simplifying}

\firstpageheader{\class}{\sect - \topic}{}
\runningheader{\class}{\sect - \topic}{}
\firstpagefooter{\class}{}{Page \thepage\ of \numpages}
\runningfooter{\class}{}{Page \thepage\ of \numpages}

\newif\ifprintselected
\printselectedtrue
%\printselectedfalse

\newenvironment{select}
{\ifprintselected
	\printanswers
	\fi
}
{}

\theoremstyle{definition}
\newtheorem{theorem}{Theorem}
\newtheorem{example}{Example}[subsection]
%\newtheorem{definition}{Definition}
\newtheorem{definition}{Definition}[subsection]
%\newmdtheoremenv{example}{Example}[subsection]
\AtBeginEnvironment{defn}{\begin{minipage}{\textwidth}}
\AtEndEnvironment{defn}{\end{minipage}}
%\AtBeginEnvironment{example}{\begin{minipage}{\textwidth}}
%\AtEndEnvironment{example}{\end{minipage}}
\newcommand{\iu}{{i\mkern1mu}}

\setlength{\gridsize}{5mm}
\setlength{\gridlinewidth}{0.1pt}

\printanswers
\DeclareMathSizes{12}{12}{12}{12}

\titleformat{\subsubsection}
   {\large\bfseries}% format
   {}% label
   {0pt}% sep
   {\titlerule \vspace{.1in} #1}% before code
      [{\titlerule[0.4pt]\vspace{.1in}}]% after code
\titlespacing{\subsubsection}
   {0pt}% left
   {0pt}% before sep
   {\baselineskip}% after sep


\begin{document}
\setcounter{section}{7}
\setcounter{subsection}{0}

\subsection{Rational Expressions \& Simplifying}

\begin{mdframed}
\begin{definition}[rational expression]
the quotient of two polynomials; $\dfrac{x-3}{2}$, $\dfrac{-3}{x^2+4x}$
\end{definition}
\end{mdframed}
\vspace{.25in}

\noindent Rational expressions are \emph{undefined} for a value of $x$ if the denominator becomes 0.

\vspace{.25in}

\begin{example}
Find the numbers which make the given rational expressions undefined.
\begin{enumerate}
\item $\dfrac{7x-28}{8x-4}$
\vspace{1.25in}
\item $\dfrac{8x-40}{x^2+3x-28}$
\vspace{1.25in}
\end{enumerate}
\end{example}

\subsubsection*{Simplifying Rational Expressions}
\noindent A rational expression is considered \emph{simplified} when the numerator and denominator share no factors, much like a reduced fraction.

\vspace{.25in}
\begin{mdframed}
\textbf{Process}
\begin{enumerate}
\item factor both the numerator and denominator fully
\item divide \& cancel any common factors
\end{enumerate}
\end{mdframed}

\newpage

\begin{example}
Simplify fully: \[\dfrac{7x+28}{21x}\]
\vspace{1.5in}
\end{example}
\begin{example}
Simplify fully: \[\dfrac{x^3-x^2}{7x-7}\]
\vspace{1.5in}
\end{example}
\begin{example}
Simplify fully: \[\dfrac{x^2-1}{x^2+2x+1}\]
\vspace{1.5in}
\end{example}

\newpage

\subsubsection*{Opposite Factors}
\noindent Remember that there are two values we can always factor out: 1 (boring) and -1 (less boring and sometimes useful).

\noindent If we have something similar to $\dfrac{x-2}{2-x}$ we may notice that there is some similarity between the numerator and denominator. If we were to factor out a -1 from either, we would see that they share a common factor that could be canceled.

\vspace{.25in}

\begin{example}
Simplify fully: \[\dfrac{x-2}{2-x}\]
\vspace{1.5in}
\end{example}

\begin{example}
Simplify fully: \[\dfrac{9x^2-49}{28-12x}\]
\end{example}

\end{document}