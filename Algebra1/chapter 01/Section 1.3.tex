\documentclass[addpoints,12pt]{exam}
\usepackage{amsmath}
\usepackage{amsthm}
\usepackage{amsfonts}
\usepackage{systeme}
\usepackage{graphicx}
\usepackage{caption}
\usepackage{xfrac}
\usepackage{physics}
\usepackage{microtype}
\usepackage{eulervm}
%\usepackage[framemethod=tikz]{mdframed}
\usepackage{thmtools}
\usepackage{etoolbox}
%\usepackage{fouriernc}
\usepackage{mdframed}
\usepackage[overload]{empheq}
\usepackage{adjustbox}
\usepackage{enumitem}
\usepackage[explicit]{titlesec}

\pagestyle{headandfoot}
\runningfootrule
\firstpageheadrule
\runningheadrule

\newcommand{\class}{Math 0097}
\newcommand{\sem}{2211}
\newcommand{\due}{}
\newcommand{\sect}{1.3}
\newcommand{\topic}{The Real Numbers}

\firstpageheader{\class}{\sect - \topic}{}
\runningheader{\class}{\sect - \topic}{}
\firstpagefooter{\class}{}{Page \thepage\ of \numpages}
\runningfooter{\class}{}{Page \thepage\ of \numpages}

\newif\ifprintselected
\printselectedtrue
%\printselectedfalse

\newenvironment{select}
{\ifprintselected
	\printanswers
	\fi
}
{}

\theoremstyle{definition}
\newtheorem{theorem}{Theorem}
\newtheorem{example}{Example}[subsection]
%\newtheorem{definition}{Definition}
\newtheorem{definition}{Definition}[subsection]
%\newmdtheoremenv{example}{Example}[subsection]
\AtBeginEnvironment{defn}{\begin{minipage}{\textwidth}}
\AtEndEnvironment{defn}{\end{minipage}}
%\AtBeginEnvironment{example}{\begin{minipage}{\textwidth}}
%\AtEndEnvironment{example}{\end{minipage}}
\newcommand{\iu}{{i\mkern1mu}}

\setlength{\gridsize}{5mm}
\setlength{\gridlinewidth}{0.1pt}

\printanswers
\DeclareMathSizes{12}{12}{12}{12}

\titleformat{\subsubsection}
   {\large\bfseries}% format
   {}% label
   {0pt}% sep
   {\titlerule \vspace{.1in} #1}% before code
      [{\titlerule[0.4pt]\vspace{.1in}}]% after code
\titlespacing{\subsubsection}
   {0pt}% left
   {0pt}% before sep
   {\baselineskip}% after sep


\begin{document}
\setcounter{section}{1}
\setcounter{subsection}{2}

\subsection{The Real Numbers}
\noindent We like to classify numbers and groups of numbers by similar properties. A few of these groupings may seem familiar or intuitive, while others may not.

\vspace{.25in}

\begin{definition}[set]
one of the most fundamental objects in math; a collection of \emph{distinct} items; often denoted using curly brackets - \{\}
\end{definition}

\vspace{.25in}

\begin{example}
Which of the following are sets and which are not?
\begin{enumerate}
\item $A = \{a,b,c\}$
\item $B = \{97, 98, 110, 133\}$
\item $C = \{20, 24, 26, 20\}$
\item $D = \{1, 2, 3, \dots\}$
\item $E = \{\dots,-3,-2,-1,0,1,2,3,\dots\}$
\end{enumerate}
\end{example}

\vspace{.25in}

\begin{definition}[Real Numbers]
abbreviated as $\mathbb{R}$; the numbers that we use in day-to-day life; the largest set of numbers that we'll use in this class
\end{definition}
\vspace{.25in}

\begin{definition}[Natural Numbers]
abbreviated as $\mathbb{N}$; also known as the counting numbers; $\mathbb{N} = \{1,2,3,\dots\}$
\end{definition}
\vspace{.25in}

\begin{definition}[Whole Numbers]
the natural numbers with 0 included;\\ $\{0,1,2,3,\dots\}$
\end{definition}
\vspace{.25in}

\begin{definition}[Integers]
abbreviated as $\mathbb{Z}$; all natural numbers along with their negatives and 0;\\
$\mathbb{Z} = \{\dots,-3,-2,-1,0,1,2,3,\dots\}$\\
$\mathbb{Z}^+ = \{1,2,3,\dots\} = \mathbb{N}$\\
$\mathbb{Z}^- = \{\dots,-3,-2,-1\}$
\end{definition}
\vspace{.25in}

\subsubsection*{The Number Line}
\noindent The number line is a graphical representation of the real numbers. It is a one-dimensional object that displays the real numbers from $-\infty$ to $\infty$ going from left to right.

\begin{example}
Draw a number line and graph the following:
\vspace{.5in}

\begin{enumerate}
\item $-2$
\item $0$
\item $3$
\item $\dfrac{1}{2}$
\end{enumerate}
\end{example}

\subsubsection*{Rational Numbers}

\begin{definition}[Rational Numbers]
root word: ratio\\
abbreviated as $\mathbb{Q}$; all numbers that can be expressed as the quotient of two integers where the denominator is not 0; essentially, all numbers that can be written as fractions
\end{definition}

\vspace{.25in}

\noindent We can easily see that integers are rational numbers. How could we show that the integers 10 and -15 are rational?

\vspace{1in}

\noindent What about mixed numbers/mixed fractions?

\vspace{1in}

\noindent What happens if we add or subtract two rational numbers? Is that new number rational?
\newpage


\noindent How about decimals? Are they rational numbers?
\vspace{1in}

\subsubsection*{Graphing Rational Numbers}

\begin{example}
Draw a number line and plot the following rational numbers:
\vspace{.5in}
\begin{enumerate}
\item $\dfrac{9}{2}$
\item $-1.2$
\item $3\dfrac{1}{3}$
\end{enumerate}
\end{example}
\vspace{.25in}

\subsubsection*{Converting Rationals to Decimals}
\noindent There are two easy ways to convert rational numbers to a decimal format - either long division or use a calculator.

\vspace{.25in}

\begin{example}
Convert the rational numbers $\dfrac{3}{8}$ and $\dfrac{2}{5}$ to a decimal.
\vspace{1in}
\end{example}
\newpage

\subsubsection*{Irrational Numbers}
\noindent Most things in math have an inverse or a complement (opposite). The complement of the rational numbers are the \emph{irrationals}.
\vspace{.25in}

\begin{definition}[Irrational Numbers]
numbers that are not rational; numbers that can't be written as the quotient of two integers
\end{definition}

\vspace{.25in}

\noindent A famous irrational number is $\pi$. Most people can name the first few digits of $\pi$ as 3.14 and if you've got a good memory, you might be able to get to 3.14159; however, $\pi$ continues on and \emph{does not terminate}, meaning that there are an infinite number of decimals places. These decimal values do not follow a pattern and they do not repeat. Therefor, $\pi$ is irrational. We can, however, approximate $\pi$ as $\pi \approx \dfrac{22}{7} = 3.142857\dots$.

\vspace{.25in}

\noindent We could also consider another famous example, $\sqrt{2}$. In general, radicals are irrational if they cannot be simplified. We can find that $\sqrt{2} = 1.41414\dots$. There is a proof that is often taught in introductory discrete math classes that proves the irrationality of $\sqrt{2}$, but we will leave that be for now.

\vspace{.25in}

\noindent What is an example of a radical that isn't irrational?

\vspace{.5in}

\begin{example}
Consider the following set of numbers. Put each number into the correct category. Note that numbers may belong to more than one set.
\[\{-9,-1.3,0,0.\bar{3},\dfrac{\pi}{2},\sqrt{9},\sqrt{10}\}\]

\begin{itemize}
\item Natural?
\item Whole?
\item Integers?
\item Rational?
\item Irrational?
\item Real?
\end{itemize}
\end{example}

\subsubsection*{Ordering the Reals}

\noindent Reals are considered \emph{well ordered} meaning that if we have two real numbers, we can determine which is larger and which is smaller. We typically denote this with inequality symbols $(<, >, \le, \ge)$. On a number line, numbers are ordered from smallest to largest going from left to right.

\begin{example}\mbox{}\\
\begin{itemize}
\item $14 > 5$ because 14 is right of 5 on the number line
\item $-19 < -6$ because -19 is left of -6 on the number line
\item $\dfrac{1}{4} < \dfrac{1}{2}$ because $\dfrac{1}{4}$ is left of $\dfrac{1}{2}$ on the number line
\end{itemize}
\end{example}

\vspace{.25in}

\noindent We can modify the inequality symbols to include a number as well.
\begin{itemize}
\item $a \le b$ reads as "a is less than or equal to b"
\item $a \ge b$ reads as "a is greater than or equal to b"
\end{itemize}

\begin{example}
True or false?
\begin{enumerate}
\item $-2 \le 3$
\item $-2 \ge -2$
\item $-4 \ge 1$
\end{enumerate}
\end{example}

\begin{definition}[Absolute Value]
represents the distance from zero on a number line; makes a number positive; denoted as $\abs{a}$
\end{definition}

\vspace{.25in}

\begin{example}
Find each of the following:
\begin{enumerate}
\item $\abs{-4} = $
\item $\abs{6} = $
\item $\abs{-\sqrt{2}} = $
\item $\abs{3-5} = $
\item $-2\abs{-4+3} = $
\end{enumerate}
\end{example}

\end{document}