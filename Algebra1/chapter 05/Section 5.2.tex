\documentclass[addpoints,12pt]{exam}
\usepackage{amsmath}
\usepackage{amsthm}
\usepackage{amsfonts}
\usepackage{systeme}
\usepackage{graphicx}
\usepackage{caption}
\usepackage{xfrac}
\usepackage{physics}
\usepackage{microtype}
\usepackage{eulervm}
%\usepackage[framemethod=tikz]{mdframed}
\usepackage{thmtools}
\usepackage{etoolbox}
%\usepackage{fouriernc}
\usepackage{mdframed}
\usepackage[overload]{empheq}
\usepackage{adjustbox}
\usepackage{enumitem}
\usepackage[explicit]{titlesec}
% adds in \varnothing for empty set
\usepackage{amssymb}
% adds in formated SI units
%\usepackage{siunitx}
\usepackage{pgfplots}
\usepackage{multirow}
\usepackage{array}

\pagestyle{headandfoot}
\runningfootrule
\firstpageheadrule
\runningheadrule

\newcommand{\class}{Math 0097}
\newcommand{\sem}{2211}
\newcommand{\due}{}
\newcommand{\sect}{5.2}
\newcommand{\topic}{Multiplying Polynomials}

\firstpageheader{\class}{\sect - \topic}{}
\runningheader{\class}{\sect - \topic}{}
\firstpagefooter{\class}{}{Page \thepage\ of \numpages}
\runningfooter{\class}{}{Page \thepage\ of \numpages}

\newif\ifprintselected
\printselectedtrue
%\printselectedfalse

\newenvironment{select}
{\ifprintselected
	\printanswers
	\fi
}
{}

\theoremstyle{definition}
\newtheorem{theorem}{Theorem}
%\newtheorem{example}{Example}[subsection]
%\newtheorem{definition}{Definition}
%\newmdtheoremenv{definition}{Definition}[subsection]
%\newmdtheoremenv{example}{Example}[subsection]
\AtBeginEnvironment{defn}{\begin{minipage}{\textwidth}}
\AtEndEnvironment{defn}{\end{minipage}}
%\AtBeginEnvironment{example}{\begin{minipage}{\textwidth}}
%\AtEndEnvironment{example}{\end{minipage}}
\newcommand{\iu}{{i\mkern1mu}}

\setlength{\gridsize}{5mm}
\setlength{\gridlinewidth}{0.1pt}

\printanswers
\DeclareMathSizes{12}{12}{12}{12}

%%%%%%%%%%%%%%%%%%%%%%%%
% Create bars around subsubsection
%%%%%%%%%%%%%%%%%%%%%%%%

\titleformat{\subsubsection}
   {\large\bfseries}% format
   {}% label
   {0pt}% sep
   {\titlerule \vspace{.1in} #1}% before code
      [{\titlerule[0.4pt]\vspace{.1in}}]% after code
\titlespacing{\subsubsection}
   {0pt}% left
   {0pt}% before sep
   {\baselineskip}% after sep
   
%%%%%%%%%%%%%%%%%%%%%%%
% Create line break after definition label
%%%%%%%%%%%%%%%%%%%%%%%   
\newtheoremstyle{break}
  {\topsep}{\topsep}%
  {}{}%\itshape
  {\bfseries}{}%
  {\newline}{}%
\theoremstyle{break}
\newmdtheoremenv{definition}{Definition}[subsection]
\theoremstyle{break}
\newtheorem{example}{Example}[subsection]

%%%%%%%%%%%%%%%%%%%%%%
% start document
% set section, subsection (use n-1 for sub)
%%%%%%%%%%%%%%%%%%%%%%


\begin{document}
\setcounter{section}{5}
\setcounter{subsection}{1}

\subsection{Multiplying Polynomials}

\vspace{.15in}

\subsubsection*{Exponent Rules}
\noindent We know that $a^n = a\cdot a \cdot a \cdots a$. For example, we know that $2^4 = 2\cdot 2\cdot 2\cdot 2 = 16$.
\\\\
What about $2^4\cdot 2^3$ though?
\[2^4\cdot 2^3 = (2\cdot 2\cdot 2\cdot 2)\cdot(2\cdot 2\cdot 2) = 2\cdot 2\cdot 2\cdot 2\cdot 2\cdot 2\cdot 2 = 2^7\]

\noindent This gives us our first rule for exponents, the product rule:
\begin{mdframed}
\textbf{Product Rule}\mbox{}\\
$b^m\cdot b^n = b^{m+n}$
\end{mdframed}

\begin{example}
Find each of the following:
\begin{enumerate}
\item $2^2\cdot 2^3 = $
\vspace{.25in}
\item $x^6\cdot x^4 = $
\vspace{.25in}
\item $y\cdot y^7 = $
\vspace{.25in}
\item $y^4\cdot y^3\cdot y^2 = $
\vspace{.25in}
\end{enumerate}
\end{example}

Now, what if we raised an exponent to an exponent, such as $(2^3)^2$?
\begin{eqnarray*}
(2^3)^2 &=& (2^3)(2^3)\\
&=& (2\cdot 2\cdot 2)\cdot(2\cdot 2\cdot 2)\\
&=& 2\cdot 2\cdot 2\cdot 2\cdot 2\cdot 2\\
&=& 2^6 = 64
\end{eqnarray*}

\noindent This can be generalized into our second rule, the power rule:
\begin{mdframed}
\textbf{Power Rule}\mbox{}\\
$(b^m)^n = b^{mn}$
\end{mdframed}

\newpage
\begin{example}
Find each of the following:
\begin{enumerate}
\item $(3^4)^5 = $
\vspace{.25in}
\item $(x^9)^{10} = $
\vspace{.25in}
\item $[(-5)^7]^3 = $
\vspace{.25in}
\end{enumerate}
\end{example}
\vspace{.25in}

\noindent Let's generalize this further - what if we had a product raised to a power? For example, $(2x^2)^3$?
\begin{eqnarray*}
(2x^2)^3 &=& (2x^2)\cdot(2x^2)\cdot(2x^2)\\
&=& (2\cdot 2\cdot 2)\cdot (x^2\cdot x^2\cdot x^2)\\
&=& 2^3 \cdot (x^2)^3\\
&=& 2^3\cdot x^{2\cdot 3} = 8x^6
\end{eqnarray*}

\vspace{.15in}

\begin{mdframed}
\textbf{General Power Rule}\mbox{}\\
$(ab)^n = a^nb^n$
\end{mdframed}

\vspace{.15in}

\begin{example}
Find each of the following:
\begin{enumerate}
\item $(2x)^4 = $
\vspace{.5in}
\item $(-4y^2)^3 = $
\end{enumerate}
\end{example}


\newpage

\subsubsection*{Multiplying Polynomials}
\vspace{.15in}
\begin{example}
Find the following:
\[(7x^2)(10x)\]
\vspace{.5in}
\end{example}

\begin{example}
Find the following:
\[(-5x^4)(4x^5)\]
\vspace{.5in}
\end{example}

\begin{example}
Find the following:
\[3x(x+5)\]
\vspace{1in}
\end{example}

\begin{example}
Find the following:
\[6x^2(5x^3 - 2x + 3)\]
\end{example}

\newpage

\noindent What if both factors have more than one term? \textbf{FOIL}
\vspace{.15in}
\begin{mdframed}
\textbf{FOIL} -- \textbf{F}irst, \textbf{O}utside, \textbf{I}nside, \textbf{L}ast
\end{mdframed}
\vspace{.15in}

\begin{example}
Find the following:
\[(x+4)(x+5)\]
\vspace{.5in}
\end{example}

\begin{example}
Find the following:
\[(5x+3)(2x-7)\]
\vspace{.5in}
\end{example}

\begin{example}
Find the following:
\[(5x+2)(x^2-4x+3)\]
\vspace{1in}
\end{example}

\begin{example}
Find the following:
\[(3x^2-2x)(2x^3 - 5x^2 + 4x)\]
\end{example}


\end{document}