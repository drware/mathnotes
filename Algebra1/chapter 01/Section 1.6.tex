\documentclass[addpoints,12pt]{exam}
\usepackage{amsmath}
\usepackage{amsthm}
\usepackage{amsfonts}
\usepackage{systeme}
\usepackage{graphicx}
\usepackage{caption}
\usepackage{xfrac}
\usepackage{physics}
\usepackage{microtype}
\usepackage{eulervm}
%\usepackage[framemethod=tikz]{mdframed}
\usepackage{thmtools}
\usepackage{etoolbox}
%\usepackage{fouriernc}
\usepackage{mdframed}
\usepackage[overload]{empheq}
\usepackage{adjustbox}
\usepackage{enumitem}
\usepackage[explicit]{titlesec}

\pagestyle{headandfoot}
\runningfootrule
\firstpageheadrule
\runningheadrule

\newcommand{\class}{Math 0097}
\newcommand{\sem}{2211}
\newcommand{\due}{}
\newcommand{\sect}{1.6}
\newcommand{\topic}{Subtraction of Real Numbers}

\firstpageheader{\class}{\sect - \topic}{}
\runningheader{\class}{\sect - \topic}{}
\firstpagefooter{\class}{}{Page \thepage\ of \numpages}
\runningfooter{\class}{}{Page \thepage\ of \numpages}

\newif\ifprintselected
\printselectedtrue
%\printselectedfalse

\newenvironment{select}
{\ifprintselected
	\printanswers
	\fi
}
{}

\theoremstyle{definition}
\newtheorem{theorem}{Theorem}
\newtheorem{example}{Example}[subsection]
%\newtheorem{definition}{Definition}
\newtheorem{definition}{Definition}[subsection]
%\newmdtheoremenv{example}{Example}[subsection]
\AtBeginEnvironment{defn}{\begin{minipage}{\textwidth}}
\AtEndEnvironment{defn}{\end{minipage}}
%\AtBeginEnvironment{example}{\begin{minipage}{\textwidth}}
%\AtEndEnvironment{example}{\end{minipage}}
\newcommand{\iu}{{i\mkern1mu}}

\setlength{\gridsize}{5mm}
\setlength{\gridlinewidth}{0.1pt}

\printanswers
\DeclareMathSizes{12}{12}{12}{12}

\titleformat{\subsubsection}
   {\large\bfseries}% format
   {}% label
   {0pt}% sep
   {\titlerule \vspace{.1in} #1}% before code
      [{\titlerule[0.4pt]\vspace{.1in}}]% after code
\titlespacing{\subsubsection}
   {0pt}% left
   {0pt}% before sep
   {\baselineskip}% after sep


\begin{document}
\setcounter{section}{1}
\setcounter{subsection}{5}

\subsection{Subtraction of Real Numbers}

\noindent Subtraction can be thought of as adding a negative number. In fact, we define subtraction as: \[a - b = a + (-b)\]

\noindent We can use this definition to rewrite expressions as sums (addition).

\vspace{.1in}

\begin{example}
Rewrite as a sum: \[2xy - 13y - 6x\]
\vspace{.75in}
\end{example}

\noindent However, we typically use this property to go the other direction instead. When we have multiple signs, we rewrite them as one sign. If we have $+ +$ or $- -$, we replace with a single $+$. If we have opposite signs - $+ - $ or $- +$, we replace with a single $-$.

\vspace{.1in}

\begin{example}
Evaluate: \[-6 - (-3) + (+8) + (-11)\]
\vspace{1in}
\end{example}

\begin{example}
Evaluate and simplify fully: \[15 - (-3x) + 8x - (-10)\]
\vspace{1.5in}
\end{example}

\newpage

\begin{example}
Evaluate: \[ \abs{-9 - (-3 + 7)} - \abs{-17 - (-2)}\]
\end{example}
\end{document}