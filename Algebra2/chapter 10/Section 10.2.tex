\documentclass[addpoints,12pt]{exam}
\usepackage{amsmath}
\usepackage{amsthm}
\usepackage{amsfonts}
\usepackage{systeme}
\usepackage{graphicx}
\usepackage{caption}
\usepackage{xfrac}
\usepackage{physics}
\usepackage{microtype}
\usepackage{eulervm}
%\usepackage[framemethod=tikz]{mdframed}
\usepackage{thmtools}
\usepackage{etoolbox}
%\usepackage{fouriernc}
\usepackage{mdframed}
\usepackage[overload]{empheq}
\usepackage{adjustbox}
\usepackage{enumitem}

\pagestyle{headandfoot}
\runningfootrule
\firstpageheadrule
\runningheadrule

\newcommand{\class}{Math 0098}
\newcommand{\sem}{2201}
\newcommand{\due}{}
\newcommand{\sect}{10.2}
\newcommand{\topic}{Rational Exponents}

\firstpageheader{\class}{\sect - \topic}{}
\runningheader{\class}{\sect - \topic}{}
\firstpagefooter{\class}{}{Page \thepage\ of \numpages}
\runningfooter{\class}{}{Page \thepage\ of \numpages}

\newif\ifprintselected
\printselectedtrue
%\printselectedfalse

\newenvironment{select}
{\ifprintselected
	\printanswers
	\fi
}
{}

\theoremstyle{definition}
\newtheorem{theorem}{Theorem}
\newtheorem{example}{Example}[subsection]
%\newtheorem{definition}{Definition}
\newtheorem{definition}{Definition}[subsection]
%\newmdtheoremenv{example}{Example}[subsection]
\AtBeginEnvironment{defn}{\begin{minipage}{\textwidth}}
\AtEndEnvironment{defn}{\end{minipage}}
%\AtBeginEnvironment{example}{\begin{minipage}{\textwidth}}
%\AtEndEnvironment{example}{\end{minipage}}
\newcommand{\iu}{{i\mkern1mu}}

\setlength{\gridsize}{5mm}
\setlength{\gridlinewidth}{0.1pt}

\printanswers
\DeclareMathSizes{12}{12}{12}{12}

\begin{document}
\setcounter{section}{10}
\setcounter{subsection}{1}

\subsection{Rational Exponents}

\begin{mdframed}
\textbf{Rational Exponent Definitions}

For some integers $m$ and $n$, we have the following:
\begin{enumerate}
\item $a^{\sfrac{1}{n}} = \sqrt[n]{a}$
\item $a^{\sfrac{m}{n}} = \sqrt[n]{a^m} = \sqrt[n]{a}^m$
\end{enumerate}
\end{mdframed}

\vspace{.25in}
\begin{example}
Rewrite each of the following using radical notations. Simplify if possible.
\begin{enumerate}
\begin{minipage}{.5\textwidth}
\item $25^{\sfrac{1}{2}}$
\vspace{1.25in}
\item $(-8)^{\sfrac{1}{3}}$
\vspace{1.25in}
\item $(5xy^2)^{\sfrac{1}{4}}$
\vspace{1.25in}
\end{minipage}%
\begin{minipage}{.5\textwidth}
\item $8^{\sfrac{4}{3}}$
\vspace{1.25in}
\item $25^{\sfrac{3}{2}}$
\vspace{1.25in}
\item $-81^{\sfrac{3}{4}}$
\vspace{1.25in}
\end{minipage}%
\end{enumerate}
\end{example}


\newpage

\begin{example}
Rewrite each of the following using rational exponents. Simplify if possible.
\begin{enumerate}
\begin{minipage}{.5\textwidth}
\item $\sqrt[4]{5xy}$
\vspace{1.25in}
\item $\sqrt[5]{\dfrac{a^3b}{2}}$
\vspace{1.25in}
\end{minipage}%
\begin{minipage}{.5\textwidth}
\item $\sqrt[3]{6^4}$
\vspace{1.25in}
\item $\sqrt[5]{2xy}^7$
\vspace{1.25in}
\end{minipage}%
\end{enumerate}
\end{example}

\subsubsection*{Properties of Rational Exponents}
\begin{enumerate}[itemsep=.5cm,label=\theenumi)]
\adjustbox{valign=t}{\begin{minipage}{.5\textwidth}
\item Product Rule:\\$a^m\cdot a^n = a^{m+n}$
\item Quotient Rule:\\$\dfrac{a^m}{a^n} = a^{m-n}$
\item Power-to-Power Rule:\\$\left(a^m\right)^n = a^{mn}$
\item Product-to-Power Rule:\\$\left(ab\right)^m = a^mb^m$
\item Quotient-to-Power Rule:\\$\left(\dfrac{a}{b}\right)^m = \dfrac{a^m}{b^m}$
\end{minipage}}%
\adjustbox{valign=t}{\begin{minipage}{.5\textwidth}
\item $a^m = a^n$ iff $m = n$
\item $a^m = b^m$ iff $a = b, m\neq 0$
\item Negative Exponents:\\\vspace{-.35in}
	\begin{enumerate}
	\item $a^{-m} = \dfrac{1}{a^m}$
	\item $ba^{-m} = \dfrac{b}{a^m}$
	\item $\dfrac{1}{a^{-m}} = a^m$
	\item $\dfrac{b}{a^{-m}} = ba^m$
	\end{enumerate}
\item Zero Exponent Rule:\\$a^0 = 1$
\end{minipage}}%
\end{enumerate}

\newpage

\begin{example}
Use the properties of rational exponents to simplify the following statements:
\begin{enumerate}
\begin{minipage}{.5\textwidth}
\item $7^{\sfrac{1}{2}}\cdot 7^{\sfrac{1}{3}}$
\vspace{1.25in}
\item $\dfrac{50x^{\sfrac{1}{3}}}{10x^{\sfrac{4}{3}}}$
\vspace{1.25in}
\end{minipage}%
\begin{minipage}{.5\textwidth}
\item $\left(9.1^{\sfrac{2}{5}}\right)^{\sfrac{3}{4}}$
\vspace{1.25in}
\item $\left(x^{-\sfrac{3}{5}}y^{\sfrac{1}{4}}\right)^{\sfrac{1}{3}}$
\vspace{1.25in}
\end{minipage}%
\end{enumerate}
\end{example}

\begin{example}
Use the properties of rational exponents to simplify the following radical statements:
\begin{enumerate}
\begin{minipage}{.5\textwidth}
\item $\sqrt[6]{x^3}$
\vspace{1.25in}
\item $\sqrt[3]{8x^{12}}$
\vspace{1.25in}
\end{minipage}%
\begin{minipage}{.5\textwidth}
\item $\dfrac{\sqrt{x}}{\sqrt[3]{x}}$
\vspace{1.25in}
\item $\sqrt{\sqrt[3]{x}}$
\vspace{1.25in}
\end{minipage}%
\end{enumerate}
\end{example}

\end{document}