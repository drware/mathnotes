\documentclass[addpoints,12pt]{exam}
\usepackage{amsmath}
\usepackage{amsthm}
\usepackage{amsfonts}
\usepackage{systeme}
\usepackage{graphicx}
\usepackage{caption}
\usepackage{xfrac}
\usepackage{physics}
\usepackage{microtype}
\usepackage{eulervm}
%\usepackage[framemethod=tikz]{mdframed}
\usepackage{thmtools}
\usepackage{etoolbox}
%\usepackage{fouriernc}

\pagestyle{headandfoot}
\runningfootrule
\firstpageheadrule
\runningheadrule

\newcommand{\class}{Math 0098}
\newcommand{\sem}{2201}
\newcommand{\due}{}
\newcommand{\sect}{7.6}
\newcommand{\topic}{Solving Rational Equations}

\firstpageheader{\class}{\sect - \topic}{}
\runningheader{\class}{\sect - \topic}{}
\firstpagefooter{\class}{}{Page \thepage\ of \numpages}
\runningfooter{\class}{}{Page \thepage\ of \numpages}

\newif\ifprintselected
\printselectedtrue
%\printselectedfalse

\newenvironment{select}
{\ifprintselected
	\printanswers
	\fi
}
{}

\theoremstyle{definition}
\newtheorem{theorem}{Theorem}
\newtheorem{example}{Example}[subsection]
%\newtheorem{definition}{Definition}
\newtheorem{definition}{Definition}[subsection]
%\newmdtheoremenv{example}{Example}[subsection]
\AtBeginEnvironment{defn}{\begin{minipage}{\textwidth}}
\AtEndEnvironment{defn}{\end{minipage}}
%\AtBeginEnvironment{example}{\begin{minipage}{\textwidth}}
%\AtEndEnvironment{example}{\end{minipage}}
\newcommand{\iu}{{i\mkern1mu}}

\setlength{\gridsize}{5mm}
\setlength{\gridlinewidth}{0.1pt}

\printanswers
\DeclareMathSizes{12}{12}{12}{12}

\begin{document}
\setcounter{section}{7}
\setcounter{subsection}{5}

\subsection{Solving Rational Equations}

We solve proportions by cross multiplying, as below.
\begin{example}
Solve: $\dfrac{x}{10} = \dfrac{3}{5}$
\vspace{1.5in}

\end{example}

What if we have multiple terms on each side? We can remove them by multiplying both sides by the LCD which removes the denominator for us - removing the fact that they are rational expressions.

\begin{example}
Solve: $\dfrac{x}{6} = \dfrac{1}{6} + \dfrac{x}{8}$
\vspace{2.5in}
\end{example}

If we have an $x$ or any other variable in the denominator, we need to consider where our expressions are undefined (Section 7.1). If our answer makes the expression undefined, we call it an \emph{extraneous solution}. An extraneous solution is a solution that we can find algebraically, but does not actually solve the problem. We are not guaranteed to have extraneous solutions, but we do need to check for them.

\newpage

\begin{example}
Solve: $\dfrac{5}{2x} = \dfrac{17}{18} - \dfrac{1}{3x}$
\vspace{3.5in}
\end{example}

\begin{example}
Solve: $x + \dfrac{1}{x} = \dfrac{5}{2}$
\end{example}

\newpage

\begin{example}
Solve: $\dfrac{3x}{x^2-9} + \dfrac{1}{x-3} = \dfrac{3}{x+3}$
\vspace{3.5in}
\end{example}

\begin{example}
Solve: $\dfrac{x}{x-3} = \dfrac{3}{x-3} + 9$
\end{example}

\newpage

\subsubsection*{Solving Literal Equations}
Remember that a literal equation is an equation containing all or mostly variables. We can solve these for a specific value by rearranging them.

\begin{example}
Solve for $x$: $a = \dfrac{b}{x+c}$
\vspace{2.75in}
\end{example}

\begin{example}
Solve for $x$: $\dfrac{1}{x} + \dfrac{1}{y} = \dfrac{1}{z}$
\end{example}


\end{document}