\documentclass[addpoints,12pt]{exam}
\usepackage{amsmath}
\usepackage{amsthm}
\usepackage{amsfonts}
\usepackage{systeme}
\usepackage{graphicx}
\usepackage{caption}
\usepackage{xfrac}
\usepackage{physics}
\usepackage{microtype}
\usepackage{eulervm}
%\usepackage[framemethod=tikz]{mdframed}
\usepackage{thmtools}
\usepackage{etoolbox}
%\usepackage{fouriernc}
\usepackage{mdframed}
\usepackage[overload]{empheq}
\usepackage{adjustbox}
\usepackage{enumitem}
\usepackage[explicit]{titlesec}

\pagestyle{headandfoot}
\runningfootrule
\firstpageheadrule
\runningheadrule

\newcommand{\class}{Math 0098}
\newcommand{\sem}{2211}
\newcommand{\due}{}
\newcommand{\sect}{7.3}
\newcommand{\topic}{Adding \& Subtracting with Like Denominators}

\firstpageheader{\class}{\sect - \topic}{}
\runningheader{\class}{\sect - \topic}{}
\firstpagefooter{\class}{}{Page \thepage\ of \numpages}
\runningfooter{\class}{}{Page \thepage\ of \numpages}

\newif\ifprintselected
\printselectedtrue
%\printselectedfalse

\newenvironment{select}
{\ifprintselected
	\printanswers
	\fi
}
{}

\theoremstyle{definition}
\newtheorem{theorem}{Theorem}
\newtheorem{example}{Example}[subsection]
%\newtheorem{definition}{Definition}
\newtheorem{definition}{Definition}[subsection]
%\newmdtheoremenv{example}{Example}[subsection]
\AtBeginEnvironment{defn}{\begin{minipage}{\textwidth}}
\AtEndEnvironment{defn}{\end{minipage}}
%\AtBeginEnvironment{example}{\begin{minipage}{\textwidth}}
%\AtEndEnvironment{example}{\end{minipage}}
\newcommand{\iu}{{i\mkern1mu}}

\setlength{\gridsize}{5mm}
\setlength{\gridlinewidth}{0.1pt}

\printanswers
\DeclareMathSizes{12}{12}{12}{12}

\titleformat{\subsubsection}
   {\large\bfseries}% format
   {}% label
   {0pt}% sep
   {\titlerule \vspace{.1in} #1}% before code
      [{\titlerule[0.4pt]\vspace{.1in}}]% after code
\titlespacing{\subsubsection}
   {0pt}% left
   {0pt}% before sep
   {\baselineskip}% after sep


\begin{document}
\setcounter{section}{7}
\setcounter{subsection}{2}

\subsection{Adding \& Subtracting with Like Denominators}

\noindent Addition and subtraction of rational expressions works exactly the same as with fractions.
\vspace{.25in}
\noindent Let $\dfrac{P}{R}$ and $\dfrac{Q}{R}$ be rational expressions for some polynomials $P(x), Q(x)$ and $R(x)\neq 0$. Then, \[\dfrac{P}{R} \pm \dfrac{Q}{R} = \dfrac{P\pm Q}{R}\]
\vspace{.25in}
\noindent Since rational expressions are essentially fractions, treat them the same way. Combine the numerators with either addition or subtraction into one numerator and keep the like denominator. Then simplify the numerator by combining like terms. If possible, factor both the numerator and denominator to reduce the rational expressions to a fully simplified form.

\begin{example}
Find the following: \[\dfrac{3x-2}{5} + \dfrac{2x+12}{5}\]
\vspace{2in}
\end{example}

\newpage

\begin{example}
Find the following: \[\dfrac{x^2}{x^2-25}+\dfrac{25-10x}{x^2-25}\]
\vspace{3.25in}
\end{example}

\noindent If we are instead subtracting, distribute a $-1$ to the second numerator and then treat as addition.
\vspace{.15in}
\begin{example}
Find the following: \[\dfrac{4x+5}{x+7} - \dfrac{x}{x+7}\]
\end{example}

\newpage

\begin{example}
Find the following: \[\dfrac{3x^2 + 4x}{x-1} - \dfrac{11x-4}{x-1}\]
\vspace{3in}
\end{example}

\begin{example}
Find the following: \[\dfrac{y^2+3y-6}{y^2 - 5y + 4} - \dfrac{4y-4-2y^2}{y^2-5y+4}\]
\end{example}

\newpage

\begin{example}
Find the following: \[\dfrac{x^2}{x-7} + \dfrac{4x+21}{7-x}\]
\vspace{3in}
\end{example}

\begin{example}
Find the following: \[\dfrac{7x-x^2}{x^2-2x-9}-\dfrac{5x-3x^2}{9+2x-x^2}\]
\end{example}

\end{document}