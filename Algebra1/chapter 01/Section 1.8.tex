\documentclass[addpoints,12pt]{exam}
\usepackage{amsmath}
\usepackage{amsthm}
\usepackage{amsfonts}
\usepackage{systeme}
\usepackage{graphicx}
\usepackage{caption}
\usepackage{xfrac}
\usepackage{physics}
\usepackage{microtype}
\usepackage{eulervm}
%\usepackage[framemethod=tikz]{mdframed}
\usepackage{thmtools}
\usepackage{etoolbox}
%\usepackage{fouriernc}
\usepackage{mdframed}
\usepackage[overload]{empheq}
\usepackage{adjustbox}
\usepackage{enumitem}
\usepackage[explicit]{titlesec}

\pagestyle{headandfoot}
\runningfootrule
\firstpageheadrule
\runningheadrule

\newcommand{\class}{Math 0097}
\newcommand{\sem}{2211}
\newcommand{\due}{}
\newcommand{\sect}{1.8}
\newcommand{\topic}{Exponents \& Order of Operations}

\firstpageheader{\class}{\sect - \topic}{}
\runningheader{\class}{\sect - \topic}{}
\firstpagefooter{\class}{}{Page \thepage\ of \numpages}
\runningfooter{\class}{}{Page \thepage\ of \numpages}

\newif\ifprintselected
\printselectedtrue
%\printselectedfalse

\newenvironment{select}
{\ifprintselected
	\printanswers
	\fi
}
{}

\theoremstyle{definition}
\newtheorem{theorem}{Theorem}
\newtheorem{example}{Example}[subsection]
%\newtheorem{definition}{Definition}
\newmdtheoremenv{definition}{Definition}[subsection]
%\newmdtheoremenv{example}{Example}[subsection]
\AtBeginEnvironment{defn}{\begin{minipage}{\textwidth}}
\AtEndEnvironment{defn}{\end{minipage}}
%\AtBeginEnvironment{example}{\begin{minipage}{\textwidth}}
%\AtEndEnvironment{example}{\end{minipage}}
\newcommand{\iu}{{i\mkern1mu}}

\setlength{\gridsize}{5mm}
\setlength{\gridlinewidth}{0.1pt}

\printanswers
\DeclareMathSizes{12}{12}{12}{12}

\titleformat{\subsubsection}
   {\large\bfseries}% format
   {}% label
   {0pt}% sep
   {\titlerule \vspace{.1in} #1}% before code
      [{\titlerule[0.4pt]\vspace{.1in}}]% after code
\titlespacing{\subsubsection}
   {0pt}% left
   {0pt}% before sep
   {\baselineskip}% after sep


\begin{document}
\setcounter{section}{1}
\setcounter{subsection}{7}

\subsection{Exponents \& Order of Operations}

\vspace{.25in}

\begin{definition}[Exponentiation]
shorthand notation for repeated multiplication\\ $b^n = b\cdot b\cdot b\cdots$
\end{definition}

\vspace{.15in}

\begin{example}
Evaluate each of the follow exponents.
\begin{enumerate}
\item $4^2$
\item $6^3$
\item $(-4)^3$
\item $(-1)^4$
\item $-1^4$
\end{enumerate}
\end{example}

\vspace{.15in}

\subsubsection*{Simplifying Algebraic Expressions}
\begin{itemize}
\item variable terms are like terms \emph{if and only if} the exponent on the variables match
\item the same rules as always apply
\end{itemize}
\vspace{.15in}

\begin{minipage}{.33\textwidth}
\begin{example}\mbox{}\\
Simplify: $16x^2 + 5x^2$
\end{example}
\end{minipage}%
\begin{minipage}{.33\textwidth}
\begin{example}\mbox{}\\
Simplify: $7x^3 + x^3$
\end{example}
\end{minipage}%
\begin{minipage}{.33\textwidth}
\begin{example}\mbox{}\\
Simplify: $10x^2 + 8x^3$
\end{example}
\end{minipage}%

\newpage

\subsubsection*{PEMDAS - Order of Operations}

\begin{minipage}{.65\textwidth}
\begin{enumerate}
\item do all operations within grouping symbols: (), [], \{\}
\item evaluate all exponents
\item multiply and divide in the order they appear from left to right
\item add and subtract in the order they appear from left to right
\end{enumerate}
\end{minipage}%
\begin{minipage}{.1\textwidth}
\;
\end{minipage}%
\begin{minipage}{.25\textwidth}
\begin{mdframed}
\textbf{P} - Parenthesis\\
\textbf{E} - Exponents\\
\textbf{M} - Multiplication\\
\textbf{D} - Division\\
\textbf{A} - Addition\\
\textbf{S} - Subtraction
\end{mdframed}
\end{minipage}%

\vspace{.5in}
\noindent
\begin{minipage}{.5\textwidth}
\begin{example}\mbox{}\\
Simplify: $20 + 4\cdot 3 - 17$
\vspace{1.75in}
\end{example}
\end{minipage}%
\begin{minipage}{.5\textwidth}
\begin{example}\mbox{}\\
Simplify: $6^2 - 24 \divisionsymbol 2^3\cdot 3 - 1$
\vspace{1.75in}
\end{example}
\end{minipage}%
\\
\begin{minipage}{.5\textwidth}
\begin{example}\mbox{}\\
Simplify: $(3\cdot 2)^2$
\vspace{1.25in}
\end{example}
\end{minipage}%
\begin{minipage}{.5\textwidth}
\begin{example}\mbox{}\\
Simplify: $3\cdot 2^2$
\vspace{1.25in}
\end{example}
\end{minipage}%
\\

\begin{example}
Simplify fully: \[4[3(6-11)+5]\]
\end{example}

\newpage

\begin{example}
Simplify fully: \[\left(-\dfrac{1}{2}\right)^2 - \left(\dfrac{7}{10}-\dfrac{8}{15}\right)^2(-18)\]
\vspace{3in}
\end{example}

\begin{example}
Simplify fully: \[25\divisionsymbol 5 + 3[4+2(7-9)^3]\]
\end{example}

\newpage

\begin{example}
Simplify fully: \[\dfrac{2(3-12)+6\cdot 4}{2^4+1}\]
\vspace{2.5in}
\end{example}

\begin{example}
Evaluate $-x^2-7x$ for $x = -2$.
\vspace{1.5in}
\end{example}

\begin{example}
Simplify fully: \[14x^2+5 - [7(x^2-2)+4]\]
\end{example}
\end{document}