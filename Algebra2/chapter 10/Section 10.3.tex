\documentclass[addpoints,12pt]{exam}
\usepackage{amsmath}
\usepackage{amsthm}
\usepackage{amsfonts}
\usepackage{systeme}
\usepackage{graphicx}
\usepackage{caption}
\usepackage{xfrac}
\usepackage{physics}
\usepackage{microtype}
\usepackage{eulervm}
%\usepackage[framemethod=tikz]{mdframed}
\usepackage{thmtools}
\usepackage{etoolbox}
%\usepackage{fouriernc}
\usepackage{mdframed}
\usepackage[overload]{empheq}
\usepackage{adjustbox}
\usepackage{enumitem}

\pagestyle{headandfoot}
\runningfootrule
\firstpageheadrule
\runningheadrule

\newcommand{\class}{Math 0098}
\newcommand{\sem}{2201}
\newcommand{\due}{}
\newcommand{\sect}{10.3}
\newcommand{\topic}{Multiplying and Simplifying Radical Expressions}

\firstpageheader{\class}{\sect - \topic}{}
\runningheader{\class}{\sect - \topic}{}
\firstpagefooter{\class}{}{Page \thepage\ of \numpages}
\runningfooter{\class}{}{Page \thepage\ of \numpages}

\newif\ifprintselected
\printselectedtrue
%\printselectedfalse

\newenvironment{select}
{\ifprintselected
	\printanswers
	\fi
}
{}

\theoremstyle{definition}
\newtheorem{theorem}{Theorem}
\newtheorem{example}{Example}[subsection]
%\newtheorem{definition}{Definition}
\newtheorem{definition}{Definition}[subsection]
%\newmdtheoremenv{example}{Example}[subsection]
\AtBeginEnvironment{defn}{\begin{minipage}{\textwidth}}
\AtEndEnvironment{defn}{\end{minipage}}
%\AtBeginEnvironment{example}{\begin{minipage}{\textwidth}}
%\AtEndEnvironment{example}{\end{minipage}}
\newcommand{\iu}{{i\mkern1mu}}

\setlength{\gridsize}{5mm}
\setlength{\gridlinewidth}{0.1pt}

\printanswers
\DeclareMathSizes{12}{12}{12}{12}

\begin{document}
\setcounter{section}{10}
\setcounter{subsection}{2}

\subsection{Multiplying and Simplifying Radical Expressions}

\begin{mdframed}
\textbf{Product Rule for Radicals}

If $\sqrt[n]{a}$ and $\sqrt[n]{b}$ are real numbers, then we have \[\sqrt[n]{a}\cdot\sqrt[n]{b} = \sqrt[n]{ab}\] which follows from the rule for multiplying like-bases with exponents.
\end{mdframed}

\vspace{.25in}

\begin{example}
Find each of the following products:
\begin{enumerate}
\begin{minipage}{.5\textwidth}
\item $\sqrt{5}\sqrt{11}$
\vspace{1.5in}
\item $\sqrt[3]{6}\sqrt[3]{10}$
\vspace{1.5in}
\end{minipage}%
\begin{minipage}{.5\textwidth}
\item $\sqrt{x+4}\sqrt{x-4}$
\vspace{1.5in}
\item $\sqrt[7]{2x}\sqrt[7]{6x^3}$
\vspace{1.5in}
\end{minipage}%
\end{enumerate}
\end{example}


\newpage

\begin{mdframed}
\textbf{Simplifying Radical Expressions by Factoring}

We say that a radical expression (with index $n$) is simplified if the radicand has no factors that are perfect $n$ powers. Use the following method to simplify:
\begin{enumerate}
\item Write the radicand as the product of two factors where one of the factors has an exponent that is a multiple of $n$
\item Use the product rule to take the $n^{th}$ root of each factor
\item Find the $n^{th}$ root of the perfect $n^{th}$ power.
\end{enumerate}
\end{mdframed}

\vspace{.25in}

\begin{example}
Simplify by factoring:
\begin{enumerate}
\begin{minipage}{.5\textwidth}
\item $\sqrt{80}$
\vspace{1.5in}
\item $\sqrt[4]{32}$
\vspace{1.5in}
\end{minipage}%
\begin{minipage}{.5\textwidth}
\item $\sqrt[3]{40}$
\vspace{1.5in}
\item $\sqrt{200x^2y}$
\vspace{1.5in}
\end{minipage}%
\end{enumerate}
\end{example}

\vspace{.25in}

\begin{example}
Simplify the following radical function: \[f(x) = \sqrt{3x^2-12x+12}\]
\vspace{1.5in}
\end{example}

\begin{example}
Simplify the following expression: \[\sqrt{x^9y^{11}z^3}\]
\vspace{1.5in}
\end{example}

\begin{example}
Simplify the following expression: \[\sqrt[3]{40x^{10}y^{14}}\]
\vspace{1.5in}
\end{example}

\begin{example}
Simplify the following expression: \[\sqrt[5]{32x^{12}y^2z^8}\]
\vspace{1.5in}
\end{example}

\newpage

\begin{example}
Multiply then simplify: \[\sqrt{6}\cdot\sqrt{2}\]
\vspace{1.25in}
\end{example}

\begin{example}
Multiply then simplify: \[10\sqrt[3]{16}\cdot 5\sqrt[3]{2}\]
\vspace{1.25in}
\end{example}

\begin{example}
Multiply then simplify: \[\sqrt[4]{4x^2y}\cdot\sqrt[4]{8x^6y^3}\]
\vspace{1.25in}
\end{example}


\end{document}