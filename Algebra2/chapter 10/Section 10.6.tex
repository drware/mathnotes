\documentclass[addpoints,12pt]{exam}
\usepackage{amsmath}
\usepackage{amsthm}
\usepackage{amsfonts}
\usepackage{systeme}
\usepackage{graphicx}
\usepackage{caption}
\usepackage{xfrac}
\usepackage{physics}
\usepackage{microtype}
\usepackage{eulervm}
%\usepackage[framemethod=tikz]{mdframed}
\usepackage{thmtools}
\usepackage{etoolbox}
%\usepackage{fouriernc}
\usepackage{mdframed}
\usepackage[overload]{empheq}
\usepackage{adjustbox}
\usepackage{enumitem}
\usepackage[explicit]{titlesec}

\pagestyle{headandfoot}
\runningfootrule
\firstpageheadrule
\runningheadrule

\newcommand{\class}{Math 0098}
\newcommand{\sem}{2201}
\newcommand{\due}{}
\newcommand{\sect}{10.6}
\newcommand{\topic}{Radical Equations}

\firstpageheader{\class}{\sect - \topic}{}
\runningheader{\class}{\sect - \topic}{}
\firstpagefooter{\class}{}{Page \thepage\ of \numpages}
\runningfooter{\class}{}{Page \thepage\ of \numpages}

\newif\ifprintselected
\printselectedtrue
%\printselectedfalse

\newenvironment{select}
{\ifprintselected
	\printanswers
	\fi
}
{}

\theoremstyle{definition}
\newtheorem{theorem}{Theorem}
\newtheorem{example}{Example}[subsection]
%\newtheorem{definition}{Definition}
\newtheorem{definition}{Definition}[subsection]
%\newmdtheoremenv{example}{Example}[subsection]
\AtBeginEnvironment{defn}{\begin{minipage}{\textwidth}}
\AtEndEnvironment{defn}{\end{minipage}}
%\AtBeginEnvironment{example}{\begin{minipage}{\textwidth}}
%\AtEndEnvironment{example}{\end{minipage}}
\newcommand{\iu}{{i\mkern1mu}}

\setlength{\gridsize}{5mm}
\setlength{\gridlinewidth}{0.1pt}

\printanswers
\DeclareMathSizes{12}{12}{12}{12}

\titleformat{\subsubsection}
   {\large\bfseries}% format
   {}% label
   {0pt}% sep
   {\titlerule \vspace{.1in} #1}% before code
      [{\titlerule[0.4pt]\vspace{.1in}}]% after code
\titlespacing{\subsubsection}
   {0pt}% left
   {0pt}% before sep
   {\baselineskip}% after sep


\begin{document}
\setcounter{section}{10}
\setcounter{subsection}{5}

\subsection{Radical Equations}

\subsubsection*{Solving Radical Equations}

Solving a \emph{radical equation} involves removing the radical from the equation and turning the equation into a form that we can otherwise solve. When doing so, however, we need to be careful of \emph{extraneous solutions} -- solutions that we can arrive at alegebraically, but are not actually solutions of the original equation. These typically occur when there is a contradiction of some sort during the process.

\vspace{.2in}

\begin{mdframed}
\textbf{Method}
\begin{enumerate}
\item Isolate the radical
\item Raise both sides of the equation to the $n^{th}$ exponent (removing the radical)
\item Solve the new equation using an appropriate technique
\item Verify possible solutions
\end{enumerate}
\end{mdframed}

\vspace{.25in}

\begin{example}
Solve and verify: \[\sqrt{3x+4}=8\]
\vspace{1.5in}
\end{example}

\begin{example}
Solve and verify: \[\sqrt{x-1}+7=2\]
\vspace{1.5in}
\end{example}

\begin{example}
Solve and verify: \[\sqrt{6x+7}-x=2\]
\vspace{2in}
\end{example}

\begin{example}
Solve and verify: \[\sqrt{x+5}-\sqrt{x-3}=2\]
\vspace{2.5in}
\end{example}

\newpage
\begin{example}
Solve and verify: \[(2x-3)^{\sfrac{1}{3}}+3=0\]
\vspace{2in}
\end{example}

\begin{example}
Solve and verify: \[(9x+2)^{\sfrac{1}{4}}-(5x+18)^{\sfrac{1}{4}}=0\]
\vspace{2in}
\end{example}

\end{document}