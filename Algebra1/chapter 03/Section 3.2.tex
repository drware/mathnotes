\documentclass[addpoints,12pt]{exam}
\usepackage{amsmath}
\usepackage{amsthm}
\usepackage{amsfonts}
\usepackage{systeme}
\usepackage{graphicx}
\usepackage{caption}
\usepackage{xfrac}
\usepackage{physics}
\usepackage{microtype}
\usepackage{eulervm}
%\usepackage[framemethod=tikz]{mdframed}
\usepackage{thmtools}
\usepackage{etoolbox}
%\usepackage{fouriernc}
\usepackage{mdframed}
\usepackage[overload]{empheq}
\usepackage{adjustbox}
\usepackage{enumitem}
\usepackage[explicit]{titlesec}
% adds in \varnothing for empty set
\usepackage{amssymb}
% adds in formated SI units
%\usepackage{siunitx}

\pagestyle{headandfoot}
\runningfootrule
\firstpageheadrule
\runningheadrule

\newcommand{\class}{Math 0097}
\newcommand{\sem}{2211}
\newcommand{\due}{}
\newcommand{\sect}{3.2}
\newcommand{\topic}{Graphing with Intercepts}

\firstpageheader{\class}{\sect - \topic}{}
\runningheader{\class}{\sect - \topic}{}
\firstpagefooter{\class}{}{Page \thepage\ of \numpages}
\runningfooter{\class}{}{Page \thepage\ of \numpages}

\newif\ifprintselected
\printselectedtrue
%\printselectedfalse

\newenvironment{select}
{\ifprintselected
	\printanswers
	\fi
}
{}

\theoremstyle{definition}
\newtheorem{theorem}{Theorem}
%\newtheorem{example}{Example}[subsection]
%\newtheorem{definition}{Definition}
%\newmdtheoremenv{definition}{Definition}[subsection]
%\newmdtheoremenv{example}{Example}[subsection]
\AtBeginEnvironment{defn}{\begin{minipage}{\textwidth}}
\AtEndEnvironment{defn}{\end{minipage}}
%\AtBeginEnvironment{example}{\begin{minipage}{\textwidth}}
%\AtEndEnvironment{example}{\end{minipage}}
\newcommand{\iu}{{i\mkern1mu}}

\setlength{\gridsize}{5mm}
\setlength{\gridlinewidth}{0.1pt}

\printanswers
\DeclareMathSizes{12}{12}{12}{12}

%%%%%%%%%%%%%%%%%%%%%%%%
% Create bars around subsubsection
%%%%%%%%%%%%%%%%%%%%%%%%

\titleformat{\subsubsection}
   {\large\bfseries}% format
   {}% label
   {0pt}% sep
   {\titlerule \vspace{.1in} #1}% before code
      [{\titlerule[0.4pt]\vspace{.1in}}]% after code
\titlespacing{\subsubsection}
   {0pt}% left
   {0pt}% before sep
   {\baselineskip}% after sep
   
%%%%%%%%%%%%%%%%%%%%%%%
% Create line break after definition label
%%%%%%%%%%%%%%%%%%%%%%%   
\newtheoremstyle{break}
  {\topsep}{\topsep}%
  {}{}%\itshape
  {\bfseries}{}%
  {\newline}{}%
\theoremstyle{break}
\newmdtheoremenv{definition}{Definition}[subsection]
\theoremstyle{break}
\newtheorem{example}{Example}[subsection]

%%%%%%%%%%%%%%%%%%%%%%
% start document
% set section, subsection (use n-1 for sub)
%%%%%%%%%%%%%%%%%%%%%%


\begin{document}
\setcounter{section}{3}
\setcounter{subsection}{1}

\subsection{Graphing with Intercepts}

\vspace{.15in}

\begin{definition}[$x$-intercept]
\;
\begin{itemize}
\item the point at which the graph of the equation crosses the $x$-axis
\item found by setting $y=0$ and solving for $x$
\item written as the point $(x,0)$
\end{itemize}
\end{definition}
\vspace{.15in}

\begin{definition}[$y$-intercept]
\;
\begin{itemize}
\item the point at which the graph of the equation crosses the $y$-axis
\item represents the initial or starting amount in word problems
\item found by setting $x=0$ and solving for $y$
\item written as the point $(0,y)$
\end{itemize}
\end{definition}
\vspace{.15in}

\begin{example}
Find both intercepts of the equation $y=6x-2$.
\end{example}

\newpage

\begin{definition}[Standard Form of a Linear Equation]
\;
\begin{itemize}
\item written as $Ax + By = C$ where $A$, $B$, and $C$ are integers and $A>0$
\item exponents on $x$,$y$ must be 1
\end{itemize}
\end{definition}

\vspace{.15in}

\begin{example}
Find both intercepts of the standard form equation below.
\[4x - 3y = 12\]
\vspace{2.5in}
\end{example}

\begin{example}
Rewrite the following equation into standard form.
\[y = \dfrac{3}{4}x + \dfrac{7}{2}\]
\end{example}
\newpage

\subsubsection*{Using Intercepts to Graph Equations}
\begin{enumerate}
\item find and plot the $x$-intercept
\item find and plot the $y$-intercept
\item connect the dots
\end{enumerate}

\vspace{.15in}

\begin{example}
Graph $2x+3y=6$ using the $x$ and $y$-intercepts.
\vspace{2.75in}
\end{example}

\begin{example}
Graph $x+3y=0$.
\end{example}

\newpage

\subsubsection*{Special Cases of Lines}

\textbf{Horizontal Line}
\begin{itemize}
\item written as $y=c$ where $c$ is any real number
\item any value we choose as $x$ gives the same $y$ value
\end{itemize}
\vspace{.15in}
\begin{example}
Graph $y=2$.
\vspace{2in}
\end{example}

\textbf{Vertical Line}
\begin{itemize}
\item written as $x=c$ where $c$ is any real number
\item This $x$ value works for \emph{every} $y$ value.
\end{itemize}
\vspace{.15in}

\begin{example}
Graph $x=3$.
\end{example}
\end{document}