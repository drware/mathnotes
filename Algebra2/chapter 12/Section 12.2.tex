\documentclass[addpoints,12pt]{exam}
\usepackage{amsmath}
\usepackage{amsthm}
\usepackage{amsfonts}
\usepackage{systeme}
\usepackage{graphicx}
\usepackage{caption}
\usepackage{xfrac}
\usepackage{physics}
\usepackage{microtype}
\usepackage{eulervm}
%\usepackage[framemethod=tikz]{mdframed}
\usepackage{thmtools}
\usepackage{etoolbox}
%\usepackage{fouriernc}
\usepackage{mdframed}
\usepackage[overload]{empheq}
\usepackage{adjustbox}
\usepackage{enumitem}
\usepackage[explicit]{titlesec}
% adds in \varnothing for empty set
\usepackage{amssymb}
% adds in formated SI units
%\usepackage{siunitx}
%\usepackage{pgfplots}
\usepackage{multirow}
\usepackage{array}

\pagestyle{headandfoot}
\runningfootrule
\firstpageheadrule
\runningheadrule

\newcommand{\class}{Math 0098}
\newcommand{\sem}{2211}
\newcommand{\due}{}
\newcommand{\sect}{12.2}
\newcommand{\topic}{Logarithmic Functions}

\firstpageheader{\class}{\sect - \topic}{}
\runningheader{\class}{\sect - \topic}{}
\firstpagefooter{\class}{}{Page \thepage\ of \numpages}
\runningfooter{\class}{}{Page \thepage\ of \numpages}

\newif\ifprintselected
\printselectedtrue
%\printselectedfalse

\newenvironment{select}
{\ifprintselected
	\printanswers
	\fi
}
{}

\theoremstyle{definition}
\newtheorem{theorem}{Theorem}
%\newtheorem{example}{Example}[subsection]
%\newtheorem{definition}{Definition}
%\newmdtheoremenv{definition}{Definition}[subsection]
%\newmdtheoremenv{example}{Example}[subsection]
\AtBeginEnvironment{defn}{\begin{minipage}{\textwidth}}
\AtEndEnvironment{defn}{\end{minipage}}
%\AtBeginEnvironment{example}{\begin{minipage}{\textwidth}}
%\AtEndEnvironment{example}{\end{minipage}}
\newcommand{\iu}{{i\mkern1mu}}

\setlength{\gridsize}{5mm}
\setlength{\gridlinewidth}{0.1pt}

\printanswers
\DeclareMathSizes{12}{12}{12}{12}

%%%%%%%%%%%%%%%%%%%%%%%%
% Create bars around subsubsection
%%%%%%%%%%%%%%%%%%%%%%%%

\titleformat{\subsubsection}
   {\large\bfseries}% format
   {}% label
   {0pt}% sep
   {\titlerule \vspace{.1in} #1}% before code
      [{\titlerule[0.4pt]\vspace{.1in}}]% after code
\titlespacing{\subsubsection}
   {0pt}% left
   {0pt}% before sep
   {\baselineskip}% after sep
   
%%%%%%%%%%%%%%%%%%%%%%%
% Create line break after definition label
%%%%%%%%%%%%%%%%%%%%%%%   
\newtheoremstyle{break}
  {\topsep}{\topsep}%
  {}{}%\itshape
  {\bfseries}{}%
  {\newline}{}%
\theoremstyle{break}
\newmdtheoremenv{definition}{Definition}[subsection]
\theoremstyle{break}
\newtheorem{example}{Example}[subsection]

%%%%%%%%%%%%%%%%%%%%%%
% start document
% set section, subsection (use n-1 for sub)
%%%%%%%%%%%%%%%%%%%%%%


\begin{document}
\setcounter{section}{12}
\setcounter{subsection}{1}

\subsection{Logarithmic Functions}
\vspace{.15in}

\begin{definition}[Logarithm]\mbox{}
\vspace{-.15in}
For $x> 0,\; b>0,\; b\neq 1$,
\[ y = \log_b\left(x\right) \equiv b^y = x\]

\end{definition}
\vspace{.15in}

\subsubsection*{Circle Trick}
\vspace{.25in}
\[ y = \log_b\left(x\right) \]
\vspace{.25in}

\begin{example}
Write each of the following in exponential form.
\begin{enumerate}
\item $3 = \log_7{\left(x\right)}$
\vspace{.5in}
\item $2 = \log_b{\left(25\right)}$
\vspace{.5in}
\item $y = \log_4{\left(26\right)}$
\end{enumerate}
\end{example}

\newpage

\begin{example}
Write each exponential as a logarithm.
\begin{enumerate}
\item $2^5 = x$
\vspace{.25in}
\item $b^3 = 27$
\vspace{.25in}
\item $e^y = 33$
\vspace{.25in}
\end{enumerate}
\end{example}

\begin{example}
Evaluate.
\begin{enumerate}
\item $\log_{10}{\left(100\right)} = $
\vspace{.25in}
\item $\log_{3}{\left(3\right)} = $
\vspace{.25in}
\item $\log_{36}{\left(6\right)} = $
\vspace{.25in}
\end{enumerate}
\end{example}

\subsubsection*{Basic Log Properties}
\begin{enumerate}
\item $\log_b{(b)} = 1$
\vspace{.1in}
\item $\log_b{(1)} = 0$
\vspace{.1in}
\item $\log_b{\left(b^x\right)} = x$
\vspace{.1in}
\item $b^{\log_b{(x)}} = x$
\end{enumerate}

\newpage

\begin{example}
Evaluate.
\begin{enumerate}
\item $\log_{7}{\left(7\right)} = $
\vspace{.25in}
\item $\log_{8}{\left(1\right)} = $
\vspace{.25in}
\item $\log_{7}{\left(7^8\right)} = $
\vspace{.25in}
\item $3^{\log_{3}{\left(15\right)}} = $
\vspace{.25in}
\end{enumerate}
\end{example}

\subsubsection*{Graphs of $\log_b{(x)}$}
\vspace{2in}
\begin{itemize}
\item Domain: $\mathbb{Z}^+ = (0,\infty)$
\item Range: $\mathbb{R} = (-\infty,\infty)$
\item Logarithmic functions always have two guaranteed points: $(1,0)$ and $(b,1)$.
\item The $y$-axis is a horizontal asymptote.
\item If $b>1$, it is an \emph{increasing} function.
\item If $0 < b < 1$, it is a \emph{decreasing} function.
\end{itemize}

\newpage

\begin{example}
Find the domain of each log function.
\begin{enumerate}
\item $h(x) = \log_4{(x-5)}$
\vspace{1.75in}
\item $g(x) = \log_6{(2x - 4)}$
\vspace{1.75in}
\item $f(x) = \log_3{(6-4x)} + 3$
\end{enumerate}
\end{example}

\newpage

\subsubsection*{Special Logs}
\begin{enumerate}
\begin{minipage}{.5\textwidth}
\item Common Log
\begin{itemize}
\item Base 10
\item $\log{(x)} = \log_{10}{(x)}$
\end{itemize}
\end{minipage}%
\begin{minipage}{.5\textwidth}
\item Natural Log
\begin{itemize}
\item Base $e$
\item $\ln{(x)} = \log_{e}{(x)}$
\end{itemize}
\end{minipage}%
\end{enumerate}
\vspace{.15in}

\begin{example}
Find each of the following values:
\begin{enumerate}
\item $\log{(1000)} = $
\vspace{.5in}
\item $\ln{(50} = $
\vspace{.5in}
\end{enumerate}
\end{example}


\end{document}