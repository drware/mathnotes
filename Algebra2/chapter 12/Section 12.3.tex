\documentclass[addpoints,12pt]{exam}
\usepackage{amsmath}
\usepackage{amsthm}
\usepackage{amsfonts}
\usepackage{systeme}
\usepackage{graphicx}
\usepackage{caption}
\usepackage{xfrac}
\usepackage{physics}
\usepackage{microtype}
\usepackage{eulervm}
%\usepackage[framemethod=tikz]{mdframed}
\usepackage{thmtools}
\usepackage{etoolbox}
%\usepackage{fouriernc}
\usepackage{mdframed}
\usepackage[overload]{empheq}
\usepackage{adjustbox}
\usepackage{enumitem}
\usepackage[explicit]{titlesec}
% adds in \varnothing for empty set
\usepackage{amssymb}
% adds in formated SI units
%\usepackage{siunitx}
%\usepackage{pgfplots}
\usepackage{multirow}
\usepackage{array}

\pagestyle{headandfoot}
\runningfootrule
\firstpageheadrule
\runningheadrule

\newcommand{\class}{Math 0098}
\newcommand{\sem}{2211}
\newcommand{\due}{}
\newcommand{\sect}{12.3}
\newcommand{\topic}{Properties of Logarithms}

\firstpageheader{\class}{\sect - \topic}{}
\runningheader{\class}{\sect - \topic}{}
\firstpagefooter{\class}{}{Page \thepage\ of \numpages}
\runningfooter{\class}{}{Page \thepage\ of \numpages}

\newif\ifprintselected
\printselectedtrue
%\printselectedfalse

\newenvironment{select}
{\ifprintselected
	\printanswers
	\fi
}
{}

\theoremstyle{definition}
\newtheorem{theorem}{Theorem}
%\newtheorem{example}{Example}[subsection]
%\newtheorem{definition}{Definition}
%\newmdtheoremenv{definition}{Definition}[subsection]
%\newmdtheoremenv{example}{Example}[subsection]
\AtBeginEnvironment{defn}{\begin{minipage}{\textwidth}}
\AtEndEnvironment{defn}{\end{minipage}}
%\AtBeginEnvironment{example}{\begin{minipage}{\textwidth}}
%\AtEndEnvironment{example}{\end{minipage}}
\newcommand{\iu}{{i\mkern1mu}}

\setlength{\gridsize}{5mm}
\setlength{\gridlinewidth}{0.1pt}

\printanswers
\DeclareMathSizes{12}{12}{12}{12}

%%%%%%%%%%%%%%%%%%%%%%%%
% Create bars around subsubsection
%%%%%%%%%%%%%%%%%%%%%%%%

\titleformat{\subsubsection}
   {\large\bfseries}% format
   {}% label
   {0pt}% sep
   {\titlerule \vspace{.1in} #1}% before code
      [{\titlerule[0.4pt]\vspace{.1in}}]% after code
\titlespacing{\subsubsection}
   {0pt}% left
   {0pt}% before sep
   {\baselineskip}% after sep
   
%%%%%%%%%%%%%%%%%%%%%%%
% Create line break after definition label
%%%%%%%%%%%%%%%%%%%%%%%   
\newtheoremstyle{break}
  {\topsep}{\topsep}%
  {}{}%\itshape
  {\bfseries}{}%
  {\newline}{}%
\theoremstyle{break}
\newmdtheoremenv{definition}{Definition}[subsection]
\theoremstyle{break}
\newtheorem{example}{Example}[subsection]

%%%%%%%%%%%%%%%%%%%%%%
% start document
% set section, subsection (use n-1 for sub)
%%%%%%%%%%%%%%%%%%%%%%


\begin{document}
\setcounter{section}{12}
\setcounter{subsection}{2}

\subsection{Properties of Logarithms}
\vspace{.15in}

\begin{enumerate}
\item $\log_b{\left(mn\right)} = \log_b{(m)} + \log_b{(n)}$
\vspace{.15in}
\item $\log_b{\left(\dfrac{m}{n}\right)} = \log_b{(m)} - \log_b{(n)}$
\vspace{.15in}
\item $\log_b{\left(m^p\right)} = p\log_b{(m)}$
\vspace{.15in}
\end{enumerate}

\begin{example}
Expand and rewrite. Simplify where appropriate.
\begin{enumerate}
\item $\log_8{\left(\dfrac{23}{x}\right)} = $
\vspace{.75in}
\item $\ln{\left(\dfrac{e^5}{11}\right)} = $
\vspace{.75in}
\item $\ln{\sqrt[3]{x}} = $
\vspace{.75in}
\item $\log{[(x+2)(x+2)]} = $
\newpage

\item $\log_b{\left(x^4\sqrt[3]{y}\right)} = $
\vspace{1in}
\item $\log_5{\left(\dfrac{\sqrt{x}}{25y^{-3}}\right)} = $
\vspace{1in}
\end{enumerate}
\end{example}

\begin{example}
Write each as a single log.
\begin{enumerate}
\item $\log{(25)} + \log{(4)} = $
\vspace{1in}
\item $\log{(7x^2 - 6x)} - \log{(x)} = $
\vspace{1in}
\item $2\ln{(x)} + \dfrac{1}{3}\ln{(x+5)} = $
\newpage
\item $2\log{(x-3)} - \log{x} = $
\vspace{1in}
\item $\dfrac{1}{4}\log_b{(x)} - 2\log_b{(5)} - 10\log_b{(y)} = $
\end{enumerate}
\end{example}

\end{document}