\documentclass[addpoints,12pt]{exam}
\usepackage{amsmath}
\usepackage{amsthm}
\usepackage{amsfonts}
\usepackage{systeme}
\usepackage{graphicx}
\usepackage{caption}
\usepackage{xfrac}
\usepackage{physics}
\usepackage{microtype}
\usepackage{eulervm}
%\usepackage[framemethod=tikz]{mdframed}
\usepackage{thmtools}
\usepackage{etoolbox}
%\usepackage{fouriernc}
\usepackage{mdframed}
\usepackage[overload]{empheq}
\usepackage{adjustbox}
\usepackage{enumitem}
\usepackage[explicit]{titlesec}

\pagestyle{headandfoot}
\runningfootrule
\firstpageheadrule
\runningheadrule

\newcommand{\class}{Math 0097}
\newcommand{\sem}{2211}
\newcommand{\due}{}
\newcommand{\sect}{1.4}
\newcommand{\topic}{Rules of Algebra}

\firstpageheader{\class}{\sect - \topic}{}
\runningheader{\class}{\sect - \topic}{}
\firstpagefooter{\class}{}{Page \thepage\ of \numpages}
\runningfooter{\class}{}{Page \thepage\ of \numpages}

\newif\ifprintselected
\printselectedtrue
%\printselectedfalse

\newenvironment{select}
{\ifprintselected
	\printanswers
	\fi
}
{}

\theoremstyle{definition}
\newtheorem{theorem}{Theorem}
\newtheorem{example}{Example}[subsection]
%\newtheorem{definition}{Definition}
\newtheorem{definition}{Definition}[subsection]
%\newmdtheoremenv{example}{Example}[subsection]
\AtBeginEnvironment{defn}{\begin{minipage}{\textwidth}}
\AtEndEnvironment{defn}{\end{minipage}}
%\AtBeginEnvironment{example}{\begin{minipage}{\textwidth}}
%\AtEndEnvironment{example}{\end{minipage}}
\newcommand{\iu}{{i\mkern1mu}}

\setlength{\gridsize}{5mm}
\setlength{\gridlinewidth}{0.1pt}

\printanswers
\DeclareMathSizes{12}{12}{12}{12}

\titleformat{\subsubsection}
   {\large\bfseries}% format
   {}% label
   {0pt}% sep
   {\titlerule \vspace{.1in} #1}% before code
      [{\titlerule[0.4pt]\vspace{.1in}}]% after code
\titlespacing{\subsubsection}
   {0pt}% left
   {0pt}% before sep
   {\baselineskip}% after sep


\begin{document}
\setcounter{section}{1}
\setcounter{subsection}{3}

\subsection{Rules of Algebra}
\subsubsection*{Vocabulary}
\begin{itemize}
\item \textbf{term}: parts of an algebraic expression separated by + or -
\item \textbf{coefficient}: the numerical part of a term
\item \textbf{constant term}: a term that has no variable
\item \textbf{like terms}: terms that have exactly the same variables and exponents
\end{itemize}
\vspace{.25in}

\begin{example}
Consider the following algebraic expression and answer each question.
\[6x + 2x + 11\]
\begin{enumerate}
\item how many terms are there?
\vspace{.1in}
\item what is the coefficient of the first term?
\vspace{.1in}
\item what is the constant term?
\vspace{.1in}
\item what like terms are there, if any?
\vspace{.1in}
\end{enumerate}
\end{example}

\noindent It is also important that you be able to identify \textbf{equivalent algebraic expressions}. When two or more expressions are evaluated for the same values of $x$, we call them \emph{equivalent} if they evaluate to the same value.

\newpage

\subsubsection*{Properties of Real Numbers and Algebraic Expressions}

\begin{itemize}
\item Commutative
\begin{itemize}
\item addition: $a + b = b + a$
\item multiplication: $ab = ba$
\end{itemize}
\item Associative
\begin{itemize}
\item addition: $(a + b) + c = a + (b + c)$
\item multiplication: $(ab)c = a(bc)$
\end{itemize}
\item Distributive
\begin{itemize}
\item $a(b+c) = ab + ac$
\end{itemize}
\end{itemize}

\begin{example}
Simplify fully: \[8 + (12 + x)\]
\vspace{.75in}
\end{example}

\begin{example}
Simplify fully: \[ 6(5x)\]
\vspace{.75in}
\end{example}

\begin{example}
Write an equivalent expression for \[8 + (x + 4)\]
\vspace{.75in}
\end{example}

\newpage

\begin{example}
Simplify fully: \[5(x+3)\]
\vspace{.75in}
\end{example}

\begin{example}
Simplify fully: \[6(4y+7)\]
\vspace{.75in}
\end{example}

\subsubsection*{Combining Like Terms}
\noindent The distribution property allows us to combine like terms and reduce the number of terms given in an expression.

\noindent How does this work? Consider the following where $ax$ and $bx$ are like terms.
\[ax + bx = x(a + b) = (a+b)x\]

\begin{example}
Combine like terms: \[7x + 3x\]
\vspace{.75in}
\end{example}

\begin{example}
Combine like terms: \[9a-4a\]
\vspace{.75in}
\end{example}

\subsubsection*{Grouping Multiple Like Terms}
\noindent The process remains the same when it comes to combining like terms even if we have multiple groups of like terms.

\begin{example}
Combine like terms: \[8x + 7 + 10x + 3\]
\vspace{1in}
\end{example}

\begin{example}
Combine like terms: \[9x + 6y - 5x + 2y\]
\vspace{1in}
\end{example}

\newpage

\subsubsection*{Simplifying Algebraic Expressions}

\noindent An algebraic expression is considered \textbf{simplified} when all parenthesis have been removed and all like terms have been combined.

\vspace{.1in}

\begin{mdframed}
\textbf{Process}
\begin{enumerate}
\item Use the distributive property to remove parentheses.
\item Rearrange terms and group like terms together.
\item Combine like terms.
\end{enumerate}
\end{mdframed}

\begin{example}
Simplify fully: \[7(2x+3) + 11x\]
\vspace{1.55in}
\end{example}

\begin{example}
Simplify fully: \[7(4x+3y) + 2(5x+y)\]
\end{example}
\end{document}