\documentclass[addpoints,12pt]{exam}
\usepackage{amsmath}
\usepackage{amsthm}
\usepackage{amsfonts}
\usepackage{systeme}
\usepackage{graphicx}
\usepackage{caption}
\usepackage{xfrac}
\usepackage{physics}
\usepackage{microtype}
\usepackage{eulervm}
%\usepackage[framemethod=tikz]{mdframed}
\usepackage{thmtools}
\usepackage{etoolbox}
%\usepackage{fouriernc}
\usepackage{mdframed}

\pagestyle{headandfoot}
\runningfootrule
\firstpageheadrule
\runningheadrule

\newcommand{\class}{Math 0098}
\newcommand{\sem}{2201}
\newcommand{\due}{}
\newcommand{\sect}{9.2}
\newcommand{\topic}{Compound Inequalities}

\firstpageheader{\class}{\sect - \topic}{}
\runningheader{\class}{\sect - \topic}{}
\firstpagefooter{\class}{}{Page \thepage\ of \numpages}
\runningfooter{\class}{}{Page \thepage\ of \numpages}

\newif\ifprintselected
\printselectedtrue
%\printselectedfalse

\newenvironment{select}
{\ifprintselected
	\printanswers
	\fi
}
{}

\theoremstyle{definition}
\newtheorem{theorem}{Theorem}
\newtheorem{example}{Example}[subsection]
%\newtheorem{definition}{Definition}
\newtheorem{definition}{Definition}[subsection]
%\newmdtheoremenv{example}{Example}[subsection]
\AtBeginEnvironment{defn}{\begin{minipage}{\textwidth}}
\AtEndEnvironment{defn}{\end{minipage}}
%\AtBeginEnvironment{example}{\begin{minipage}{\textwidth}}
%\AtEndEnvironment{example}{\end{minipage}}
\newcommand{\iu}{{i\mkern1mu}}

\setlength{\gridsize}{5mm}
\setlength{\gridlinewidth}{0.1pt}

\printanswers
\DeclareMathSizes{12}{12}{12}{12}

\begin{document}
\setcounter{section}{9}
\setcounter{subsection}{1}

\subsection{\topic}

\subsubsection*{Introductory Set Theory}

\begin{definition}[Set]
a set is a collection of \emph{distinct} objects; each object in the set is called an \emph{element}
\end{definition}

\vspace{.2in}
\begin{definition}[Intersection of Sets]
the intersection of sets $A$ and $B$ is given as $A\cap B$ and is the set of elements that are found in \emph{both} sets
\[A \cap B = \{x \mid x \in A \text{ and } x \in B\}\]

\end{definition}

\vspace{.2in}
\begin{definition}[Union of Sets]
the union of sets $A$ and $B$ is given as $A \cup B$ and is the set of elements that are found \emph{either} set
\[A \cup B = \{x \mid x \in A \text{ or } x \in B\}\]
\end{definition}

\vspace{.2in}
\begin{definition}[Set Substraction]
the subtraction of two sets $A$ and $B$ is given as $A\backslash B$ and represents what remains after all elements that occur in $B$ are removed from $A$
\[ A\backslash B = \{x \mid x \in A \text{ and } x \notin B\}\]
\end{definition}

\vspace{.2in}
\begin{definition}[Set Cardinality]
the cardinality (size) of a set is the number of distinct elements in the set and is given by $\abs{\abs{A}}$
\end{definition}
\vspace{.2in}
\begin{example}
Given $A = \{a,b,c,d,e,f\}$ and $B = \{b,d,f,h,j,l\}$, find each of the following:
\begin{enumerate}
\begin{minipage}{.5\textwidth}

\item $A \cap B = $
\vspace{.25in}
\item $A \cup B = $
\vspace{.25in}
\item $A\backslash B = $
\vspace{.25in}
\item $B\backslash A = $
\vspace{.25in}
\end{minipage}%
\begin{minipage}{.5\textwidth}
\item $\abs{\abs{A \cap B}} = $
\vspace{.25in}
\item $\abs{\abs{A \cup B }}= $
\vspace{.25in}
\item $\abs{\abs{A\backslash B}} = $
\vspace{.25in}
\item $\abs{\abs{B\backslash A}} = $
\vspace{.25in}

\end{minipage}%
\end{enumerate}
\end{example}

\newpage
\begin{example}
Given $A = \{1,2,3,\dots,10\}$ and $B = \{2,4,6,\dots,20\}$, find each of the following:
\begin{enumerate}
\begin{minipage}{.5\textwidth}

\item $A \cap B = $
\vspace{.35in}
\item $A \cup B = $
\vspace{.35in}
\item $A\backslash B = $
\vspace{.35in}
\item $B\backslash A = $
\vspace{.35in}
\end{minipage}%
\begin{minipage}{.5\textwidth}
\item $\abs{\abs{A \cap B}} = $
\vspace{.35in}
\item $\abs{\abs{A \cup B }}= $
\vspace{.35in}
\item $\abs{\abs{A\backslash B}} = $
\vspace{.35in}
\item $\abs{\abs{B\backslash A}} = $
\vspace{.35in}

\end{minipage}%
\end{enumerate}
\end{example}

\vspace{.2in}

\subsubsection*{Compound Inequalities with "And"}

A number is a solution of a compound inequality involving "and" if and only if it satisfies both of the given inequalities. In other words, the solution set is the \emph{intersection} of the solution to each individual inequality.

\begin{example}
Solve the compound inequality:
\[x + 2 < 5 \text{ and } 2x-4 <-2\]
\vspace{2in}
\end{example}

\newpage

\begin{example}
Solve the compound inequality:
\[ 4x - 5 > 7 \text{ and } 5x-2 < 3\]
\vspace{3.5in}
\end{example}

\begin{example}
Solve the compound inequality:
\[ 1 \le 2x + 3 < 11\]
\end{example}

\newpage

\subsubsection*{Compound Inequalities with "Or"}

A number is a solution of a compound inequality with the word "or" if it is a solution of either inequality. In other words, the solution set is the \emph{union} of the solution to each individual inequality.
\vspace{.2in}

\begin{example}
Solve the compound inequality:
\[3x -5 \le -2 \text{ or } 10 - 2x < 4\]
\vspace{2in}
\end{example}

\begin{example}
Solve the compound inequality:
\[2x + 5 \ge 3 \text{ or } 2x + 3 < 3\]
\end{example}

\end{document}