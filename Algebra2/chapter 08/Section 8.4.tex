\documentclass[addpoints,12pt]{exam}
\usepackage{amsmath}
\usepackage{amsthm}
\usepackage{amsfonts}
\usepackage{systeme}
\usepackage{graphicx}
\usepackage{caption}
\usepackage{xfrac}
\usepackage{physics}
\usepackage{microtype}
\usepackage{eulervm}
%\usepackage[framemethod=tikz]{mdframed}
\usepackage{thmtools}
\usepackage{etoolbox}
%\usepackage{fouriernc}
\usepackage{mdframed}

\pagestyle{headandfoot}
\runningfootrule
\firstpageheadrule
\runningheadrule

\newcommand{\class}{Math 0098}
\newcommand{\sem}{2201}
\newcommand{\due}{}
\newcommand{\sect}{8.4}
\newcommand{\topic}{Composite and Inverse Functions}

\firstpageheader{\class}{\sect - \topic}{}
\runningheader{\class}{\sect - \topic}{}
\firstpagefooter{\class}{}{Page \thepage\ of \numpages}
\runningfooter{\class}{}{Page \thepage\ of \numpages}

\newif\ifprintselected
\printselectedtrue
%\printselectedfalse

\newenvironment{select}
{\ifprintselected
	\printanswers
	\fi
}
{}

\theoremstyle{definition}
\newtheorem{theorem}{Theorem}
\newtheorem{example}{Example}[subsection]
%\newtheorem{definition}{Definition}
\newtheorem{definition}{Definition}[subsection]
%\newmdtheoremenv{example}{Example}[subsection]
\AtBeginEnvironment{defn}{\begin{minipage}{\textwidth}}
\AtEndEnvironment{defn}{\end{minipage}}
%\AtBeginEnvironment{example}{\begin{minipage}{\textwidth}}
%\AtEndEnvironment{example}{\end{minipage}}
\newcommand{\iu}{{i\mkern1mu}}

\setlength{\gridsize}{5mm}
\setlength{\gridlinewidth}{0.1pt}

\printanswers
\DeclareMathSizes{12}{12}{12}{12}

\begin{document}
\setcounter{section}{8}
\setcounter{subsection}{3}

\subsection{Composite and Inverse Functions}

\subsubsection*{Composite Functions}

Composite functions are \emph{composed} of two other functions. 

\vspace{.2in}

\begin{definition}[Function Composition]
Function composition uses the following notation:
\[ (f\circ g)(x) = f(g(x))\]

which reads as "$f$ composed with $g$ at/of $x$". The domain of $f\circ g$ is the set of all values of $x$ such that $x$ is in the domain of $g$ and $g(x)$ is in the domain of $f$.
\end{definition}

\vspace{.25in}

\begin{example}
Given $f(x) = 5x+6$ and $g(x) = x^2-1$, find the following:
\begin{enumerate}
\item $(f\circ g)(x) = $
\vspace{1.5in}
\item $(g\circ f)(x) = $
\vspace{2.5in}
\end{enumerate}
\end{example}

\newpage

\begin{example}
Given $f(x) = -2x+3$ and $g(x) = 2x^2-4x$, find the following:
\begin{enumerate}
\item $(f\circ g)(x) = $
\vspace{1.5in}
\item $(g\circ f)(x) = $
\vspace{2.5in}
\end{enumerate}
\end{example}

\subsubsection*{Inverse Functions}

Two functions are said to be inverses if they undo each other; that is, if $(f\circ g)(x) = f(g(x)) = x$. We denote the inverse function of $f(x)$ as $f^{-1}(x)$. The $-1$ in this notation is not an exponent.

\vspace{.2in}

\begin{definition}[Function Inverse]
Let $f$ and $g$ be two functions such that $f(g(x)) = x$ for all $x$ in the domain of $g$ and $g(f(x)) = x$ for all $x$ in the domain of $f$. The functions $f$ and $g$ are then \emph{inverses}.

\vspace{.2in}
If we notate the inverse of $f$ as $f^{-1}$, we then have $f(f^{-1}(x)) = x$ and $f^{-1}(f(x)) = x$. The domain of $f$ is the range of $f^{-1}$ and the domain of $f^{-1}$ is the range of $f$.
\end{definition}

\newpage

\begin{example}
Show that the functions $f(x) = 3x$ and $g(x) = \dfrac{x}{3}$ are inverses.
\vspace{3.5in}
\end{example}

\begin{example}
Show that the functions $f(x) = 4x-7$ and $g(x) = \dfrac{x+7}{4}$ are inverses.
\vspace{2.5in}
\end{example}

\newpage

\begin{mdframed}
\textbf{Finding the Inverse of a Function}
\vspace{.1in}
\begin{enumerate}
\item Replace $f(x)$ with $y$
\item Swap $x$ with $y$
\item Solve the new equation for $y$
\item Replace $y$ with $f^{-1}$
\item Verify that $(f\circ f^{-1})(x) = x$
\end{enumerate}

\end{mdframed}


\begin{example}
Find the inverse of $f(x) = 4x+5$.
\vspace{3.5in}
\end{example}

\begin{example}
Find the inverse of $f(x) = -3x - 7$.
\vspace{3.5in}
\end{example}

\newpage

\begin{example}
Find the inverse of $f(x) = \dfrac{2}{x-1}$.
\vspace{3.5in}
\end{example}

\begin{example}
Find the inverse of $f(x) = \dfrac{2x-1}{x+3}$.
\vspace{3.5in}
\end{example}
\end{document}