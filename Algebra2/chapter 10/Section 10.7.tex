\documentclass[addpoints,12pt]{exam}
\usepackage{amsmath}
\usepackage{amsthm}
\usepackage{amsfonts}
\usepackage{systeme}
\usepackage{graphicx}
\usepackage{caption}
\usepackage{xfrac}
\usepackage{physics}
\usepackage{microtype}
\usepackage{eulervm}
%\usepackage[framemethod=tikz]{mdframed}
\usepackage{thmtools}
\usepackage{etoolbox}
%\usepackage{fouriernc}
\usepackage{mdframed}
\usepackage[overload]{empheq}
\usepackage{adjustbox}
\usepackage{enumitem}
\usepackage[explicit]{titlesec}

\pagestyle{headandfoot}
\runningfootrule
\firstpageheadrule
\runningheadrule

\newcommand{\class}{Math 0098}
\newcommand{\sem}{2201}
\newcommand{\due}{}
\newcommand{\sect}{10.7}
\newcommand{\topic}{Complex Numbers}

\firstpageheader{\class}{\sect - \topic}{}
\runningheader{\class}{\sect - \topic}{}
\firstpagefooter{\class}{}{Page \thepage\ of \numpages}
\runningfooter{\class}{}{Page \thepage\ of \numpages}

\newif\ifprintselected
\printselectedtrue
%\printselectedfalse

\newenvironment{select}
{\ifprintselected
	\printanswers
	\fi
}
{}

\theoremstyle{definition}
\newtheorem{theorem}{Theorem}
\newtheorem{example}{Example}[subsection]
%\newtheorem{definition}{Definition}
\newtheorem{definition}{Definition}[subsection]
%\newmdtheoremenv{example}{Example}[subsection]
\AtBeginEnvironment{defn}{\begin{minipage}{\textwidth}}
\AtEndEnvironment{defn}{\end{minipage}}
%\AtBeginEnvironment{example}{\begin{minipage}{\textwidth}}
%\AtEndEnvironment{example}{\end{minipage}}
\newcommand{\iu}{{i\mkern1mu}}

\setlength{\gridsize}{5mm}
\setlength{\gridlinewidth}{0.1pt}

\printanswers
\DeclareMathSizes{12}{12}{12}{12}

\titleformat{\subsubsection}
   {\large\bfseries}% format
   {}% label
   {0pt}% sep
   {\titlerule \vspace{.1in} #1}% before code
      [{\titlerule[0.4pt]\vspace{.1in}}]% after code
\titlespacing{\subsubsection}
   {0pt}% left
   {0pt}% before sep
   {\baselineskip}% after sep


\begin{document}
\setcounter{section}{10}
\setcounter{subsection}{6}

\subsection{Complex Numbers}

\subsubsection*{The Imaginary Unit $\iu$}
Before we define complex numbers, we need to define \emph{imaginary numbers}. When solving some quadratic (or higher degree) equations, we come across solutions that don't necessarily seem to work. Consider the quadratic $y = x^2 +1$. If we were to factor that quadratic to find the zeroes (roots/solutions), we see that it, well, it doesn't factor. We could use the squareroot property to solve it though.
\begin{eqnarray*}
x^2 + 1 &=& 0\\
x^2 &=& -1\\
x &=& \pm\sqrt{-1}
\end{eqnarray*}

We define the \emph{imaginary unit} $\iu$ as the principal root of the above equation: $\iu = \sqrt{-1}$.

Using this imaginary unit, we can further reduce radicals that we previously could not.

\begin{mdframed}
\textbf{Taking the Square Root of a Negative Real Number}\mbox{}\\
If $b$ is a positive real number, then \[\sqrt{-b} = \sqrt{b\cdot -1} = \sqrt{b}\cdot\sqrt{-1} = \sqrt{b}\iu\]
\end{mdframed}

\begin{example}
Find each of the following:
\begin{enumerate}
\begin{minipage}{.5\textwidth}
\item $\sqrt{-9}$
\end{minipage}%
\begin{minipage}{.5\textwidth}
\item $\sqrt{-43}$
\end{minipage}%
\end{enumerate}
\vspace{1.5in}
\end{example}

\vspace{.25in}

\subsubsection*{Complex Numbers}
Complex numbers are named such because they are composed of two parts -- a \emph{real} part ($a$) and an \emph{imaginary} part ($b\i$) where both $a$ and $b$ are real numbers. The symbol $\mathbb{R}$ is used to represent the set of real numbers and the symbol $\mathbb{C}$ is used to represent the set of complex numbers. While there isn't a standard symbol to be used for imaginary numbers, the convention is to use $\iu\mathbb{R}$.

The set $\mathbb{C}$ of complex numbers is given as \[\mathbb{C} = \{a+b\iu \mid a\in \mathbb{R}, b \in\mathbb{R}\}\].

\subsubsection*{Operations on Complex Numbers}
Operations on complex numbers can be fairly straightforward. As far as addition and subtraction go, treat the real and imaginary parts as like terms and combine as appropriate.
\vspace{.25in}

\begin{mdframed}
\textbf{General Method for Adding, Subtracting and Multiplying}\mbox{}\\
\begin{enumerate}
\item $(a+b\iu) + (c+d\iu) = (a+c)+(b+d)\iu$
\item $(a+b\iu) - (c+d\iu) = (a-c)+(b-d)\iu$
\item $(a+b\iu)(c+d\iu) = (ac-db) + (ad+db)\iu$
\end{enumerate}
\end{mdframed}

\vspace{.25in}
While the definition for multiplying complex numbers looks unpleasant, it follows the FOIL method that we have employed throughout the semester. The only difference is that we will encounter an $\iu^2$ and will need to deal with that.

\vspace{.25in}
We know that $\iu = \sqrt{-1}$ by definition. If we want to find $\iu^2$, use the definition as follows:
\[\iu^2 = \sqrt{-1}^2 = -1\]

We will use a similar method to find other powers of $\iu$ later in this section.

\newpage

\begin{example}
Let $a = 3-4\iu$ and $b = 6+2\iu$. Find each of the following:
\begin{enumerate}
\item $a + b$
\vspace{1in}
\item $a - b$
\vspace{1in}
\item $a\cdot b$
\vspace{1.25in}
\end{enumerate}
\end{example}

\begin{example}
Let $a = -2+3\iu$ and $b = -5-7\iu$. Find each of the following:
\begin{enumerate}
\item $a + b$
\vspace{1in}
\item $a - b$
\vspace{1in}
\item $a\cdot b$
\end{enumerate}
\end{example}
\newpage

\subsubsection*{Dividing Complex Numbers}

In Chapter 10, we simplified expressions involving radicals -- this included dividing radical expressions by other radical expressions. In order to do so, we used the \emph{conjugate} to remove (\emph{rationalize}) the radical from the denominator. Dividing complex numbers uses a similar technique. 

\vspace{.25in}

\begin{mdframed}
\textbf{Finding the Conjugate of a Complex Number}\mbox{}\\
The conjugate of a complex number is found much like the conjugate of a radical expression. Identify the sign (or operation) and change to its inverse. In general, the conjugate of a complex number $z = a+b\iu$ is given as $\bar{z} = a - b\iu$.
\end{mdframed}

\begin{example}
Find the complex conjugate of each.
\begin{enumerate}
\begin{minipage}{.5\textwidth}
\item $12 - 4\iu$
\end{minipage}%
\begin{minipage}{.5\textwidth}
\item $-3 + 2\iu$
\end{minipage}%
\end{enumerate}
\vspace{1in}
\end{example}

\begin{mdframed}
\textbf{Dividing Complex Numbers}\mbox{}\\
Dividing complex numbers involves multiplying and dividing by the conjugate of the denominator. In general, we say \[\dfrac{a + b\iu}{c + d\iu} = \dfrac{a + b\iu}{c + d\iu}\cdot \dfrac{c - d\iu}{c - d\iu} = \dfrac{(a+b\iu)(c-d\iu)}{(c+d\iu)(c-d\iu)} = \dfrac{(a+b\iu)(c-d\iu)}{c^2+d^2}\]
\end{mdframed}

\begin{example}
Divide and write as a complex number in the form $a + b\iu$.
\[\dfrac{6+2\iu}{4-3\iu}\]
\vspace{2in}
\end{example}

\newpage

\begin{example}
Divide and write as a complex number in the form $a + b\iu$.
\[\dfrac{3-2\iu}{4\iu}\]
\vspace{1.5in}
\end{example}

\subsubsection*{Powers of $\iu$}
We used the fact that $\iu^2 = -1$ to multiply complex numbers, but what about other powers of $\iu$? The first few are easy enough to compute:
\begin{eqnarray*}
\iu &=& \sqrt{-1}\\
\iu^2 &=& \sqrt{-1}^2 = -1\\
\iu^3 &=& \iu^2\cdot\iu = -1\cdot\iu = -\iu\\
\iu^4 &=& \iu^2\cdot\iu^2 = (-1)(-1) = 1
\end{eqnarray*}

After the fourth power, the values become cyclical -- that is, they begin to repeat: $\iu^5 = \iu, \iu^6 = -1, \iu^7 = -\iu, \dots$.

\vspace{.2in}

\begin{mdframed}
\textbf{Simplifying Powers of $\iu$}\mbox{}\\
Use the fact that $\iu^4 = 1$ and that $1^m = 1$ for any value of $m$ in order to simplify powers of $\iu$. Start by finding the largest multiple of 4 less than the exponent. Break the original exponent into two problems using exponent rules -- one that is a multiple of 4 and one that is the remainder. The factor that has a multiple of 4 becomes 1, leaving you with the second factor.
\end{mdframed}

\vspace{.15in}

\begin{example}
Simplify each of the following:
\begin{enumerate}
\begin{minipage}{.3\textwidth}
\item $\iu^{16}$
\end{minipage}%
\begin{minipage}{.3\textwidth}
\item $\iu^{25}$
\end{minipage}%
\begin{minipage}{.3\textwidth}
\item $\iu^{35}$
\end{minipage}%
\end{enumerate}
\end{example}

\end{document}