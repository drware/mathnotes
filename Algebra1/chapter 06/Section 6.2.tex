\documentclass[addpoints,12pt]{exam}
\usepackage{amsmath}
\usepackage{amsthm}
\usepackage{amsfonts}
\usepackage{systeme}
\usepackage{graphicx}
\usepackage{caption}
\usepackage{xfrac}
\usepackage{physics}
\usepackage{microtype}
\usepackage{eulervm}
%\usepackage[framemethod=tikz]{mdframed}
\usepackage{thmtools}
\usepackage{etoolbox}
%\usepackage{fouriernc}
\usepackage{mdframed}
\usepackage[overload]{empheq}
\usepackage{adjustbox}
\usepackage{enumitem}
\usepackage[explicit]{titlesec}
% adds in \varnothing for empty set
\usepackage{amssymb}
% adds in formated SI units
%\usepackage{siunitx}
\usepackage{pgfplots}
\usepackage{multirow}
\usepackage{array}

\pagestyle{headandfoot}
\runningfootrule
\firstpageheadrule
\runningheadrule

\newcommand{\class}{Math 0097}
\newcommand{\sem}{2211}
\newcommand{\due}{}
\newcommand{\sect}{6.2}
\newcommand{\topic}{Factoring Trinomials with Leading Coefficient 1}

\firstpageheader{\class}{\sect - \topic}{}
\runningheader{\class}{\sect - \topic}{}
\firstpagefooter{\class}{}{Page \thepage\ of \numpages}
\runningfooter{\class}{}{Page \thepage\ of \numpages}

\newif\ifprintselected
\printselectedtrue
%\printselectedfalse

\newenvironment{select}
{\ifprintselected
	\printanswers
	\fi
}
{}

\theoremstyle{definition}
\newtheorem{theorem}{Theorem}
%\newtheorem{example}{Example}[subsection]
%\newtheorem{definition}{Definition}
%\newmdtheoremenv{definition}{Definition}[subsection]
%\newmdtheoremenv{example}{Example}[subsection]
\AtBeginEnvironment{defn}{\begin{minipage}{\textwidth}}
\AtEndEnvironment{defn}{\end{minipage}}
%\AtBeginEnvironment{example}{\begin{minipage}{\textwidth}}
%\AtEndEnvironment{example}{\end{minipage}}
\newcommand{\iu}{{i\mkern1mu}}

\setlength{\gridsize}{5mm}
\setlength{\gridlinewidth}{0.1pt}

\printanswers
\DeclareMathSizes{12}{12}{12}{12}

%%%%%%%%%%%%%%%%%%%%%%%%
% Create bars around subsubsection
%%%%%%%%%%%%%%%%%%%%%%%%

\titleformat{\subsubsection}
   {\large\bfseries}% format
   {}% label
   {0pt}% sep
   {\titlerule \vspace{.1in} #1}% before code
      [{\titlerule[0.4pt]\vspace{.1in}}]% after code
\titlespacing{\subsubsection}
   {0pt}% left
   {0pt}% before sep
   {\baselineskip}% after sep
   
%%%%%%%%%%%%%%%%%%%%%%%
% Create line break after definition label
%%%%%%%%%%%%%%%%%%%%%%%   
\newtheoremstyle{break}
  {\topsep}{\topsep}%
  {}{}%\itshape
  {\bfseries}{}%
  {\newline}{}%
\theoremstyle{break}
\newmdtheoremenv{definition}{Definition}[subsection]
\theoremstyle{break}
\newtheorem{example}{Example}[subsection]

%%%%%%%%%%%%%%%%%%%%%%
% start document
% set section, subsection (use n-1 for sub)
%%%%%%%%%%%%%%%%%%%%%%


\begin{document}
\setcounter{section}{6}
\setcounter{subsection}{1}

\subsection{Factoring Trinomials with Leading Coefficient 1}

\vspace{.15in}
If we have a polynomial of the form $x^2 + bx + c$, we can try to factor it using the "product-sum" method.
\\
Say that we have a factored polynomial written as $(x+r_1)(x+r_2)$. We can FOIL this product and make a comparison with $x^2 + bx + c$.
\\
\begin{eqnarray*}
(x+r_1)(x+r_2) &=& x^2 + r_2x + r_1x + r_1r_2\\
&=& x^2 + (r_1+r_2)x + r_1r_2\\
&=& x^2 + bx + c
\end{eqnarray*}

Looking at the above work, we can determine that $b = r_1 + r_2$ and $c = r_1r_2$. This gives us the "product-sum" method of factoring.
\\
\begin{mdframed}
\textbf{Factoring with "Product-Sum"}
\begin{enumerate}
\item Find two numbers that multiply to $c$ and that add to $b$.
\item Determine the signs of each.
\item Write as the product of two binomials.
\end{enumerate}
\end{mdframed}

\vspace{.15in}

\begin{example}
Factor $x^2 + 5x + 6$
\vspace{1.25in}
\end{example}

\begin{example}
Factor $x^2 - 6x + 8$
\end{example}

\newpage

\begin{example}
Factor $x^2 + 3x - 10$
\vspace{1.25in}
\end{example}

\begin{example}
Factor $x^2 + x - 7$
\vspace{1.25in}
\end{example}

\begin{example}
Factor $x^2 - 4xy + 3y^2$
\vspace{2in}
\end{example}

\noindent Multiple types of factoring can be combined. In almost every case of factoring, you should attempt the GCF method first and then apply some other method.
\vspace{.15in}

\begin{example}
Factor completely: $2x^3 + 6x^2 - 56x$
\vspace{1.5in}
\end{example}

\begin{example}
Factor completely: $-2y^2 - 10y + 28$
\end{example}
\end{document}