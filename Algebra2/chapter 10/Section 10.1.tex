\documentclass[addpoints,12pt]{exam}
\usepackage{amsmath}
\usepackage{amsthm}
\usepackage{amsfonts}
\usepackage{systeme}
\usepackage{graphicx}
\usepackage{caption}
\usepackage{xfrac}
\usepackage{physics}
\usepackage{microtype}
\usepackage{eulervm}
%\usepackage[framemethod=tikz]{mdframed}
\usepackage{thmtools}
\usepackage{etoolbox}
%\usepackage{fouriernc}
\usepackage{mdframed}
\usepackage[overload]{empheq}

\pagestyle{headandfoot}
\runningfootrule
\firstpageheadrule
\runningheadrule

\newcommand{\class}{Math 0098}
\newcommand{\sem}{2201}
\newcommand{\due}{}
\newcommand{\sect}{10.1}
\newcommand{\topic}{Radical Expressions and Functions}

\firstpageheader{\class}{\sect - \topic}{}
\runningheader{\class}{\sect - \topic}{}
\firstpagefooter{\class}{}{Page \thepage\ of \numpages}
\runningfooter{\class}{}{Page \thepage\ of \numpages}

\newif\ifprintselected
\printselectedtrue
%\printselectedfalse

\newenvironment{select}
{\ifprintselected
	\printanswers
	\fi
}
{}

\theoremstyle{definition}
\newtheorem{theorem}{Theorem}
\newtheorem{example}{Example}[subsection]
%\newtheorem{definition}{Definition}
\newtheorem{definition}{Definition}[subsection]
%\newmdtheoremenv{example}{Example}[subsection]
\AtBeginEnvironment{defn}{\begin{minipage}{\textwidth}}
\AtEndEnvironment{defn}{\end{minipage}}
%\AtBeginEnvironment{example}{\begin{minipage}{\textwidth}}
%\AtEndEnvironment{example}{\end{minipage}}
\newcommand{\iu}{{i\mkern1mu}}

\setlength{\gridsize}{5mm}
\setlength{\gridlinewidth}{0.1pt}

\printanswers
\DeclareMathSizes{12}{12}{12}{12}

\begin{document}
\setcounter{section}{10}
\setcounter{subsection}{0}

\subsection{Radical Expressions and Functions}

\begin{definition}[Principal Square Root]
If $a$ is a non-negative real number, then the non-negative number $b$ such that $b^2 = a$, denoted by $b=\sqrt{a}$, is the principal square root of $a$.
\end{definition}
\vspace{.25in}
\begin{example}
Evaluate each of the following square roots.
\begin{enumerate}
\begin{minipage}{.3\textwidth}
\item $\sqrt{64}$
\vspace{.5in}
\item $-\sqrt{49}$
\vspace{.5in}
\end{minipage}%
\begin{minipage}{.3\textwidth}
\item $\sqrt{\dfrac{16}{25}}$
\vspace{.5in}
\item $\sqrt{0.0081}$
\vspace{.5in}
\end{minipage}%
\begin{minipage}{.3\textwidth}
\item $\sqrt{9+16}$
\vspace{.5in}
\item $\sqrt{9}+\sqrt{16}$
\vspace{.5in}
\end{minipage}%
\end{enumerate}
\end{example}
\vspace{.25in}

\subsubsection*{Functions with Square Roots}

We can define the square root as a function with $f(x) = \sqrt{x}$. Both the domain and range of this function are the non-negative numbers - $[0,\infty)$.

\vspace{.25in}

To evaluate square root functions, we treat them the same as anything - make the substitution for the independent variable and simplify.

\vspace{.25in}

\begin{example}
Find the indicated value for each given function:
\begin{enumerate}
\item $f(3)$ when $f(x) = \sqrt{12x-20}$
\vspace{1.25in}
\item $g(-5)$ when $g(x) = -\sqrt{9-3x}$

\end{enumerate}
\end{example}

\newpage

The square root function is only defined when the \emph{radicand} - the portion under the radical - is non-negative. If we need to find the domain of a square root function, set the radicand greater than or equal to 0 and solve for $x$. State the domain using whichever method is specified.

\vspace{.25in}

\begin{example}
Find the domain of each of the following functions:
\begin{enumerate}
\item $f(x) = \sqrt{9x-27}$
\vspace{1.5in}
\item $g(x) = -3\sqrt{2(3x-4)}+4$
\vspace{1.5in}
\end{enumerate}
\end{example}

\subsubsection*{Simplifying $\sqrt{a^2}$}

For any real value of $a$, we have \[\sqrt{a^2} = \abs{a}\]

\begin{example}
Simplify:
\begin{enumerate}
\item $\sqrt{(-7)^2}$
\vspace{1.25in}
\newpage
\item $\sqrt{(x+8)^2}$
\vspace{1.25in}
\item $\sqrt{49x^{10}}$
\vspace{1.25in}
\item $\sqrt{x^2-6x+9}$
\vspace{1.5in}
\end{enumerate}
\end{example}

\subsubsection*{Cube Roots}
Similar to a square root, a cube root is given as $\sqrt[3]{a} = b$ where $b^3 = a$. The $3$ is the \emph{index} of the radical. Unlike the square root, however, the cube root has negative numbers in its domain. Both the domain and range of $f(x) = \sqrt[3]{x}$ are $(-\infty,\infty)$.

\begin{example}
Find the indicated value for each given function:
\begin{enumerate}
\item $f(127)$ when $f(x) = \sqrt[3]{x-2}$
\vspace{1.5in}
\newpage
\item $g(-7)$ when $g(x) = \sqrt[3]{8x-8}$
\vspace{1.5in}
\end{enumerate}
\end{example}

\subsubsection*{Simplifying $\sqrt[3]{a^3}$}
For any real number $a$, we have \[\sqrt[3]{a^3} = a\].

\begin{example}
Simplify the following: \[\sqrt[3]{-27x^3}\]
\end{example}


\subsubsection*{Simplifying Odd or Even Roots}

For any real number $a$:
\begin{itemize}
\item If $n$ is even, $\sqrt[n]{a^n} = \abs{a}$.
\item If $n$ is odd, $\sqrt[n]{a^n} = a$.
\end{itemize}

\begin{example}
Find each of the following:
\begin{enumerate}
\item $\sqrt[4]{(x+6)^4}$
\vspace{1.5in}

\newpage
\item $\sqrt[5]{(3x-2)^5}$
\vspace{1.5in}
\item $\sqrt[6]{(-8)^6}$
\vspace{1.5in}
\end{enumerate}
\end{example}


\end{document}