\documentclass[addpoints,12pt]{exam}
\usepackage{amsmath}
\usepackage{amsthm}
\usepackage{amsfonts}
\usepackage{systeme}
\usepackage{graphicx}
\usepackage{caption}
\usepackage{xfrac}
\usepackage{physics}
\usepackage{microtype}
\usepackage{eulervm}
%\usepackage[framemethod=tikz]{mdframed}
\usepackage{thmtools}
\usepackage{etoolbox}
%\usepackage{fouriernc}
\usepackage{mdframed}
\usepackage[overload]{empheq}
\usepackage{adjustbox}
\usepackage{enumitem}
\usepackage[explicit]{titlesec}

\pagestyle{headandfoot}
\runningfootrule
\firstpageheadrule
\runningheadrule

\newcommand{\class}{Math 0097}
\newcommand{\sem}{2211}
\newcommand{\due}{}
\newcommand{\sect}{1.1}
\newcommand{\topic}{Intro to Variables and Models}

\firstpageheader{\class}{\sect - \topic}{}
\runningheader{\class}{\sect - \topic}{}
\firstpagefooter{\class}{}{Page \thepage\ of \numpages}
\runningfooter{\class}{}{Page \thepage\ of \numpages}

\newif\ifprintselected
\printselectedtrue
%\printselectedfalse

\newenvironment{select}
{\ifprintselected
	\printanswers
	\fi
}
{}

\theoremstyle{definition}
\newtheorem{theorem}{Theorem}
\newtheorem{example}{Example}[subsection]
%\newtheorem{definition}{Definition}
\newtheorem{definition}{Definition}[subsection]
%\newmdtheoremenv{example}{Example}[subsection]
\AtBeginEnvironment{defn}{\begin{minipage}{\textwidth}}
\AtEndEnvironment{defn}{\end{minipage}}
%\AtBeginEnvironment{example}{\begin{minipage}{\textwidth}}
%\AtEndEnvironment{example}{\end{minipage}}
\newcommand{\iu}{{i\mkern1mu}}

\setlength{\gridsize}{5mm}
\setlength{\gridlinewidth}{0.1pt}

\printanswers
\DeclareMathSizes{12}{12}{12}{12}

\titleformat{\subsubsection}
   {\large\bfseries}% format
   {}% label
   {0pt}% sep
   {\titlerule \vspace{.1in} #1}% before code
      [{\titlerule[0.4pt]\vspace{.1in}}]% after code
\titlespacing{\subsubsection}
   {0pt}% left
   {0pt}% before sep
   {\baselineskip}% after sep


\begin{document}
\setcounter{section}{1}
\setcounter{subsection}{0}

\subsection{Intro to Variables and Models}

\begin{definition}[Variable]
a letter or character representing a variety of numbers or quantities
\end{definition}

\begin{definition}[Algebraic Expression]
a statement that relates variables and numbers (constants) through a combination of algebraic operations
\end{definition}

\vspace{.5in}
\noindent What are operations? Which are the most common?

\vspace{.75in}

\begin{example}
Rewrite each example as an algebraic expression.
\begin{enumerate}
\item two more than a number
\vspace{.5in}
\item three less than a value
\vspace{.5in}
\item a number divided by six
\vspace{.5in}
\item four times a number
\vspace{.5in}
\item five more than three times a value
\end{enumerate}
\vspace{.5in}
\end{example}

\newpage

\subsubsection*{Evaluating Algebraic Expressions}
\begin{itemize}
\item replace the variable with the given value
\item simplify the expression using PEMDAS
\end{itemize}

\vspace{.5in}

\begin{example}
Evaluate each of the following:
\begin{enumerate}
\item $2 + 3x$; $x = 4$
\vspace{1in}
\item $5(x-2)$; $x = -3$
\vspace{1in}
\item $2x-7y$; $x = 3, y = 2$
\vspace{1in}
\item $\dfrac{x+7y-4}{3x-y}$; $x = 1, y = 5$
\end{enumerate}
\end{example}

\newpage

\subsubsection*{Equations}
\begin{definition}[Equations]
An equation is a statement that two algebraic expressions are the same (equivalent). They always contain an equal $(=)$ sign.
\end{definition}

\begin{definition}[Solution(s) of an Equation]
Solutions are values of a variable(s) that make an equation true meaning that when the expressions are evaluated, both sides of the equals sign are the same. We find the solutions by solving for a specific variable.
\end{definition}
\vspace{.5in}
\begin{example}
Determine whether or not the given value is a solution to each.
\begin{enumerate}
\item $9x - 3 = 42$; $x = 6$
\vspace{1in}
\item $9x - 3 = 42$; $x = 5$
\vspace{1in}
\item $2(y+3) = 5y-3$; $y = 4$
\vspace{1in}
\item $2(y+3) = 5y-3$; $ y = 3$
\end{enumerate}
\end{example}
\newpage
\begin{example}
Write each sentence as an equation.
\begin{enumerate}
\item The quotient of a number and six is five.
\vspace{.5in}
\item Seven decreased by twice a number yields one.
\vspace{.5in}
\end{enumerate}
\end{example}

\subsubsection*{Formulas \& Models}
\begin{definition}[Formula]
an equation that expresses the relationship between two or more variables
\end{definition}
\vspace{.1in}

\begin{definition}[Mathematical Model]
a formula that describes real-world situations and processes
\end{definition}
\vspace{.25in}

\begin{example}
The number of times my dogs bark in a day, $B$, is determined by the number of people who walk by my house, $w$.

\vspace{.1in}
\begin{enumerate}
\item Find the equation representing this situation.
\vspace{1in}
\item How many times do they bark if two people walk by?
\vspace{1in}
\item How many times do they bark if ten people walk by?
\vspace{1in}
\item How many times do they bark if no one walks by?
\vspace{1in}
\item How many times do they bark if -5 people walk by?
\end{enumerate}
\end{example}
\end{document}