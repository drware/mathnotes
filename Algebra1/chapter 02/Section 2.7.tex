\documentclass[addpoints,12pt]{exam}
\usepackage{amsmath}
\usepackage{amsthm}
\usepackage{amsfonts}
\usepackage{systeme}
\usepackage{graphicx}
\usepackage{caption}
\usepackage{xfrac}
\usepackage{physics}
\usepackage{microtype}
\usepackage{eulervm}
%\usepackage[framemethod=tikz]{mdframed}
\usepackage{thmtools}
\usepackage{etoolbox}
%\usepackage{fouriernc}
\usepackage{mdframed}
\usepackage[overload]{empheq}
\usepackage{adjustbox}
\usepackage{enumitem}
\usepackage[explicit]{titlesec}
% adds in \varnothing for empty set
\usepackage{amssymb}
% adds in formated SI units
%\usepackage{siunitx}

\pagestyle{headandfoot}
\runningfootrule
\firstpageheadrule
\runningheadrule

\newcommand{\class}{Math 0097}
\newcommand{\sem}{2211}
\newcommand{\due}{}
\newcommand{\sect}{2.7}
\newcommand{\topic}{Solving Linear Inequalities}

\firstpageheader{\class}{\sect - \topic}{}
\runningheader{\class}{\sect - \topic}{}
\firstpagefooter{\class}{}{Page \thepage\ of \numpages}
\runningfooter{\class}{}{Page \thepage\ of \numpages}

\newif\ifprintselected
\printselectedtrue
%\printselectedfalse

\newenvironment{select}
{\ifprintselected
	\printanswers
	\fi
}
{}

\theoremstyle{definition}
\newtheorem{theorem}{Theorem}
%\newtheorem{example}{Example}[subsection]
%\newtheorem{definition}{Definition}
%\newmdtheoremenv{definition}{Definition}[subsection]
%\newmdtheoremenv{example}{Example}[subsection]
\AtBeginEnvironment{defn}{\begin{minipage}{\textwidth}}
\AtEndEnvironment{defn}{\end{minipage}}
%\AtBeginEnvironment{example}{\begin{minipage}{\textwidth}}
%\AtEndEnvironment{example}{\end{minipage}}
\newcommand{\iu}{{i\mkern1mu}}

\setlength{\gridsize}{5mm}
\setlength{\gridlinewidth}{0.1pt}

\printanswers
\DeclareMathSizes{12}{12}{12}{12}

%%%%%%%%%%%%%%%%%%%%%%%%
% Create bars around subsubsection
%%%%%%%%%%%%%%%%%%%%%%%%

\titleformat{\subsubsection}
   {\large\bfseries}% format
   {}% label
   {0pt}% sep
   {\titlerule \vspace{.1in} #1}% before code
      [{\titlerule[0.4pt]\vspace{.1in}}]% after code
\titlespacing{\subsubsection}
   {0pt}% left
   {0pt}% before sep
   {\baselineskip}% after sep
   
%%%%%%%%%%%%%%%%%%%%%%%
% Create line break after definition label
%%%%%%%%%%%%%%%%%%%%%%%   
\newtheoremstyle{break}
  {\topsep}{\topsep}%
  {}{}%\itshape
  {\bfseries}{}%
  {\newline}{}%
\theoremstyle{break}
\newmdtheoremenv{definition}{Definition}[subsection]
\theoremstyle{break}
\newtheorem{example}{Example}[subsection]

%%%%%%%%%%%%%%%%%%%%%%
% start document
% set section, subsection (use n-1 for sub)
%%%%%%%%%%%%%%%%%%%%%%


\begin{document}
\setcounter{section}{2}
\setcounter{subsection}{6}

\subsection{Solving Linear Inequalities}

\vspace{.25in}

\noindent A \emph{linear inequality} in \emph{one} variable is written in the form $ax + b \le c$ where the $\le$ can be replaced with any inequality symbol -- $<,\; >,\; \le,\; \ge$. Solving a linear inequality gives a \textbf{solution set} as opposed to a single solution. The solution set can be expressed as an interval or inequality and can be graphed on a number line.

\vspace{.15in}

\begin{figure}[h]
\begin{tabular}{c | c | c | c}
\textbf{Inequality} & \textbf{Interval} & \textbf{Set-Builder} & \textbf{Graph} \\\hline
& $\;\;\;\;\;\;\;\;\;\;\;\;\;\;\;\;\;\;\;\;\;\;\;\;\;\;\;\;\;\;\;\;\;\;\;$&$\;\;\;\;\;\;\;\;\;\;\;\;\;\;\;\;\;\;\;\;\;\;\;\;\;\;\;\;\;\;\;\;\;\;\;$ &$\;\;\;\;\;\;\;\;\;\;\;\;\;\;\;\;\;\;\;\;\;\;\;\;\;\;\;\;\;\;\;\;\;\;\;$\\
$x > a$ & & & \\
& & &\\\hline
& & &\\
$x \ge a$ & & & \\
& & &\\\hline
& & &\\
$x < a$ & & & \\
& & &\\\hline
& & &\\
$x \le a$ & & & \\
& & &\\\hline

\end{tabular}
\end{figure}

\vspace{.15in}

\begin{example}
Solve and graph the following inequality:
\[x + 6 < 9\]
\end{example}

\newpage

\begin{example}
Solve and graph the following inequality:
\[8x - 2 \ge 7x - 4\]
\vspace{1.5in}
\end{example}

\noindent \emph{Note:} When working with inequalities, if you \emph{multiply or divide} by a negative number, you must \emph{flip the direction of the inequality}.
\vspace{.15in}

\begin{example}
Solve and graph the inequality:
\[-6x < 18\]
\vspace{1.5in}
\end{example}

\begin{example}
Solve and graph:
\[5y - 3 \ge 17\]
\end{example}

\newpage

\begin{example}
Solve and graph:
\[6 - 3x \le 5x - 2\]
\vspace{2in}
\end{example}

\begin{example}
Solve and graph:
\[2(x-3) -1 \le 3(x+2) - 14\]
\vspace{2in}
\end{example}

\begin{example}
Solve and graph:
\[4(x+2) > 4x + 15\]
\vspace{2in}
\end{example}

\newpage

\begin{example}
Solve and graph:
\[2(x+5) \le 5x - 3x + 14\]
\vspace{2in}
\end{example}

\begin{example}
You must have an average score of 80\% to earn a B. On your first three tests, you have scores of 82\%, 74\% and 78\%. If the final counts as two grades, what must you get on the final to earn a B in the class?
\\

\noindent Remember that the average of a set of values is found by adding all the values and dividing by the total number of values.
\end{example}
\end{document}