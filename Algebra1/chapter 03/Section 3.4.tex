\documentclass[addpoints,12pt]{exam}
\usepackage{amsmath}
\usepackage{amsthm}
\usepackage{amsfonts}
\usepackage{systeme}
\usepackage{graphicx}
\usepackage{caption}
\usepackage{xfrac}
\usepackage{physics}
\usepackage{microtype}
\usepackage{eulervm}
%\usepackage[framemethod=tikz]{mdframed}
\usepackage{thmtools}
\usepackage{etoolbox}
%\usepackage{fouriernc}
\usepackage{mdframed}
\usepackage[overload]{empheq}
\usepackage{adjustbox}
\usepackage{enumitem}
\usepackage[explicit]{titlesec}
% adds in \varnothing for empty set
\usepackage{amssymb}
% adds in formated SI units
%\usepackage{siunitx}

\pagestyle{headandfoot}
\runningfootrule
\firstpageheadrule
\runningheadrule

\newcommand{\class}{Math 0097}
\newcommand{\sem}{2211}
\newcommand{\due}{}
\newcommand{\sect}{3.4}
\newcommand{\topic}{The Slope-Intercept Form}

\firstpageheader{\class}{\sect - \topic}{}
\runningheader{\class}{\sect - \topic}{}
\firstpagefooter{\class}{}{Page \thepage\ of \numpages}
\runningfooter{\class}{}{Page \thepage\ of \numpages}

\newif\ifprintselected
\printselectedtrue
%\printselectedfalse

\newenvironment{select}
{\ifprintselected
	\printanswers
	\fi
}
{}

\theoremstyle{definition}
\newtheorem{theorem}{Theorem}
%\newtheorem{example}{Example}[subsection]
%\newtheorem{definition}{Definition}
%\newmdtheoremenv{definition}{Definition}[subsection]
%\newmdtheoremenv{example}{Example}[subsection]
\AtBeginEnvironment{defn}{\begin{minipage}{\textwidth}}
\AtEndEnvironment{defn}{\end{minipage}}
%\AtBeginEnvironment{example}{\begin{minipage}{\textwidth}}
%\AtEndEnvironment{example}{\end{minipage}}
\newcommand{\iu}{{i\mkern1mu}}

\setlength{\gridsize}{5mm}
\setlength{\gridlinewidth}{0.1pt}

\printanswers
\DeclareMathSizes{12}{12}{12}{12}

%%%%%%%%%%%%%%%%%%%%%%%%
% Create bars around subsubsection
%%%%%%%%%%%%%%%%%%%%%%%%

\titleformat{\subsubsection}
   {\large\bfseries}% format
   {}% label
   {0pt}% sep
   {\titlerule \vspace{.1in} #1}% before code
      [{\titlerule[0.4pt]\vspace{.1in}}]% after code
\titlespacing{\subsubsection}
   {0pt}% left
   {0pt}% before sep
   {\baselineskip}% after sep
   
%%%%%%%%%%%%%%%%%%%%%%%
% Create line break after definition label
%%%%%%%%%%%%%%%%%%%%%%%   
\newtheoremstyle{break}
  {\topsep}{\topsep}%
  {}{}%\itshape
  {\bfseries}{}%
  {\newline}{}%
\theoremstyle{break}
\newmdtheoremenv{definition}{Definition}[subsection]
\theoremstyle{break}
\newtheorem{example}{Example}[subsection]

%%%%%%%%%%%%%%%%%%%%%%
% start document
% set section, subsection (use n-1 for sub)
%%%%%%%%%%%%%%%%%%%%%%


\begin{document}
\setcounter{section}{3}
\setcounter{subsection}{3}

\subsection{The Slope-Intercept Form}

\vspace{.15in}

Consider the equation $y = mx + b$. We know that both $x$ and $y$ represent variables or unknowns, but we haven't yet discussed $m$ and $b$ in detail. From the previous section, we know that $m$ represents the slope of the equation -- how quickly $y$ changes as $x$ changes. The last value, $b$, represents the $y$-intercept -- a topic from a few sections back -- where the line crosses the $y$-axis.

\vspace{.15in}

\begin{example}
Identify the slope and give the $y$-intercept as a point for each equation below.
\begin{enumerate}
\item $y = 5x-3$
\vspace{.5in}
\item $y = \dfrac{2}{3}x + 4$
\vspace{.5in}
\item $7x + y = 6$
\vspace{1in}
\end{enumerate}
\end{example}

\subsubsection*{Graphing with $y=mx+b$}
\begin{enumerate}
\item Identify and plot the $y$-intercept.
\item Use the slope to find and plot a second point.
\begin{enumerate}
\item Rewrite the slope as a fraction.
\item Move up/down by the numerator (rise).
\item Move left/right by the denominator (run).
\end{enumerate}
\end{enumerate}

\newpage

\begin{example}
Graph $y = 3x - 2$ using the slope and $y$-intercept.
\vspace{3in}
\end{example}

\begin{example}
Graph $y = \dfrac{3}{5}x + 1$ using the slope and $y$-intercept.
\end{example}

\newpage

\begin{example}
Graph $3x+4y = 0$ using the slope and $y$-intercept. (\emph{Hint: Solve for $y$.})
\end{example}

\end{document}