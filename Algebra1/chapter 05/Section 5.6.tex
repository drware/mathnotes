\documentclass[addpoints,12pt]{exam}
\usepackage{amsmath}
\usepackage{amsthm}
\usepackage{amsfonts}
\usepackage{systeme}
\usepackage{graphicx}
\usepackage{caption}
\usepackage{xfrac}
\usepackage{physics}
\usepackage{microtype}
\usepackage{eulervm}
%\usepackage[framemethod=tikz]{mdframed}
\usepackage{thmtools}
\usepackage{etoolbox}
%\usepackage{fouriernc}
\usepackage{mdframed}
\usepackage[overload]{empheq}
\usepackage{adjustbox}
\usepackage{enumitem}
\usepackage[explicit]{titlesec}
% adds in \varnothing for empty set
\usepackage{amssymb}
% adds in formated SI units
%\usepackage{siunitx}
\usepackage{pgfplots}
\usepackage{multirow}
\usepackage{array}

\pagestyle{headandfoot}
\runningfootrule
\firstpageheadrule
\runningheadrule

\newcommand{\class}{Math 0097}
\newcommand{\sem}{2211}
\newcommand{\due}{}
\newcommand{\sect}{5.6}
\newcommand{\topic}{Long \& Synthetic Division of Polynomials}

\firstpageheader{\class}{\sect - \topic}{}
\runningheader{\class}{\sect - \topic}{}
\firstpagefooter{\class}{}{Page \thepage\ of \numpages}
\runningfooter{\class}{}{Page \thepage\ of \numpages}

\newif\ifprintselected
\printselectedtrue
%\printselectedfalse

\newenvironment{select}
{\ifprintselected
	\printanswers
	\fi
}
{}

\theoremstyle{definition}
\newtheorem{theorem}{Theorem}
%\newtheorem{example}{Example}[subsection]
%\newtheorem{definition}{Definition}
%\newmdtheoremenv{definition}{Definition}[subsection]
%\newmdtheoremenv{example}{Example}[subsection]
\AtBeginEnvironment{defn}{\begin{minipage}{\textwidth}}
\AtEndEnvironment{defn}{\end{minipage}}
%\AtBeginEnvironment{example}{\begin{minipage}{\textwidth}}
%\AtEndEnvironment{example}{\end{minipage}}
\newcommand{\iu}{{i\mkern1mu}}

\setlength{\gridsize}{5mm}
\setlength{\gridlinewidth}{0.1pt}

\printanswers
\DeclareMathSizes{12}{12}{12}{12}

%%%%%%%%%%%%%%%%%%%%%%%%
% Create bars around subsubsection
%%%%%%%%%%%%%%%%%%%%%%%%

\titleformat{\subsubsection}
   {\large\bfseries}% format
   {}% label
   {0pt}% sep
   {\titlerule \vspace{.1in} #1}% before code
      [{\titlerule[0.4pt]\vspace{.1in}}]% after code
\titlespacing{\subsubsection}
   {0pt}% left
   {0pt}% before sep
   {\baselineskip}% after sep
   
%%%%%%%%%%%%%%%%%%%%%%%
% Create line break after definition label
%%%%%%%%%%%%%%%%%%%%%%%   
\newtheoremstyle{break}
  {\topsep}{\topsep}%
  {}{}%\itshape
  {\bfseries}{}%
  {\newline}{}%
\theoremstyle{break}
\newmdtheoremenv{definition}{Definition}[subsection]
\theoremstyle{break}
\newtheorem{example}{Example}[subsection]

%%%%%%%%%%%%%%%%%%%%%%
% start document
% set section, subsection (use n-1 for sub)
%%%%%%%%%%%%%%%%%%%%%%


\begin{document}
\setcounter{section}{5}
\setcounter{subsection}{5}

\subsection{Long \& Synthetic Division of Polynomials}

\vspace{.15in}

Long division of real numbers is a method of determining how many times one number goes into another number and what the remainder would be. For example, we could say that 5 goes into 17 three times with a remainder of 2. That is, $17 = 3\cdot 5 + 2$.

\begin{example}
Find $3983 \divisionsymbol 26$ using long division. Do not give a decimal as an answer.
\vspace{2in}
\end{example}

As we saw in the last section, we can divide polynomials as well. This process works fine when we divide by a \emph{monomial}, but what if we want to divide by any another polynomial instead? We accomplish this using either \emph{polynomial long division} or \emph{polynomial synthetic division}.
\vspace{.15in}
\begin{example}
Find $\dfrac{6x+8x^2-12}{2x+3}$ using long division. Check your work
\end{example}


\newpage

\begin{example}
Find $\dfrac{x^3-1}{x-1}$ using long division. Check your work.
\vspace{3in}
\end{example}

\begin{example}
Find $(2x^4 + 3x^3 - 7x - 10)\divisionsymbol (x^2-2x)$ using long division. Check your work.
\end{example}

\newpage

\subsubsection*{Synthetic Division}
Synthetic division is a method that allows us to divide two polynomials using a less verbose method. It works in a similar manner to long division, but is written out differently. One primary issue with synthetic division is that the divisor \textbf{must be written as $x-c$} -- it must be linear and the leading coefficient must be one. If your divisor is any other polynomial, you must use the long division method - so practice both!
\vspace{.15in}

\begin{example}
Find $(x^3 - 7x - 6) \divisionsymbol (x + 2)$ using synthetic division. Check your work.
\vspace{2.5in}
\end{example}

\begin{example}
Find $(x^5 + x^3 - 2) \divisionsymbol (x-1)$ using synthetic division. Check your work.
\end{example}

\newpage

\begin{example}
Find $\dfrac{5x^3 - 6x^2 + 3x + 11}{x-2}$ using synthetic division. Check your work.
\end{example}



\end{document}