\documentclass[addpoints,12pt]{exam}
\usepackage{amsmath}
\usepackage{amsthm}
\usepackage{amsfonts}
\usepackage{systeme}
\usepackage{graphicx}
\usepackage{caption}
\usepackage{xfrac}
\usepackage{physics}
\usepackage{microtype}
\usepackage{eulervm}
%\usepackage[framemethod=tikz]{mdframed}
\usepackage{thmtools}
\usepackage{etoolbox}
%\usepackage{fouriernc}
\usepackage{mdframed}
\usepackage[overload]{empheq}
\usepackage{adjustbox}
\usepackage{enumitem}
\usepackage[explicit]{titlesec}
% adds in \varnothing for empty set
\usepackage{amssymb}
% adds in formated SI units
%\usepackage{siunitx}
\usepackage{pgfplots}
\usepackage{multirow}
\usepackage{array}

\pagestyle{headandfoot}
\runningfootrule
\firstpageheadrule
\runningheadrule

\newcommand{\class}{Math 0097}
\newcommand{\sem}{2211}
\newcommand{\due}{}
\newcommand{\sect}{5.4}
\newcommand{\topic}{Polynomials in Several Variables}

\firstpageheader{\class}{\sect - \topic}{}
\runningheader{\class}{\sect - \topic}{}
\firstpagefooter{\class}{}{Page \thepage\ of \numpages}
\runningfooter{\class}{}{Page \thepage\ of \numpages}

\newif\ifprintselected
\printselectedtrue
%\printselectedfalse

\newenvironment{select}
{\ifprintselected
	\printanswers
	\fi
}
{}

\theoremstyle{definition}
\newtheorem{theorem}{Theorem}
%\newtheorem{example}{Example}[subsection]
%\newtheorem{definition}{Definition}
%\newmdtheoremenv{definition}{Definition}[subsection]
%\newmdtheoremenv{example}{Example}[subsection]
\AtBeginEnvironment{defn}{\begin{minipage}{\textwidth}}
\AtEndEnvironment{defn}{\end{minipage}}
%\AtBeginEnvironment{example}{\begin{minipage}{\textwidth}}
%\AtEndEnvironment{example}{\end{minipage}}
\newcommand{\iu}{{i\mkern1mu}}

\setlength{\gridsize}{5mm}
\setlength{\gridlinewidth}{0.1pt}

\printanswers
\DeclareMathSizes{12}{12}{12}{12}

%%%%%%%%%%%%%%%%%%%%%%%%
% Create bars around subsubsection
%%%%%%%%%%%%%%%%%%%%%%%%

\titleformat{\subsubsection}
   {\large\bfseries}% format
   {}% label
   {0pt}% sep
   {\titlerule \vspace{.1in} #1}% before code
      [{\titlerule[0.4pt]\vspace{.1in}}]% after code
\titlespacing{\subsubsection}
   {0pt}% left
   {0pt}% before sep
   {\baselineskip}% after sep
   
%%%%%%%%%%%%%%%%%%%%%%%
% Create line break after definition label
%%%%%%%%%%%%%%%%%%%%%%%   
\newtheoremstyle{break}
  {\topsep}{\topsep}%
  {}{}%\itshape
  {\bfseries}{}%
  {\newline}{}%
\theoremstyle{break}
\newmdtheoremenv{definition}{Definition}[subsection]
\theoremstyle{break}
\newtheorem{example}{Example}[subsection]

%%%%%%%%%%%%%%%%%%%%%%
% start document
% set section, subsection (use n-1 for sub)
%%%%%%%%%%%%%%%%%%%%%%


\begin{document}
\setcounter{section}{5}
\setcounter{subsection}{3}

\subsection{Polynomials in Several Variables}

\vspace{.15in}

\begin{example}
Evaluate the following for $x = -1$ and $y = 5$.
\[3x^3y + xy^2 + 5y + 6\]
\vspace{.75in}
\end{example}

\begin{definition}[Degree]
The degree of a monomial in several variables is the sum of the exponents. The degree of a polynomial in several variables is the largest degree of each term.
\end{definition}

\vspace{.15in}

\begin{example}
Find the degree of each term and then find the degree of the polynomial.
\[8x^4y^5 - 7x^3y^2 - x^2y - 6x + 11\]
\vspace{1in}
\end{example}

\begin{definition}[Like Terms]
Two monomials are like terms if the exponent of each variable in one term matches the corresponding exponent in the other term.
\end{definition}

\newpage

\begin{example}
Find the following:
\[ (-8x^2y - 3xy+6) + (10x^2y + 5xy - 10)\]
\vspace{1in}
\end{example}

\begin{example}
Find the following:
\[ (7x^3 - 10x^2y + 2xy^2 - 5) - (4x^3 - 12x^2y - 3xy^2 + 5)\]
\vspace{1in}
\end{example}

\begin{example}
Find the following:
\[ (6xy^3)(10x^4y^2)\]
\vspace{1in}
\end{example}

\begin{example}
Find the following:
\[ 6xy^2(10x^4y^5-2x^2y+3)\]
\end{example}

\newpage

\begin{example}
Find the following:
\[ (7x-6y)(3x-y)\]
\vspace{1in}
\end{example}

\begin{example}
Find the following:
\[ (2x+4y)^2\]
\vspace{1in}
\end{example}

\begin{example}
Find the following:
\[ (6xy^2 + 5x)(6xy^2 - 5x)\]
\vspace{1in}
\end{example}

\begin{example}
Find the following:
\[ (x-y)(x^2+xy+y^2)\]
\end{example}

\end{document}