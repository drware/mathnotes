\documentclass[addpoints,12pt]{exam}
\usepackage{amsmath}
\usepackage{amsthm}
\usepackage{amsfonts}
\usepackage{systeme}
\usepackage{graphicx}
\usepackage{caption}
\usepackage{xfrac}
\usepackage{physics}
\usepackage{microtype}
\usepackage{eulervm}
%\usepackage[framemethod=tikz]{mdframed}
\usepackage{thmtools}
\usepackage{etoolbox}
%\usepackage{fouriernc}
\usepackage{mdframed}
\usepackage[overload]{empheq}
\usepackage{adjustbox}
\usepackage{enumitem}
\usepackage[explicit]{titlesec}
% adds in \varnothing for empty set
\usepackage{amssymb}
% adds in formated SI units
%\usepackage{siunitx}
\usepackage{pgfplots}
\usepackage{multirow}
\usepackage{array}

\pagestyle{headandfoot}
\runningfootrule
\firstpageheadrule
\runningheadrule

\newcommand{\class}{Math 0097}
\newcommand{\sem}{2211}
\newcommand{\due}{}
\newcommand{\sect}{6.6}
\newcommand{\topic}{Solving Quadratics by Factoring}

\firstpageheader{\class}{\sect - \topic}{}
\runningheader{\class}{\sect - \topic}{}
\firstpagefooter{\class}{}{Page \thepage\ of \numpages}
\runningfooter{\class}{}{Page \thepage\ of \numpages}

\newif\ifprintselected
\printselectedtrue
%\printselectedfalse

\newenvironment{select}
{\ifprintselected
	\printanswers
	\fi
}
{}

\theoremstyle{definition}
\newtheorem{theorem}{Theorem}
%\newtheorem{example}{Example}[subsection]
%\newtheorem{definition}{Definition}
%\newmdtheoremenv{definition}{Definition}[subsection]
%\newmdtheoremenv{example}{Example}[subsection]
\AtBeginEnvironment{defn}{\begin{minipage}{\textwidth}}
\AtEndEnvironment{defn}{\end{minipage}}
%\AtBeginEnvironment{example}{\begin{minipage}{\textwidth}}
%\AtEndEnvironment{example}{\end{minipage}}
\newcommand{\iu}{{i\mkern1mu}}

\setlength{\gridsize}{5mm}
\setlength{\gridlinewidth}{0.1pt}

\printanswers
\DeclareMathSizes{12}{12}{12}{12}

%%%%%%%%%%%%%%%%%%%%%%%%
% Create bars around subsubsection
%%%%%%%%%%%%%%%%%%%%%%%%

\titleformat{\subsubsection}
   {\large\bfseries}% format
   {}% label
   {0pt}% sep
   {\titlerule \vspace{.1in} #1}% before code
      [{\titlerule[0.4pt]\vspace{.1in}}]% after code
\titlespacing{\subsubsection}
   {0pt}% left
   {0pt}% before sep
   {\baselineskip}% after sep
   
%%%%%%%%%%%%%%%%%%%%%%%
% Create line break after definition label
%%%%%%%%%%%%%%%%%%%%%%%   
\newtheoremstyle{break}
  {\topsep}{\topsep}%
  {}{}%\itshape
  {\bfseries}{}%
  {\newline}{}%
\theoremstyle{break}
\newmdtheoremenv{definition}{Definition}[subsection]
\theoremstyle{break}
\newtheorem{example}{Example}[subsection]

%%%%%%%%%%%%%%%%%%%%%%
% start document
% set section, subsection (use n-1 for sub)
%%%%%%%%%%%%%%%%%%%%%%


\begin{document}
\setcounter{section}{6}
\setcounter{subsection}{5}

\subsection{Solving Quadratics by Factoring}

\vspace{.15in}

\begin{definition}[Quadratic Equation]
A quadratic equation is an equation of the form $ax^2 + bx + c = 0$ where $a \neq 0$.
\end{definition}
\vspace{.15in}

\noindent We solve a quadratic equation by using the \emph{zero-product principle} to find the \emph{roots/zeroes/solutions/$x$-intercepts}.
\vspace{.15in}

\begin{definition}[Zero-Product Principle]
If $ab = 0$, then $a = 0$, $b=0$ or both $a$ and $b$ equal 0.
\end{definition}
\vspace{.15in}

\begin{example}
Solve the equation and check.
\[(2x+1)(x-4) = 0\]
\vspace{2in}
\end{example}

\begin{mdframed}
\textbf{Solving a Quadratic Equation by Factoring}
\begin{enumerate}
\item Rewrite the equation in standard form.
\item Factor using an appropriate method.
\item Apply the zero-product rule and set each factor equal to 0.
\item Solve the equations from the previous step.
\item Check your solutions.
\end{enumerate}
\end{mdframed}
\vspace{.15in}

\begin{example}
Solve for $x$: $x^2 - 6x + 5 = 0$
\vspace{2.25in}
\end{example}

\begin{example}
Solve for $x$: $4x^2 = 2x$
\vspace{2.25in}
\end{example}

\begin{example}
Solve for $x$: $x^2 = 10x - 25$
\end{example}
\newpage

\begin{example}
Solve for $x$: $16x^2 = 25$
\vspace{2.25in}
\end{example}

\begin{example}
Solve for $x$: $(x-5)(x-2) = 28$
\end{example}

\end{document}