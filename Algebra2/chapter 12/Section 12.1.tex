\documentclass[addpoints,12pt]{exam}
\usepackage{amsmath}
\usepackage{amsthm}
\usepackage{amsfonts}
\usepackage{systeme}
\usepackage{graphicx}
\usepackage{caption}
\usepackage{xfrac}
\usepackage{physics}
\usepackage{microtype}
\usepackage{eulervm}
%\usepackage[framemethod=tikz]{mdframed}
\usepackage{thmtools}
\usepackage{etoolbox}
%\usepackage{fouriernc}
\usepackage{mdframed}
\usepackage[overload]{empheq}
\usepackage{adjustbox}
\usepackage{enumitem}
\usepackage[explicit]{titlesec}
% adds in \varnothing for empty set
\usepackage{amssymb}
% adds in formated SI units
%\usepackage{siunitx}
%\usepackage{pgfplots}
\usepackage{multirow}
\usepackage{array}

\pagestyle{headandfoot}
\runningfootrule
\firstpageheadrule
\runningheadrule

\newcommand{\class}{Math 0098}
\newcommand{\sem}{2211}
\newcommand{\due}{}
\newcommand{\sect}{12.1}
\newcommand{\topic}{Exponential Functions}

\firstpageheader{\class}{\sect - \topic}{}
\runningheader{\class}{\sect - \topic}{}
\firstpagefooter{\class}{}{Page \thepage\ of \numpages}
\runningfooter{\class}{}{Page \thepage\ of \numpages}

\newif\ifprintselected
\printselectedtrue
%\printselectedfalse

\newenvironment{select}
{\ifprintselected
	\printanswers
	\fi
}
{}

\theoremstyle{definition}
\newtheorem{theorem}{Theorem}
%\newtheorem{example}{Example}[subsection]
%\newtheorem{definition}{Definition}
%\newmdtheoremenv{definition}{Definition}[subsection]
%\newmdtheoremenv{example}{Example}[subsection]
\AtBeginEnvironment{defn}{\begin{minipage}{\textwidth}}
\AtEndEnvironment{defn}{\end{minipage}}
%\AtBeginEnvironment{example}{\begin{minipage}{\textwidth}}
%\AtEndEnvironment{example}{\end{minipage}}
\newcommand{\iu}{{i\mkern1mu}}

\setlength{\gridsize}{5mm}
\setlength{\gridlinewidth}{0.1pt}

\printanswers
\DeclareMathSizes{12}{12}{12}{12}

%%%%%%%%%%%%%%%%%%%%%%%%
% Create bars around subsubsection
%%%%%%%%%%%%%%%%%%%%%%%%

\titleformat{\subsubsection}
   {\large\bfseries}% format
   {}% label
   {0pt}% sep
   {\titlerule \vspace{.1in} #1}% before code
      [{\titlerule[0.4pt]\vspace{.1in}}]% after code
\titlespacing{\subsubsection}
   {0pt}% left
   {0pt}% before sep
   {\baselineskip}% after sep
   
%%%%%%%%%%%%%%%%%%%%%%%
% Create line break after definition label
%%%%%%%%%%%%%%%%%%%%%%%   
\newtheoremstyle{break}
  {\topsep}{\topsep}%
  {}{}%\itshape
  {\bfseries}{}%
  {\newline}{}%
\theoremstyle{break}
\newmdtheoremenv{definition}{Definition}[subsection]
\theoremstyle{break}
\newtheorem{example}{Example}[subsection]

%%%%%%%%%%%%%%%%%%%%%%
% start document
% set section, subsection (use n-1 for sub)
%%%%%%%%%%%%%%%%%%%%%%


\begin{document}
\setcounter{section}{12}
\setcounter{subsection}{0}

\subsection{Exponential Functions}
\vspace{.15in}

\begin{definition}[Exponential Function]\mbox{}
\vspace{-.15in}
\begin{itemize}
\item $f(x) = b^x,\; b> 0,\; b\neq 1$
\item Domain: $\mathbb{R} = (-\infty,\infty)$
\item Range: $\mathbb{Z}^+ = (0,\infty)$
\item Exponential functions always have two guaranteed points: $(0,1)$ and $(1,b)$.
\item The $x$-axis is a horizontal asymptote.
\end{itemize}
\end{definition}
\vspace{.15in}

\subsubsection*{Graphs}
\begin{itemize}
\item If $0 < b < 1$, then the function exhibits \emph{exponential decay}.
\item If $b > 1$, then the function exhibits \emph{exponential growth}.
\end{itemize}

\newpage

\begin{example}
Identify the base, make a table of values, and graph the function.
\[f(x) = 2^x\]
\vspace{3.25in}
\end{example}
\begin{example}
Identify the base, make a table of values, and graph the function.
\[f(x) = 3^{-x}\]
\end{example}

\newpage

\subsubsection*{Base "e"}
\vspace{.15in}
\begin{definition}[Euler's Number]
$e \approx 2.718281827\dots$\\
$e = \lim_{n\rightarrow\infty}\left(1+\dfrac{1}{n}\right)^n$
\end{definition}

\vspace{.15in}

\begin{example}
The exponential function
\[f(x) = 1145e^{0.0325x}\]
gives the population of gray wolves $x$ years after 1978.

\begin{enumerate}
\item What is the approximate gray wolf population in 1979?
\vspace{1in}
\item What about in 1990?
\vspace{1in}
\item What about in 2020?
\vspace{1in}
\end{enumerate}
\end{example}

\newpage

\subsubsection*{Compound Interest}
\noindent There are two main types of compounding interest that we'll cover in this course.
\vspace{.15in}
\begin{mdframed}
\textbf{Periodically Compounding}\\
\[A = P\left(1+\dfrac{r}{n}\right)^{nt}\]
\begin{itemize}
\item $A = $ balance/amount
\item $P = $ principal/initial amount
\item $r = $ interest rate (as a decimal)
\item $n = $ number of times compounded in \emph{one} time period
\item $t = $ the number of time periods
\end{itemize}
\end{mdframed}
\vspace{.15in}
\begin{mdframed}
\textbf{Continuously Compounding}\\
\[A = Pe^{rt}\]
\begin{itemize}
\item $A = $ balance/amount
\item $P = $ principal/initial amount
\item $r = $ interest rate (as a decimal)
\item $t = $ the number of time periods
\end{itemize}
\end{mdframed}

\newpage

\begin{example}
Say that I invested \$10,000 at 8\% interest for 5 years. Compare the ending balances if compounded quarterly, twice monthly, and continuously.
\begin{enumerate}
\item Quarterly:
\vspace{2.25in}
\item Twice monthly:
\vspace{2.25in}
\item Continuously:
\end{enumerate}
\end{example}

\end{document}